\chapter*{Kurzfassung}
\addcontentsline{toc}{chapter}{Kurzfassung}

In den vergangenen Jahren hat die Idee des intelligenten Wohnens zunehmend an Bedeutung gewonnen. Ein intelligenter Wohnraum (ein \emph{Smart Home}) verfügt über eine Art Intelligenz, die es ihm ermöglicht, seinen BewohnerInnen Tätigkeiten abzunehmen. Das Ziel ist, den Bewoh\-nerInnen mehr Komfort zu bieten und gleichzeitig Energieverbrauch und Kosten zu senken.

Im Kontext eines Projekts, welches die Entwicklung von Smart Homes zum Ziel hat, wird in dieser Masterarbeit eine \emph{OWL}-Ontologie für Wetterinformation entworfen, die Daten sowohl über die aktuelle Wetterlage als auch über Vorhersagen enthält. Die von dieser Ontologie beschriebenen Daten werden es Smart Home-Systemen ermöglichen, Entscheidungen auf Basis der aktuellen und der zukünftigen Wetterverhältnisse zu treffen.

Zunächst wird die Masterarbeit untersuchen, auf welche Art und Weise Wetterdaten in Smart Homes verwendet werden können. Weiters werden mögliche Quellen für Wetterdaten analysiert. Die wichtigsten Quellen werden über das Internet abrufbare Wetterdienste sein; optional können lokale Wetterstationen weitere Daten über die aktuelle Wetterlage zur Verfügung stellen. Eine Auswahl an über das Internet verfügbaren Quellen für Wetterdaten wird hinsichtlich ihrer Eignung zur Verwendung in Smart Homes untersucht.

Anschließend werden bereits existierende Ontologien bezüglich ihrer Struktur, ihren Vorteilen und ihren Nachteilen untersucht, um Ideen zu sammeln, die wiederverwendet werden können. Einige bekannte Verfahren, um neue Ontologien von Grund auf zu erstellen, werden im Detail erläutert.

Die Arbeit wird \methontology, das von diesen Verfahren am besten geeignete, verwenden, um \smarthomeweather zu entwerfen, eine \eacs{OWL}-Ontologie, die sowohl die zur Verfügung stehenden Wetterdaten als auch die Konzepte abdeckt, die notwendig sind, um in Smart \mbox{Homes} wetterbezogene Aufgaben zu erledigen; gleichzeitig wird darauf geachtet, einfaches und effizientes \emph{\acs{OWL}-Reasoning} zu ermöglichen.

Schließlich wird \emph{Weather Importer} entwickelt, eine Java-Applikation, die Daten von Wetterdiensten und lokalen Wetterstationen abruft und sie so transformiert, dass sie mit der \emph{Smart\-Home\-Weather}-Ontologie verwendet werden können.