\chapter*{Kurzfassung}

In den vergangenen Jahren hat die Idee des \emph{Intelligenten Wohnens} an Bedeutung gewonnen. Ein intelligenter Wohnraum (\smarthome) besitzt verfügt über eine Art Intelligenz, die es ihm ermöglicht, seinen BewohnerInnen Tätigkeiten abzunehmen. Das Ziel ist, den BewohnerInnen mehr Komfort zu bieten, gleichzeitig werden aber auch Energieverbrauch und Kosten gesenkt.

Ziel des \thinkhome-Projekts ist es, intelligenten Wohnraum zu entwerfen, der in der Lage ist, den Energieverbrauch zu vermindern. Damit es Steuerungsentscheidungen treffen kann, ist \thinkhome in der Lage, Eingabedaten aus verschiedenen Quellen zu verarbeiten.

Im Kontext des \thinkhome-Projekts soll diese Masterarbeit eine \emph{OWL}-Ontologie für Wetterinformation entwerfen, die Daten sowohl über die aktuelle Wetterlage als auch über Vorhersagen enthält. Die von dieser Ontologie beschriebenen Daten werden es dem \thinkhome-System ermöglichen, Entscheidungen auf Basis der aktuellen und der zukünftigen Wetterverhältnisse zu treffen.

Zunächst wird die Masterarbeit untersuchen, auf welche Art und Weise Wetterdaten in \thinkhome verwendet werden können. Weiters werden mögliche Quellen für Wetterdaten analysiert. Die wichtigsten Quellen werden über das Internet abrufbare Wetterdienste sein. Optional können lokale Wetterstationen weitere Daten über die aktuelle Wetterlage zur Verfügung stellen. Eine Auswahl an über das Internet verfügbaren Quellen für Wetterdaten werden hinsichtlich ihrer Eignung zur Verwendung im \thinkhome-Projekt untersucht.

Anschließend werden bereits existierende Ontologien bezüglich ihrer Struktur, ihren Vorteilen und ihren Nachteilen untersucht, um Ideen zu sammeln, die wiederverwendet werden können. Einige bekannte Verfahren, um neue Ontologien von Grund auf zu erstellen, werden im Detail erläutert.

Die Arbeit wird \methontology, das von diesen Verfahren am besten geeignete, verwenden, um \smarthomeweather zu entwerfen, eine \textit{OWL}-Ontologie, die sowohl die zur Verfügung stehenden Wetterdaten als auch die Konzepte abdeckt, die notwendig sind, um in \thinkhome wetterbezogene Aufgaben zu erledigen; gleichzeitig wird darauf geachtet, dass einfaches und effizientes \emph{OWL-Reasoning} möglich ist.

Schließlich wird \emph{Weather Importer} entwickelt, eine Java-Applikation, die Daten von Wetterdiensten und lokalen Wetterstationen abruft und sie so transformiert, dass sie mit der \smarthomeweather-Ontologie verwendet werden können.