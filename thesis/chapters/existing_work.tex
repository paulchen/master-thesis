\chapter{Existing work}
\label{ch:existing_work}

The first part of this chapter covers the foundations the work of this thesis builds upon. It gives introductions into all relevant topics, e.g. ontologies and associated modelling languages such as \eacs{RDF}, \eacs{RDFS}, and \eacs{OWL}. Furthermore, it discusses the basic principles of \thinkhome which serves as an example of a smart home that uses a knowledge base built upon a set of ontologies.

The second part sheds light on a selection of existing work regarding weather data in smart homes and ontologies for weather data. Additionally, some ontologies which may be reused in an ontology for weather data are discussed. The final section of this chapter reviews all ontologies that are presented and identifies elements can be used in the \smarthomeweather ontology.

\section{Foundations}

This chapter presents the foundations the work in this thesis builds upon: The concept of an ontology, the \emph{Web Ontology Language} (\eacs{OWL}) and smart homes utilising ontologies using the example of \thinkhome.

\subsection{Ontologies}
\label{subsec:ontologies}

In computer science, an ontology represents knowledge as a set of concepts in a certain domain and relationships between pairs of concepts~\cite{OntologiesSilverBullet}. The basic elements of ontologies -- concepts, properties, and relations -- comprise a shared vocabulary which can be used to model a certain domain.

Each object that is mapped into an ontology is represented by an \emph{individual} (also known as \emph{object}). Individuals of the same type can be defined to be \emph{instances} of \emph{concepts} (also called \emph{classes}). Both classes and individuals can have \emph{attributes} that specify their characteristics and properties. Two arbitrary classes or individuals can be related to each other via a \emph{relation}.

Furthermore, ontologies may contain \emph{function terms} (structures formed from relations that can be used in place of terms in statements), \emph{restrictions} (descriptions of what must be true for additional knowledge to be accepted), \emph{rules} (statements in if-then notation that describe logical inferences that can be drawn), \emph{axioms} (core knowledge of the ontology that is known to be true), and \emph{events} (changes to attributes or relations).

\vspace{1em}

\begin{figure}
\centering
\includegraphics[width=\textwidth]{figures/diagrams/ontology_example.pdf}
\caption[Example of a simple ontological model]{Example of a simple ontological model.}
\label{fig:ontology_example}
\end{figure}

Figure~\ref{fig:ontology_example} illustrates the aforementioned elements in a simple ontology: \emph{Person}, \emph{Man}, and \emph{Woman} are concepts. \emph{is subclass of} is a property that defines one concept to be a \emph{sub-concept} of another concept (i.e. \emph{B is subclass of A} states that every instance of \emph{B} is also an instance of \emph{A}). \emph{Brian}, \emph{Maria}, and \emph{Sarah} are individuals; each of them is a instance of a concept which is related to it via the \emph{is a} relation. \emph{age} is a property of the individuals \emph{Brian}, \emph{Maria}, and \emph{Sarah} and \emph{has parent} is a relation which associates two individuals to each other.

This model states the following facts: Men and women are persons. Brian (age 37) is a man while Maria (age 12) and Sarah (age 32) are women. Maria has two parents, Brian and Sarah.

\vspace{1em}

An important feature of an ontology is the support of automatic reasoning to deduce facts that are not explicitly stated in the data model from the given information. In order to make reasoning possible, the semantics of data models in ontologies (including \eacs{OWL}) are often based on \emph{Description Logics}~\cite{OWL,SROIQ}. These are a family of logics consisting of decidable parts of first-order predicate logic~\cite{FirstOrderLogic}.

In the above example, because \emph{Man} and \emph{Woman} are sub-concepts of \emph{Person}, \emph{Brian}, \emph{Maria}, and \emph{Sarah} are instances of \emph{Person}. One could define a relation \emph{has child} that is an inverse property of \emph{has parent}, i.e. \emph{A has child B} if and only if \emph{B has parent A}; then the statements \emph{Brian has child Maria} and \emph{Sarah has child Maria} can be deduced.

If a concept \emph{Mother} is defined as a \emph{Woman} who has at least one child, \emph{Sarah} can be inferred to be an instance of this concept; the same works for an analogously defined concept \emph{Father} with \emph{Brian} being an instance of. Furthermore, concepts \emph{Daughter} and \emph{Son} can be defined (someone who has at least one parent and is a \emph{Woman} or a \emph{Man}, respectively). Then \emph{Maria} can be inferred to be a \emph{Daughter}. Additionally, properties like \emph{has mother}, \emph{has father}, \emph{has son}, \emph{has daughter} etc. can be defined.

\vspace{1em}

Another core principle of an ontology is reusability~\cite{reuse1,reuse2}. In order to share knowledge across various systems and to ensure interoperability of these systems, ontologies are often reused within other ontologies. Besides the simplification of knowledge exchange, ontology reuse tries to avoid duplicate work and reduces the work that is necessary to create a new ontology for a domain.

\vspace{1em}

Ontologies are expressed using formal languages. There are many of these \emph{ontology languages} such as \eacs{KIF} (\emph{Knowledge Interchange Format})~\cite{KIFReference}, \emph{DAML+OIL}~\cite{DAML+OIL} -- a successor to \eacs{DAML}~\cite{DAML} (\emph{\eacs{DARPA} Agent Markup Language}) and \eacs{OIL} (\emph{Ontology Inference Layer})~\cite{OIL} that combines features of both --, \eacs{RDFS} (\emph{RDF Schema})~\cite{RDFS}, and \eacs{OWL}~\cite{OWL}. The latter, the \emph{Web Ontology Language}, is used in the \smarthomeweather project; thus, Section~\ref{subsec:owl} gives a brief introduction into \eacs{OWL}.

Over the years, many ontologies have been developed. Some of them define concepts and relations that can be found across many knowledge domains. These ontologies are often imported into other ontologies in order to minimise the effort of creating new ontologies and to simplify interoperability of ontologies. Examples for such ontologies that are implemented in \eacs{OWL} or \emph{\eacs{RDF} Schema} are:
\begin{itemize}
  \item \textbf{\eacs{DOAP}}: \eacs{DOAP} (\emph{Description of a Project})~\cite{DOAP} is an vocabulary for describing software projects.
  \item \textbf{Dublin Core}: \emph{Dublin Core}~\cite{DublinCoreWeb,rfc5013} is a vocabulary for describing metadata of web documents, physical resources (documents, books etc.) and other objects such as works of art.
  \item \textbf{\eacs{FOAF}}: \eacs{FOAF} (\emph{Friend of a Friend})~\cite{FOAFWeb,FOAF} is an ontology for describing social networks. It models persons, the relations between them, their activities and their relations to other objects.
  \item \textbf{\eacs{SIOC}}: The \eacs{SIOC} (\emph{Semantically-Interlinked Online Communities})~\cite{SIOC} is a technology built around an ontology for encoding information from Internet discussion methods (message boards, blogs, mailing lists etc.).
  \item \textbf{\eacs{SKOS}}: \eacs{SKOS} (\emph{Simple Knowledge Organization System})~\cite{SKOSWeb,SKOS} is a data model for sharing and linking knowledge organisation systems (thesauri, taxonomies, classification schemes and subject heading systems).
  \item \textbf{\eacs{UMBEL}}: \eacs{UMBEL} (\emph{Upper Mapping and Binding Exchange Layer})~\cite{UMBEL} is an approach towards interoperability of content on the Web. It is a vocabulary for the construction of ontologies being designed for interoperation and provides a reference structure of $25,000$ concepts that provide a scaffolding to link and interoperate datasets and domain vocabularies. Ontologies that define very general concepts which are shared between many knowledge domains are termed \emph{upper-level ontologies}.~\cite{upper_ontology}
\end{itemize}

Many standards such as \eacs{RDF}, \eacs{RDFS}, or \eacs{OWL} and some of the above ontologies have been published in the context of the \emph{\eacs{W3C} Semantic Web Activity}~\cite{w3c_semantic_web} by the \emph{World Wide Web Consortium} (\eacs{W3C})~\cite{W3C}. The \emph{Semantic Web} is an approach to enrich the World Wide Web with machine-interpretable metadata using the technologies described in this section in order to allow better interoperability between Web pages and to ease knowledge sharing~\cite{semantic_web}.

\subsection{OWL}
\label{subsec:owl}

The \emph{Resource Description Framework} (\eacs{RDF}) is a standard model for knowledge representation~\cite{RDF}. It is specified in a set of recommendations by the \eacs{W3C}.

In \eacs{RDF}, the term \emph{resources} is used for instances. Each resource can have an arbitrary number of \emph{properties}, i.e. \emph{attributes} that associate \emph{literate} values (e.g. numerical values, strings) to the resource or \emph{relations} that link this resource to other resources. Resources and properties are expressed using \emph{statements} (\emph{triples}) which consist of three parts called \emph{subject}, \emph{predicate} and \emph{object}. To identify resources and properties, \eacs{RDF} uses \eacss{URI} (\emph{Unified Resource Locators})\footnote{\eacss{URI} are strings defined by \emph{\eacs{RFC} 3986}~\cite{rfc3986} that uniquely identify things.}. In case a resource does not have an identifier, it is a \emph{blank node}.

\begin{figure}
\centering
\includegraphics[width=.9\textwidth]{figures/diagrams/rdf_example.pdf}
\caption[Example of a simple \eacs{RDF} model]{Example of a simple \eacs{RDF} model.}
\label{fig:rdf_example}
\end{figure}

Figure~\ref{fig:rdf_example} depicts a simple example for a piece of knowledge from the domain of weather data expressed using \eacs{RDF}: The resource \texttt{weather:\hspace{0pt}Current\hspace{0pt}Weather} represents the current state of the weather. The property \texttt{geo:\hspace{0pt}has\hspace{0pt}Location} links the \texttt{weather:\hspace{0pt}Current\hspace{0pt}Weather} to the resource \texttt{geo:\hspace{0pt}Vienna} which represents the city of Vienna, Austria; i.e. \texttt{weather:\hspace{0pt}Current\hspace{0pt}Weather} describes the weather for Vienna. \texttt{weather:\hspace{0pt}weather\hspace{0pt}State} has two more properties, \texttt{weather:\hspace{0pt}has\hspace{0pt}Temperature\hspace{0pt}Value} and \texttt{weather:\hspace{0pt}has\hspace{0pt}Humidity\hspace{0pt}Value}, which link two literal values to the resource: a temperature value of \SI{18.2}{\celsius} and a relative humidity value of $72 \%$.
The type \texttt{xsd:\hspace{0pt}float} of both literals is defined in \emph{XML Schema}~\cite{xml-schema,xml-schema-datatypes}, one of the several \eacs{XML} schema languages available that define the structure of \eacs{XML} documents (\emph{Extensible Markup Langauge})~\cite{XML}.

The complete \eacs{URI} of \texttt{weather:\hspace{0pt}Current\hspace{0pt}Weather} is \texttt{http://example.org/\hspace{0pt}weather\#\hspace{0pt}CurrentWeather} and the \eacs{URI} of \texttt{geo:\hspace{0pt}Vienna} is \texttt{http://example.org/geo\#Vienna}. As in \eacs{XML}, substrings at the beginning of \eacss{URI} may be replaced by \emph{prefixes} to avoid frequent recurrences of the same strings. The part of the \eacss{URI} replaced by the prefix is called a \emph{namespace} which is used to group \eacss{URI} for elements from the same source together; e.g. all concepts, properties, and individuals defined by \smarthomeweather have identifiers in the same namespace.

For the above example, the prefix \texttt{weather} has been defined to replace the string \texttt{http://\hspace{0pt}example.org/weather\#} and \texttt{geo} replaces \texttt{http://example.org/geo\#}. This results in the identifiers \texttt{weather:CurrentWeather} and \texttt{geo:Vienna} that can be found in Figure~\ref{fig:rdf_example}.

For expressing the data that is represented by an \eacs{RDF} model, several serialization formats are available. The \eacs{RDF} recommendation is based on \emph{RDF/XML} which maps the \eacs{RDF} model to an \eacs{XML} document~\cite{RDF_XML}. The representation of the above example in \emph{RDF/XML} can be seen in Listing~\ref{listing:rdfxml_example}. As \emph{RDF/XML} is a rather verbose format which may be difficult to read for humans, the \eacs{N3} (\emph{Notation3}) representation for \eacs{RDF} is available which was developed with human-readability in mind~\cite{Notation3}. \emph{Notation3} incorporates some syntax features that go beyond the expressive power of \eacs{RDF}. A subset of \emph{Notation3} named \eacs{Turtle} (\emph{Terse RDF Triple Language}) is available that is limited to the features required to map \eacs{RDF} models~\cite{Turtle}. Listing~\ref{listing:turtle_example} shows the above example in \eacs{Turtle} syntax.


\begin{mintlisting}
\begin{tikzpicture}\node[draw,rounded corners=0pt,inner sep=0pt,shade,top color=white,bottom color=listingBottom]{\begin{minipage}{\textwidth}\hspace{2mm}\begin{minipage}{.95\textwidth}\vspace{3mm}
\begin{minted}{xml}
<?xml version="1.0"?>
<rdf:RDF xmlns:xsd="http://www.w3.org/2001/XMLSchema#"
      xmlns:geo="http://example.org/geo#"
      xmlns:rdf="http://www.w3.org/1999/02/22-rdf-syntax-ns#"
      xmlns:weather="http://example.org/weather#">
  <rdf:Description rdf:about="http://example.org/weather#CurrentWeather">
    <weather:hasTemperatureValue
	  rdf:datatype="http://www.w3.org/2001/XMLSchema#float">
      18.2
    </weather:hasTemperatureValue>
    <weather:hasHumidityValue
	  rdf:datatype="http://www.w3.org/2001/XMLSchema#float">
      .72
    </weather:hasHumidityValue>
    <geo:hasLocation rdf:resource="http://example.org/geo#Vienna" />
  </rdf:Description>
</rdf:RDF>
\end{minted}
\vspace{3mm}\end{minipage}\end{minipage}};\end{tikzpicture}

\caption[\eacs{RDF} example in \emph{RDF/XML} syntax]{\eacs{RDF} example from Figure~\ref{fig:rdf_example} encoded in \emph{RDF/XML} syntax.}
\label{listing:rdfxml_example}
\end{mintlisting}


\begin{mintlisting}
\begin{tikzpicture}\node[draw,rounded corners=0pt,inner sep=0pt,shade,top color=white,bottom color=listingBottom]{\begin{minipage}{\textwidth}\hspace{2mm}\begin{minipage}{.95\textwidth}\vspace{3mm}
\begin{minted}{turtle}
@prefix rdf: <http://www.w3.org/1999/02/22-rdf-syntax-ns#> .
@prefix weather: <http://example.org/weather#> .
@prefix geo: <http://example.org/geo#> .
@prefix xsd: <http://www.w3.org/2001/XMLSchema#> .

weather:CurrentWeather weather:hasTemperatureValue "18.2"^^xsd:float ;
    weather:hasHumidityValue ".72"^^xsd:float .

weather:CurrentWeather geo:hasLocation geo:Vienna .
\end{minted}
\vspace{3mm}\end{minipage}\end{minipage}};\end{tikzpicture}

\caption[\eacs{RDF} example in \emph{Turtle} syntax]{\eacs{RDF} example from Figure~\ref{fig:rdf_example} encoded in \eacs{Turtle} syntax.}
\label{listing:turtle_example}
\end{mintlisting}

\vspace{1em}

\eacs{RDF} \emph{schema} (\eacs{RDFS}) is a recommendation by the \eacs{W3C} that builds upon \eacs{RDF}~\cite{RDFS}. It introduces a set of concepts and properties adding features that go beyond the expressive power of \eacs{RDF}.

All things described by \eacs{RDF} are instances of \texttt{rdfs:Resource}. Concepts -- which are introduced by \eacs{RDFS} -- are instances of \texttt{rdfs:Class}, and properties are instances of \texttt{rdfs:\hspace{0pt}Property}. Other concepts introduced by \eacs{RDFS} are \texttt{rdfs:\hspace{0pt}Literal}, \texttt{rdfs:\hspace{0pt}Datatype} and \texttt{rdfs:\hspace{0pt}XMLLiteral}.

The property \texttt{rdfs:domain} states that any resource that has a given property is an instance of one or more classes; the property \texttt{rdfs:range} states that the value of a property is an instance of one or more classes. \texttt{rdf:type} (often abbreviated with \texttt{a}) is used to state that a resource is an instance of a class. Hierarchies of classes can be constructed using the property \texttt{rdfs:subClassOf}: \texttt{C1 rdfs:subClassOf C2} states that any instance of \texttt{C2} is also an instance of \texttt{C1}. \texttt{rdfs:subPropertyOf} is an equivalent that is used for declaring hierachies of properties. Other properties defined by \eacs{RDFS} are \texttt{rdfs:label} and \texttt{rdfs:comment}.

\begin{figure}
\centering
\includegraphics[width=.9\textwidth]{figures/diagrams/rdfs_example.pdf}
\caption[Example of a simple \eacs{RDFS} model]{Example of a simple \eacs{RDFS} model.}
\label{fig:rdfs_example}
\end{figure}

Figure~\ref{fig:rdfs_example} shows the example from Figure~\ref{fig:rdf_example}, enriched by some elements that are introduced by \eacs{RDFS}. The data model introduces two classes, \texttt{weather:\hspace{0pt}Weather\hspace{0pt}State} and \texttt{geo:\hspace{0pt}Location}. \texttt{weather:\hspace{0pt}CurrentWeather} and \texttt{geo:\hspace{0pt}Vienna} are instances of \texttt{weather:\hspace{0pt}Weather\hspace{0pt}State} and \texttt{geo:\hspace{0pt}Location}, respectively. \texttt{geo:\hspace{0pt}has\hspace{0pt}Location} is a property with domain \texttt{weather:\hspace{0pt}Weather\hspace{0pt}State} and range \texttt{geo:\hspace{0pt}Location}.

\eacs{RDFS} comes with reasoning support~\cite{RDF_semantics}; e.g. in the above example, the statements

\noindent\begin{tikzpicture}\node[draw,rounded corners=0pt,inner sep=0pt,shade,top color=white,bottom color=listingBottom]{\begin{minipage}{\textwidth}\hspace{2mm}\begin{minipage}{.95\textwidth}\vspace{3mm}
\begin{minted}{turtle}
weather:CurrentWeather rdf:type weather:WeatherState .
geo:Vienna rdf:type geo:Location .
\end{minted}
\vspace{3mm}\end{minipage}\end{minipage}};\end{tikzpicture}

can be removed without loss of knowledge. A reasoner can deduce them from these statements:

\noindent\begin{tikzpicture}\node[draw,rounded corners=0pt,inner sep=0pt,shade,top color=white,bottom color=listingBottom]{\begin{minipage}{\textwidth}\hspace{2mm}\begin{minipage}{.95\textwidth}\vspace{3mm}
\begin{minted}{turtle}
weather:CurrentWeather geo:hasLocation geo:Vienna .
geo:location rdfs:domain weather:WeatherState .
geo:location rdfs:range geo:Location .
\end{minted}
\vspace{3mm}\end{minipage}\end{minipage}};\end{tikzpicture}

Many of the concepts and properties defined by \eacs{RDFS} are included in the \emph{Web Ontology Language} (\eacs{OWL}), a more expressive ontology language than \eacs{RDFS} which is based on \eacs{RDF} and \eacs{RDFS}. \eacs{OWL} is developed by the \emph{OWL Working Group}~\cite{OWL-working-group} of the \eacs{W3C}. \eacs{OWL} 1 was first published in July 2002 as a \emph{working draft} and became a \eacs{W3C} \emph{recommendation} in February 2004\footnote{Both \emph{working draft} and \emph{recommendation} are \emph{maturity levels} proposed by the \eacs{W3C} that indicate the state of their technical reports~\cite{w3c-process}.}~\cite{OWL1}; the first \emph{working draft} of \eacs{OWL} 2 in March 2009 and the \eacs{W3C} \emph{recommendation} of \eacs{OWL} 2 in October 2009 with a second edition being finally released in December 2012~\cite{OWL}.
\eacs{OWL} 2 remains fully compatible to \eacs{OWL} 1, i.e. all \eacs{OWL} 1 ontologies are \eacs{OWL} 2 ontologies as well, with unchanged semantics.

Compared to \eacs{RDFS}, \eacs{OWL} introduces the following elements (among others which are omitted here):

\begin{itemize}
  \item Properties are instances of \texttt{owl:ObjectProperty} or \texttt{owl:DatatypeProperty} (or both); the property is termed \emph{object property} or \emph{datatype property}, respectively. An \emph{object property} links an individual to another individual while a \emph{datatype property} links an individual to a literal value.
  \item The property \texttt{owl:equivalentClass} is used to state that two classes are equivalent while \texttt{owl:allDisjointClasses} states that there is no individual that is an instance of more than one class from the defined set of classes.
  \item Some ontology languages include a \emph{Unique Name Assumption} which states that two different names always refer to different entities in the world~\cite{unique_name_assumption}. \eacs{OWL} does not make this assumption, but provides the properties \texttt{owl:sameAs} and \texttt{owl:differentFrom} that are used to explicitely state that two individuals are the same individual or two individuals can never be the same individual.
  \item Using the properties \texttt{owl:intersectionOf}, \texttt{owl:unionOf}, and \texttt{owl:\hspace{0pt}complement\hspace{0pt}Of} can be used to describe complex classes in a notation borrowed from set theory (e.g. if there are two classes, \texttt{Man} and \texttt{Woman}, the class \texttt{Person} can be defined as the class union of them). % TODO wtf; annotation: "...if no other classes exist"?!
  \item Using the properties \texttt{owl:allValuesFrom}, \texttt{owl:someValuesFrom}, and \texttt{owl:\hspace{0pt}has\hspace{0pt}Value} classes can be defined based on the values or classes of their properties.
  \item Using the properties \texttt{owl:minCardinality}, \texttt{owl:maxCardinality}, and \texttt{owl:\hspace{0pt}cardinality}, the cardinality of properties per class can be limited.
  \item Properties can have various characteristics: A property can be inverse property of another property (\texttt{owl:inverseOf}) and two properties can be disjoint (\texttt{owl:property\hspace{0pt}Disjoint\hspace{0pt}With}). A property can be reflexive (it relates everything to itself, \texttt{owl:\hspace{0pt}Reflexive\hspace{0pt}Property}), irreflexive (no individual can be related to itself, \texttt{owl:\hspace{0pt}Irreflexive\hspace{0pt}Property}), functional (every individual can be linked to at most one other individual, \texttt{owl:\hspace{0pt}Functional\hspace{0pt}Property}), or inverse functional (the inverse property is functional, \texttt{owl:\hspace{0pt}Inverse\hspace{0pt}Functional\hspace{0pt}Property}).
\end{itemize}

Reasoning in \eacs{OWL} respects the \emph{open world assumption}~\cite{open_world_assumption1,open_world_assumption2}. If some statement cannot be inferred, it is not allowed to assume that this statement is false. Hence, reasoning in \eacs{OWL} is \emph{monotonic}: Adding more information to a model cannot cause anything become false that has previously known to be true, and vice versa~\cite{MonotonicReasoning}.

Queries on \eacs{RDF}, \eacs{RDFS}, and \eacs{OWL} models are often performed using \eacs{SPARQL}, a query language for \eacs{RDF}~\cite{SPARQL}. Listing~\ref{listing:sparql_example} shows an example for the use of \eacs{SPARQL} to query all temperature values known for Vienna, Austria in the \eacs{RDFS} model depicted in Figure~\ref{fig:rdfs_example}.

\begin{mintlisting}
\begin{tikzpicture}\node[draw,rounded corners=0pt,inner sep=0pt,shade,top color=white,bottom color=listingBottom]{\begin{minipage}{\textwidth}\hspace{2mm}\begin{minipage}{.95\textwidth}\vspace{3mm}
\begin{minted}{sparql}
@prefix weather: <http://example.org/weather#>
@prefix geo: <http://example.org/geo#>

SELECT ?temperature
WHERE {
    ?state a weather:WeatherState.
    ?state geo:Location geo:Vienna.
    ?state weather:hasTemperatureValue ?temperature.
}
\end{minted}
\vspace{3mm}\end{minipage}\end{minipage}};\end{tikzpicture}

\caption[Example \eacs{SPARQL} query]{\eacs{SPARQL} code to query all known temperature values in the model from Figure~\ref{fig:rdf_example}.}
\label{listing:sparql_example}
\end{mintlisting}

\eacs{SWRL} is a language for \eacs{OWL} which is used to express rules~\cite{SWRL}. \eacs{SWRL} can be used to define relations that may be difficult or impossible to define using \eacs{OWL} alone, e.g. due to the \emph{open world assumption}. Given a model that contains a concept \texttt{WeatherState} for a certain time and a intransitive property \texttt{hasNextWeatherState} which relates two consecutive instances of \texttt{WeatherState} to each other, Listing~\ref{listing:swrl_example} gives an example demonstrating how the semantics transitive property \texttt{hasLaterWeatherState} can be defined based on \texttt{hasNextWeatherState}.

\begin{mintlisting}
\begin{tikzpicture}\node[draw,rounded corners=0pt,inner sep=0pt,shade,top color=white,bottom color=listingBottom]{\begin{minipage}{\textwidth}\hspace{2mm}\begin{minipage}{.95\textwidth}\vspace{3mm}
\begin{minted}{text}
hasNextWeatherState(?s1, ?s2) ⇒ hasLaterWeatherState(?s1, ?s2)
hasLaterWeatherState(?s1, ?s2) ∧ hasLaterWeatherState(?s2, ?s3)
                 ⇒ hasLaterWeatherState(?s1, ?s3)
\end{minted}
\vspace{3mm}\end{minipage}\end{minipage}};\end{tikzpicture}

\caption[Example \eacs{SWRL} rule]{\eacs{SWRL} example defining a transitive property based on an intransitive one.}
\label{listing:swrl_example}
\end{mintlisting}

\eacs{OWL} ontologies are often designed using semantic editors like \protege~\cite{protege}. Common reasoners for \eacs{OWL} include \emph{Pellet}~\cite{pellet}, \emph{RacerPro}~\cite{RacerPro}, \emph{FaCT++}~\cite{factplusplus} and \emph{HermiT}~\cite{hermit}. \protege and \emph{Pellet} are used to develop the \smarthomeweather ontology in Chapter~\ref{ch:smarthomeweather_ontology}.

\section{ThinkHome}

Section~\ref{sec:motivation} gives a short introduction into the basic ideas of \thinkhome which serves as an example for an ontology-based smart home infrastructure. This section goes further into detail into the two main components that form the \thinkhome system: a comprehensive knowledge base and a multi-agent system.

The knowledge base is organised into six sub-categories~\cite{CR2011-TH_Journal,CR2010-DEST_ThinkHome}: \emph{Building} (structure of the building), \emph{Actor} (human system users and software agents), \emph{Comfort} (parameters that affect the users' well-being), \emph{Energy} (energy providers), \emph{Process} (activities), and \emph{Resource} (available equipment). Some parts of the knowledge base are static data (data that is seldom changed, e.g. the structure of the building, user preferences, control rules) while others are dynamic data (data that changes frequently, e.g. the current state of the building, the current state of external influences, recent measurements from sensors, current states of actuators). The knowledge base is implemented in \eacs{OWL}. This enables the use of the reasoning capabilities offered by semantic reasoners. With \eacs{SPARQL}, a powerful query language for \eacs{OWL} models is available that is suitable for accessing \thinkhome's knowledge base.

A multi-agent system is both a software paradigm and a method supporting distributed intelligence, interaction and cooperation to work towards predefined goals~\cite{MultiAgentSystems}. \thinkhome incorporates a multi-agent system to implement advanced control strategies. \thinkhome's \emph{agent society} consists of various different agents, each serving a specific purpose: As the core of the agent system, the \emph{Control Agent} is responsible for executing control strategies; it obtains data from various sources to get a global view of the system state, then calculates appropriate control strategies and initiates its execution. Other agents are the \emph{User Agent} which acts on behalf of a user to provide her with a comfortable environment; the \emph{Global Goals Agent} which enforces global policies towards energy efficiency; the \emph{Context Inference Agent} which sets actions in context with users, location, and time; the \emph{Auxiliary Data Agent} which obtains data from various sources;
the \emph{KB Interface Agent} which provides the link between the knowledge base and the multi-agent system and is responsible for data exchange in all parts of the system; and the \emph{BAS Interface Agent} which acts as an interface between the agent society and the building automation systems (\eacs{BAS}) available in the smart home.

\section{Ontologies for weather data}
\label{sec:weather_ontologies}

To investigate the possibility that an already existing ontology qualifies to be used as a basis for \smarthomeweather, this section presents a selection of ontologies that have been designed to cover the domain of weather data. In case no suitable ontology can be identified, the advantages and disadvantages of the ontologies discussed are analysed in order to avoid their shortcomings in \smarthomeweather and to benefit from their advantages.

This section covers \emph{Semantic Sensor Web} (Section~\ref{subsec:onto1}), the \emph{\eacs{SSN} ontology} (Section~\ref{subsec:onto2}), \emph{\eacs{SWEET}} (Section~\ref{subsec:onto5}), and \emph{NextGen} (Section~\ref{subsec:onto4}). There are several other approaches such as the \emph{SENSEI project}~\cite{sensei} which are not covered here.

Apart from ontology-based weather data models, there exist several approaches which incorporate both current and predicted weather data without embodying an ontology. A commonly used approach is the use of a mathematical model that allows the transformation of the problem of predictive control based on weather data to an optimization problem~\cite{NonOntologicalApproach1,NonOntologicalApproach2}. As the structure of these approaches differs greatly from the ontological approach that is found at smart home systems like \thinkhome, these approaches are not covered here.

\subsection{Semantic Sensor Web}
\label{subsec:onto1}

\emph{Sensor Web Enablement} (\eacs{SWE})~\cite{SensorWeb} is an initiative by the \emph{Open Geospatial Consortium} (\eacs{OGC})~\cite{OGC} for building networks of sensors based on Web technologies. Both sensors and archived sensors are intended to be discovered, accessed and optionally controlled using open standard protocols and interfaces. \eacs{SWE} is a suite of standards, each specifying encodings for describing sensors, sensor observations, and/or sensor interface definitions.

There exists a huge number of sensor networks around the globe comprising sensors for a large set of different phenomena. Therefore a vast amount of data is available, the need arises to structure that data and allow interoperability between different sensor networks. \emph{Semantic Sensor Web} (\eacs{SSW}) is an approach that builds upon \eacs{SWE} and Semantic Web activities by the \eacs{W3C} which aims at annotating sensor data with semantic metadata to increase interoperability and to provide contextual information essential for situational knowledge~\cite{SemanticSensorWeb}. Semantic metadata includes spatial, temporal, and thematic data.

The ontology of \eacs{SSW} -- which is referred to as the \emph{\eacs{SSW} ontology} from now on -- is built upon seven top-level concepts, excluding the concepts \emph{Location} and \emph{Time} which are imported from other sources. These top-level concepts are \emph{Feature} (an abstraction of a phenomenon from the real word, e.g. a weather event such as a blizzard), \emph{Observation} (the act of observing a property or phenomenon with the goal of a value of a property), \emph{ObservationCollection} (a set of \emph{Observation}s), \emph{Process} (a method, an algorithm, an instrument, or a system of these), \emph{PropertyType} (a characteristic of one or more types of \emph{Feature}s), \emph{ResultData} (an estimate of the value of some property generated by a known procedure), and \emph{UnitOfMeasurement} (as the name suggests, a unit of measurement).

The \eacs{SSW} ontology includes temporal data using \emph{OWL-Time} (see Section~\ref{subsec:date_ontologies}), units of measurements and geographical data based on the \emph{Basic (WGS84 lat/long) Vocabulary} (see Section~\ref{subsec:location_ontologies}). The concept \emph{WeatherObservation} comes with five predefined sub-concepts designed to map observations of atmospheric pressure, precipitation, radiation, temperature, or wind, respectively. Additional concepts may be added for observations of other phenomena.

The \eacs{SSW} ontology does not support forecast values, but extending the \eacs{SSW} ontology to implement those would be possible.

Applications of \eacs{SSW} include work on situation awareness based on the metadata specified by \eacs{SSW}~\cite{ssw_example1} and an architecture for a distributed semantic sensor web~\cite{ssw_example2}.

The only part from the \eacs{SSW} ontology that could be reused for \smarthomeweather is the concept \emph{Observations} together with its sub-concepts; however, a number of additional sub-concepts would have to be added. The \emph{Basic (WGS84 lat/long) Vocabulary} and \emph{OWL-Time} can be used by the \smarthomeweather ontology without the use of the \eacs{SSW} ontology. For units of measurements there are other approaches than the \eacs{SSW} approaches available that qualify for being used by \smarthomeweather. Therefore, it was decided not to use the \eacs{SSW} ontology for \smarthomeweather.

\vspace{1em}

\subsection{\eacs{SSN} Ontology}
\label{subsec:onto2}

Another approach towards semantically enriched sensor network based on \eacs{SWE} is an \emph{\eacs{OWL} 2} ontology created by the \emph{\eacs{W3C} Semantic Sensor Network Incubator group} (\eacs{SSN-XG})~\cite{SSN-XG} which is referred to as the \emph{\eacs{SSN} Ontology}~\cite{ssn_ontology}. The goal of this ontology is to simplify managing, querying, and combining sensors and observation data from different sensor networks. 

The \eacs{SSN} ontology uses \emph{DOLCE-UltraLite}~\cite{dul} as \emph{upper-level ontology}. It defines $41$ concepts and $39$ object properties which are organised into ten modules: \emph{ConstraintBlock} (for defining conditions on a system's or a sensor's operation), \emph{Data} (for encoding any input from sensors), \emph{Device} (for defining devices in the sensor network, mostly sensors), \emph{Deployment} (for specifying the deployment of \emph{Device}s), \emph{MeasuringCapability} (properties of sensors, e.g. accuracy or response time), \emph{OperatingRestriction} (for defining conditions under which the system is expected to operate, e.g. the life time of batteries or maintenance schedules), \emph{PlatformSite} (entities to which other entities -- sensors and other platforms -- can be attached), \emph{Process} (a procedure that changes the system's state in some way, takes some input and yields to some output), \emph{Skeleton} (for mapping real-world phenomena, their properties, and their relations to sensors) and \emph{System} (for describing pieces of infrastructure, e.g. the whole network, a component, its subsystems etc.).

The ontology can be used to view at the knowledge base from a number of perspectives: The \emph{sensor perspective} (which sensors are available; what and how do they sense), the \emph{observation perspective} (focusing on observations and related metadata), the \emph{system perspective} (systems of sensors), and the \emph{property perspective} (properties of physical phenomena and how they are sensed).

Work based on the \eacs{SSN} ontology includes its use for the representation of humans and personal devices as sensors~\cite{ssn_example1}, its application in sensing for manufacturing~\cite{ssn_example2}, and as part of a linked data infrastructure for \eacs{SWE}~\cite{ssn_example3}.

The primary reason for not using the \eacs{SSN} ontology for \smarthomeweather is the fact that it was published later than the development of \smarthomeweather started. Furthermore, for the purpose of \smarthomeweather, the \eacs{SSN} ontology uses a level of abstraction that makes its use in smart homes too complicated. \eacs{SSN} is well engineered towards mapping sensor networks and its observations, but it does not qualify for only representing sensor data without mapping the details of a sensor network. Additionally, forecast data is currently not supported, but an appropriate extension would be possible.

\subsection{SWEET}
\label{subsec:onto5}

\eacs{SWEET} is a set of more than 200 ontologies comprising about 6000 concepts~\cite{SWEET2,SWEETWeb}. It is developed by \eacs{NASA}'s \emph{Jet Propulsion Laboratory} at the \emph{California Institute of Technology}~\cite{nasa-jpl}. The initial version which dates back to 2003 is based on \emph{DAML+OIL}~\cite{DAML+OIL,SWEET1}. It structured its ontologies into the following categories:
\begin{itemize}
  \item \emph{Earth Realm}: The ``spheres'' (e.g. atmosphere or ocean) of Earth belong to this category.
  \item \emph{Non-Living Element}: This category includes non-living building blocks on nature (e.g. particles or electromagnetic radiation).
  \item \emph{Living Element}: All plants and animal species belong to this category.
  \item \emph{Physical Properties}: This category contains physical properties (e.g. temperature or height).
  \item \emph{Units}: This category comprises units of measurement for all literal values used in the ontologies.
  \item \emph{Numerical Entity}: This category includes numerical extents (e.g. interval or point) and numerical relations (e.g. greatherThan or max).
  \item \emph{Temporal Entity}: This category includes ontologies that cover the temporal domain: temporal extents (e.g. duration or season) and temporal relations (e.g. after or before).
  \item \emph{Spatial Entity}: Ontologies that cover the spatial domain fall into this category: spatial extents (e.g. country or equator) and spatial relations (e.g. above or northOf).
  \item \emph{Phenomena}: This category contains ontologies that define transient events; a phenomenon that is described by such an ontology crosses bounds of other ontology domains (e.g. hurricane or terrorist event).
\end{itemize}

\eacs{SWEET} is now implemented in \eacs{OWL}. The current version, \eacs{SWEET} 2.3 was last updated in 2012.

Works based on \eacs{SWEET} include work on integrating volcanic and atmospheric data in the context of volcanic eruptions~\cite{sweet_example1}, an ontology of fractures of the Earth's crust~\cite{sweet_example2}, and an extension of \eacs{SWEET} by climate and forecasting terms~\cite{sweet_example3}.

In the context of smart homes, the \eacs{SWEET} ontologies are the best-qualified approach for reusing them in an weather ontology. Units of measurements, temporal definitions, and specifications of geographical positions are supported. However, \eacs{SWEET} comes with a few downside that finally led to the decision not to use them in \smarthomeweather:
\begin{itemize}
  \item Although \eacs{SWEET} is separated into a large number of ontologies, there are many dependencies between ontologies. Importing a single ontology into an \eacs{OWL} ontology causes the import of all of its dependencies. It is impossible to import one ontology from \eacs{SWEET} without importing dozens of other ontologies.
  
  \item \eacs{SWEET} does not currently cover future events.
  
  \item \eacs{SWEET} does not cover all weather elements that are relevant.
\end{itemize}

Extensions to the \eacs{SWEET} ontologies in order address the second and the third issue would be possible; an extension of \eacs{SWEET} to include forecast data already exists~\cite{sweet_example3}. However, as a large number of extensions would be necessary, it was decided not to use any parts of \eacs{SWEET} for \smarthomeweather and to build a new ontology from scratch instead.

\subsection{NextGen}
\label{subsec:onto4}

\eacs{NextGen} (\emph{Next Generation Air Transport System})~\cite{NextGen} is a large-scale project carried out by the \emph{Federal Aviation Administration} (\eacs{FAA}), an organisation being responsible for all aspects of civil air traffic in the United States~\cite{faa}. The goal of \eacs{NextGen} is to completely reorganise the US airspace to shorten flight paths, save time and fuel, minimise delays, increase capacity, and improve safety. One of the elements that \eacs{NextGen} consists of is \emph{Next Generation Network Enabled Weather} (\eacs{NNEW}) which is designed to provide a comprehensive view on the weather across the country, built from thousands of single weather observations. Once completed, \eacs{NNEW} is expected to reduce weather-related delays in US airspace to the half of its current magnitude; currently, approximately 70 percent of all air traffic delays are attributable to weather~\cite{nnew}.

In order to efficiently process huge amounts of input weather data, \eacs{NNEW} incorporates a knowledge base implemented using a set of ontologies. The \eacs{NNEW} ontology is centered around concepts and relations describing past, current, and future weather phenomena.

The \eacs{NNEW} ontology is built on top of the \eacs{SWEET} ontologies (see Section~\ref{subsec:onto5}) to map weather phenomena. Extensions to \eacs{SWEET} include additional weather phenomena and concepts and relations that lead to the \emph{4-D Wx Data Cube} (\emph{Four Dimensional Weather Data Cube}) which uses time as the fourth dimension for the ``location'' of weather observations.

It was decided not to use the \eacs{NNEW} ontology for \smarthomeweather for the same reasons as already explained in the case of \eacs{SWEET}. In the context of smart homes, the \eacs{NNEW} ontology appears just as an extension to \eacs{SWEET} and therefore it is not qualified in equal measure. Furthermore, \eacs{NNEW} is still work in progress and no final version is available that could be seen as a standard.

\section{Related ontologies}
\label{sec:related_ontologies}

This section discusses some ontologies that cover domains that are related to weather data, e.g. location data, temporal data, or units of measurements. These are candidates for being reused as part of \smarthomeweather.

\subsection{Location data}
\label{subsec:location_ontologies}

\begin{figure}
\centering
\includegraphics[width=.9\textwidth]{figures/diagrams/wgs84_example.pdf}
\caption[Example of the use of the \emph{Basic Geo (WGS84 lat/long) Vocabulary}]{Example of the use of the \emph{Basic Geo (WGS84 lat/long) Vocabulary} to specify the location of Vienna, Austria (N~\SI{48.21}{\degree}, E~\SI{16.37}{\degree}, \SI{171}{\metre} above \eacs{MSL}); all concepts and properties in the \texttt{geo} namespace are part of this vocabulary.}
\label{fig:wgs84_example}
\end{figure}

To handle geographical location data, the \emph{\eacs{W3C} Semantic Web Interest Group} (\eacs{SWIG}) developed the \emph{Basic Geo (WGS84 lat/long) Vocabulary}~\cite{wgs84_vocabulary}. It introduces a concept called \emph{Spatial thing} and its attributes \egls{lat}, \egls{long} and \egls{alt} according to the \emph{WGS-84 geodetic reference system}~\cite{WGS84}. Figure~\ref{fig:wgs84_example} shows an example of how the vocabulary is used.

As a downside, the vocabulary contains some data properties that are incorrectly defined to be annotation properties. Thus, they must be redefined to be data properties whenever used in an \eacs{OWL} ontology.

Furthermore, the \emph{Basic Geo (WGS84 lat/long) Vocabulary} is not a standard; is has not even been submitted to the \emph{\eacs{W3C} recommendation track} for standardisation. No work on the vocabulary has been done since 2006. The \emph{\eacs{W3C} Geospatial Incubator Group} proposed the introduction of \emph{GeoRSS XML} and \emph{Geo OWL} in 2007~\cite{w3c_geo_report1} which are designed to enrich \eacs{RSS}/Atom feeds and \eacs{OWL} ontologies with geographical information. Although \emph{GeoRSS} seems to have gained popularity over the last few years, \emph{Geo OWL} continues to lead a miserable existance. The last reports by the \emph{\eacs{W3C} Geospatial Incubator Group} were published in 2007~\cite{w3c_geo_report1,w3c_geo_report2}. No standard which may be qualified to supersede the \emph{Basic Geo (\eacs{WGS84} lat/long) Vocabulary} has yet been released.

\subsection{Date and time}
\label{subsec:date_ontologies}

\begin{figure}
\centering
\includegraphics[width=.9\textwidth]{figures/diagrams/owl_time_example1.pdf}
\caption[Example of the use of the \emph{OWL-Time} for describing an instant]{Example of the use of the \emph{OWL-Time} for describing an instant (June 16, 2012 at 14:30).}
\label{fig:owl_time_example1}
\end{figure}

\begin{figure}
\centering
\includegraphics[width=.5\textwidth]{figures/diagrams/owl_time_example2.pdf}
\caption[Example of the use of the \emph{OWL-Time} for the description an interval]{Example of the use of the \emph{OWL-Time} for the description of the interval of 90 minutes (one hour and 30 minutes).}
\label{fig:owl_time_example2}
\end{figure}

For specifying temporal properties, the \eacs{W3C} offers a \emph{working draft} of \emph{OWL-Time}~\cite{owl-time}, a \emph{Time Ontology in \eacs{OWL}}. It defines the concept called \Egls{temporal entity} that can either be an \Egls{instant} or an \Egls{interval}. Using appropriate attributes, the properties of a \Egls{temporal entity} are specified. Although \emph{OWL-Time} is in the state of a \emph{working draft} since September 2006, the main concepts and attributes that will be reused by other ontologies are likely to remain unchanged in future releases of that ontology.

Figure~\ref{fig:owl_time_example1} shows an example of \emph{OWL-Time} being used to specify an instant while Figure~\ref{fig:owl_time_example2} demonstrates how a time interval is specified using \emph{OWL-Time}.

\subsection{Units of measurements}
\label{subsec:unit_ontologies}

As the use of units of measurement is a topic that occurs in many ontologies, there are several different approaches to cope with it.

\begin{figure}
\centering
\includegraphics[width=.4\textwidth]{figures/diagrams/nounits_example.pdf}
\caption[Example of a data model lacking units of measurement]{Example of a data model lacking units of measurement; the model represents an estate with a length of \SI{50}{metres} and a width of \SI{30}{metres}.}
\label{fig:nounits_example}
\end{figure}

\begin{figure}
\centering
\includegraphics[width=.9\textwidth]{figures/diagrams/muo_example.pdf}
\caption[Example of the use of \muo]{Example of the use of \muo for the introduction of units of measurements.}
\label{fig:muo_example}
\end{figure}

The \emph{Measurement Units Ontology}~\cite{MUOWeb,MUO} is a simple and light-weight approach to enrich measurement values in an ontology with appropriate units. Some of the most important units are pre-defined, others can easily be added when needed. The \emph{Measurement Units Ontology} (\muo) is still work in progress, however it is not expected that it might change heavily in the future. Everything that will be reused by other ontologies will probably remain unchanged. Hence, \muo can be imported into any ontology without problems.

Figure~\ref{fig:nounits_example} shows the representation of an estate with its length and width specified using the datatype properties \texttt{hasLength} and \texttt{hasWidth}. In Figure~\ref{fig:muo_example}, \muo is introduced to specify that both length and width are measured in metres: \texttt{hasLength} and \texttt{hasWidth} become object properties which link blank nodes to the individual \texttt{Estate}; \texttt{hasLength} and \texttt{hasWidth} are now both sub-properties of the property \texttt{muo:Quality value}. Each blank node has two properties, \texttt{muo:numerical value} states the literal value while \texttt{muo:measured in} gives the unit of measurement for the literal value. The unit (\texttt{meter} in this case) is represented by an instance of \texttt{muo:Unit of measurement}.

\begin{figure}
\centering
\includegraphics[width=.9\textwidth]{figures/diagrams/om_example.pdf}
\caption[Example of the use of \eacs{OM}]{Example of the use of \eacs{OM} for the introduction of units of measurements.}
\label{fig:om_example}
\end{figure}

The \emph{Ontology of Units of Measure and Related Concepts} (\eacs{OM})~\cite{OMWeb,OM} is another promising approach for adding measurement units to \emph{OWL} that even provides features like the conversion between different units for the same quantities and representation and checking of formulas. Figure~\ref{fig:om_example} shows an example of the usage of \eacs{OM}. In this example, the specification the length of the estate is centered around a blank node which is an instance of \texttt{om:Length} which is a subconcept of \texttt{om:Quantity}. The blank node links to \texttt{Estate} using the property \texttt{om:phenomenon} and to another blank node via the property \texttt{om:value} which is an instance of \texttt{om:Measure}. This instance has a datatype property of type \texttt{om:numerical\_value} which specifies the numeric value of the estate's length, while the object property \texttt{om:unit}
links to an instance of \texttt{om:Unit} named \texttt{om:meter}. The width of the estate is stated in a similar way.

\eacs{OM} is a rather large ontology (nearly \SI{2.4}{\mebi\byte} in \emph{RDF/XML syntax}) compared to the \smarthomeweather ontology and related to the purpose it would fulfil in \smarthomeweather; furthermore, at the time when the article about \eacs{OM} was published, development of \smarthomeweather had already completed. Hence, \eacs{OM} was not taken into account for being used in the \smarthomeweather ontology.

Besides the \emph{Measurement Units Ontology} and the \emph{Ontology of Units of Measure and Related Concepts}, several other ontologies have been examined, but all of them have shortcomings that render their use in the \smarthomeweather ontology merely impossible. These ontologies are:

\begin{itemize}
  \item \textbf{\eacs{SWEET}}: Besides concepts, attributes and individuals for atmospherical phenomena, \eacs{SWEET} also comes with support for literals more precisely specified by units~\cite{SWEET1}. However, as mentioned above, \eacs{SWEET} is an ontology that is inappropriate for use in \smarthomeweather.
  
  \item \textbf{\eacs{QUDT}}: \eacs{QUDT}, a set of ontologies for \emph{Quantities, Units, Dimensions and Data Types in OWL and XML}, is a promising approach for adding support for units to an \emph{OWL} ontology~\cite{QUDT}. However, \eacs{QUDT} does not work in \protege together with the \emph{Pellet} reasoner. Using \eacs{QUDT} would require several changes to its \emph{OWL} files. Hence, it cannot be used in the \smarthomeweather ontology.
  
  \item \textbf{\eacs{QUOMOS}}: The \emph{\eacs{OASIS} Quantities and Units of Measure Ontology Standard} (\eacs{QUOMOS})~\cite{QUOMOS} is a project that aims at developing ``an ontology for quantities, systems of measurement units, and base dimensions for use across multiple industries''. However, as no deliverables have been released at the time of writing, it cannot be used in \smarthomeweather.
  
  \item \textbf{\eacs{OBO} Foundry Initiative}: The \emph{\eacs{OBO} Foundry Initiative}~\cite{OBOFoundry} is an initiative that aims at collecting ontologies for use in the biomedical domain. The list of \emph{The Open Biological and Biomedical Ontologies} (abbreviated by \eacs{OBO}) includes an ontology for units of measurements~\cite{OBOUnits1,OBOUnits2}. This ontology apparently covers most of the units that are used in the \smarthomeweather ontology. However, it lacks any documentation and therefore is unsuitable for being used in \smarthomeweather.
\end{itemize}

\section{Conclusion}

As Section~\ref{sec:weather_ontologies} discusses, unfortunately no existing ontology covers the domain of weather data in a way suitable for using it as a starting point for \smarthomeweather.

Thus, a completely new ontology is created (see Chapter~\ref{ch:smarthomeweather_ontology}). From the insights gained in this chapter through existing weather ontologies, the following aspects are considered for the development of \smarthomeweather:

\begin{itemize}
  \item Many of the existing weather ontologies are intended to map a whole sensor network. In the case of \smarthomeweather, this is not necessary. To avoid overhead, only observations of weather elements (temperature, humidity etc.) will be covered, not their sensors. One goal of \smarthomeweather is to keep it as sophisticated as required, but as simple as possible.
  
  \item The \emph{Basic Geo (WGS84 lat/long) Vocabulary} and \emph{OWL-Time} are used by some of the existing ontologies. This qualifies these vocabularies to adequately model geographical and temporal data in a number of different domains. Hence, both vocabularies will be imported by \smarthomeweather.
  
  \item Some ontologies include concepts and properties that are defined to represent units of measurement. \smarthomeweather will support units of measurement as well, in fact by using the \emph{Measurements Units Ontology}.
\end{itemize}

