\chapter{Existing work}
\label{ch:existing_work}

The first part of this chapter covers the foundations the work of this thesis builds upon. It gives introductions into all revelant topics, e.g. ontologies, \eacs{RDF} and \eacs{OWL}. Furthermore, it discusses the basic principles of \thinkhome which are required in order to fully understand the ideas behind the weather data model which is designed in later chapters.

The second part sheds light on a selection of existing work regarding weather data in smart homes and ontologies for weather data. Additionally, some ontologies which may be reused in an ontology for weather data are discussed. The final section of this chapter reviews all ontologies that are presented and identifies elements can be used for \thinkhome.

\section{Foundations}

This chapter presents the foundations the work in this thesis builds upon: The concept of an ontology, the \emph{Web Ontology Language} (\eacs{OWL}) and \thinkhome.

\subsection{Ontologies}

% TODO citation
In computer science, an ontology represents knowledge as a set of concepts in a certain domain and relationships between pairs of concepts. The basic elements of ontologies -- concepts, properties, and relations -- comprise a shared vocabulary which can be used to model a certain domain.

Each object that is mapped into an ontology is represented by an \emph{individual} (also known as \emph{object}). Individuals of the same type can be defined to be \emph{instances} of \emph{concepts} (also called \emph{classes}). Both classes and individuals can have \emph{attributes} that specify their characteristics and properties. Two arbitrary classes or individuals can be related to each other via a \emph{relation}.

% TODO descriptions
Furthermore, ontologies often contain \emph{function terms} (), \emph{restrictions} (), \emph{rules} (), \emph{axioms} (), and \emph{events} ().

\vspace{1em}

\begin{figure}
\centering
\includegraphics[width=\textwidth]{figures/diagrams/ontology_example.pdf}
\caption{Example of a simple ontological model}
\label{fig:ontology_example}
\end{figure}

Figure~\ref{fig:ontology_example} illustrates the aforementioned elements in a simple ontology: \emph{Person}, \emph{Man}, and \emph{Woman} are concepts. \emph{is subclass of} is a property that defines one concept to be a \emph{sub-concept} of another concept (i.e. \emph{B is subclass of A} states that every instance of \emph{B} is also an instance of \emph{A}). \emph{Brian}, \emph{Maria}, and \emph{Sarah} are individuals; each of them is a instance of a concept which is related via the \emph{is a} relation. \emph{age} is a property of the individuals \emph{Brian}, \emph{Maria}, and \emph{Sarah} and \emph{has parent} is a relation which associates two individuals to each other.

This model states the following facts: Men and women are persons. Brian (age 37) is a man while Maria (age 12) and Sarah (age 32) are women. Maria has two parents, Brian and Sarah.

\vspace{1em}

An important feature of an ontology is the support of automatic reasoning to deduct facts that are not explicitely stated in the data model from the given information. In order to make reasoning possible, the semantics of data models in ontologies (including \eacs{OWL}) are often based on \emph{Description Logics}~\cite{OWL,SROIQ}. These are a family of logics consisting of decidable parts of first-order predicate logic.

In the above example, because \emph{Man} and \emph{Woman} are sub-concepts of \emph{Person}, \emph{Brian}, \emph{Maria}, and \emph{Sarah} are instances of \emph{Person}. One could define a relation \emph{has child} that is an inverse property of \emph{has parent}, i.e. \emph{A has child B} if and only if \emph{B has parent A}; then the statements \emph{Brian has child Maria} and \emph{Sarah has child Maria} can be deducted.

If a concept \emph{Mother} is defined as a \emph{Woman} who has at least one child, \emph{Sarah} can be inferred to be an instance of this concept; the same works for an analogously defined concept \emph{Father} with \emph{Brian} being an instance of. Furthermore, concepts \emph{Daugher} and \emph{Son} can be defined (someone who has at least one parent and is a \emph{Woman} or a \emph{Man}, respectively). Then \emph{Maria} can be inferred to be a \emph{Daughter}. Additionally, properties like \emph{has mother}, \emph{has father}, \emph{has son}, \emph{has daughter} etc. can be defined.

\vspace{1em}

Another core principle of an ontology is reusability\cite{reuse1,reuse2}. In order to share knowledge across various systems, ontologies are often reused within other ontologies. Besides the simplification of knowledge sharing, ontology reuse tries to avoid duplicate work and reduces the work that is necessary to create a new ontology for a domain.

\vspace{1em}

Ontologies are expressed using formal languages. There are many of these \emph{ontology languages} such as \eacs{KIF} (\emph{Knowledge Interchange Format})~\cite{KIFReference}, \emph{DAML+OIL}~\cite{DAML+OIL} -- a successor to \eacs{DAML}\footnote{\href{http://www.daml.org/}{http://www.daml.org/}} (\emph{\eacs{DARPA} Agent Markup Language}) and \eacs{OIL} (\emph{Ontology Inference Layer})~\cite{OIL} that combines features of both --, \eacs{RDFS} (\emph{RDF Schema})~\cite{RDFS}, and \eacs{OWL}~\cite{OWL}. The latter, the \emph{Web Ontology Language}, is used in the \thinkhome project; thus, section~\ref{subsec:owl} gives a brief introduction into \eacs{OWL}.

Over the years, many ontologies have been developed. Some of them define concepts and relations that can be found across many knowledge domains. These ontologies are often imported into other ontologies in order to minimize the effort of creating new ontologies and to simplify interoperability of ontologies. Examples for such ontologies that are implemented in \eacs{OWL} or \emph{\eacs{RDF} Schema} are:
\begin{itemize}
  \item \textbf{\eacs{DOAP}}: \eacs{DOAP}\footnote{\href{https://github.com/edumbill/doap/}{https://github.com/edumbill/doap/}} (\emph{Description of a Project}) is an vocabulary for describing software projects.
  \item \textbf{Dublin Core}: \emph{Dublin Core}\footnote{\href{http://dublincore.org/}{http://dublincore.org/}}~\cite{rfc5013} is a vocabulary for describing metadata of web documents, physical resources (documents, books etc.) and other objects such as works of art.
  \item \textbf{\eacs{FOAF}}: \eacs{FOAF}\footnote{\href{http://www.foaf-project.org/}{http://www.foaf-project.org/}}~\cite{FOAF} is an ontology for describing social networks. It models persons, the relations between them, their activities and their relations to other objects.
  \item \textbf{\eacs{SIOC}}: The \eacs{SIOC} \footnote{\href{http://xmlns.com/foaf/spec/}{http://xmlns.com/foaf/spec/}}~\cite{SIOC} is a technology built around an ontology for encoding information from Internet discussion methods (message boards, blogs, mailing lists etc.).
  \item \textbf{\eacs{SKOS}}: \eacs{SKOS}\footnote{\href{http://www.w3.org/2004/02/skos/}{http://www.w3.org/2004/02/skos/}} (\emph{Simple Knowledge Organization System})~\cite{SKOS} is a data model for sharing and linking knowledge organisation systems (thesauri, taxonomies, classification schemes and subject heading systems).
  \item \textbf{\eacs{UMBEL}}: \eacs{UMBEL}\footnote{\href{http://umbel.org/}{http://umbel.org/}} (\emph{Upper Mapping and Binding Exchange Layer}) is an approach towards interoperability of content on the Web. It is a vocabulary for the construction of ontologies beging designed for interoperation and provides a reference structure of $25,000$ concepts that provide a scaffolding to link and interoperate datasets and domain vocabularies.
\end{itemize}

% TODO mention semantic web

\subsection{OWL}
\label{subsec:owl}

The \emph{Resource Description Framework} (\eacs{RDF}) is a standard model for knowledge representation~\cite{RDF}. It is specified in a set of recommendations by the \emph{World Wide Web Consortium} (\eacs{W3C})\footnote{\href{http://www.w3.org/}{http://www.w3.org/}}.

The data model of \eacs{RDF} incorporates element types named \emph{resources} and \emph{properties} which can also be found in ontologies: A resource is some thing that is described by \eacs{RDF}, it corresponds to an individual as it is found in an ontology. Each resource can have an arbitrary number of properties, i.e. \emph{attributes} that associate \emph{literate} values (e.g. numerical values, strings) to the resource or \emph{relations} that link this resource to other resources. Resources and properties are expressed using \emph{statements} (\emph{triples}) which consist of three parts called \emph{subject}, \emph{predicate} and \emph{object}. To identify resources and properties, \eacs{RDF} uses \eacss{URI} (\emph{Unified Resource Locators}). \eacss{URI} are strings defined by \emph{\eacs{RFC} 3986}~\cite{rfc3986} that uniquely identify things.

\begin{figure}
\centering
\includegraphics[width=.9\textwidth]{figures/diagrams/rdf_example.pdf}
\caption{Example of a simple \eacs{RDF} model}
\label{fig:rdf_example}
\end{figure}

Figure~\ref{fig:rdf_example} depicts a simple example for a piece of knowledge from the domain of weather data expressed using \eacs{RDF}: The resource \texttt{weather:CurrentWeather} represents the current state of the weather. The property \texttt{geo:hasLocation} links the \texttt{weather:CurrentWeather} to the resource \texttt{geo:Vienna} which represents the city of Vienna, Austria; i.e. \texttt{weather:CurrentWeather} describes the weather for Vienna. \texttt{weather:weatherState} has two more properties, \texttt{weather:hasTemperatureValue} and \texttt{weather:hasHumidityValue}, which link two literal values to the resource: a temperature value of \SI{18.2}{\celsius} and a humidity value of $72 \%$. The type \texttt{xsd:float} of both literals is defined by \emph{XML Schema}~\cite{xml-schema,xml-schema-datatypes}, one of the several \eacs{XML} schema languages available that define the structure of \eacs{XML} documents.

The complete \eacs{URI} of \texttt{weather:CurrentWeather} is \texttt{http://example.org/weather\#CurrentWeather} and the \eacs{URI} of \texttt{geo:Vienna} is \texttt{http://example.org/geo\#Vienna}. As in \eacs{XML}, parts of \eacss{URI} may be replaced by \emph{prefixes} to avoid frequent recurrence of the same strings. For the above example, the prefix \texttt{weather} has been defined to replace the string \texttt{http://example.org/weather\#} and \texttt{geo} replaces \texttt{http://example.org/geo\#}. This results in the identifiers \texttt{weather:CurrentWeather} and \texttt{geo:Vienna} that can be found in figure~\ref{fig:rdf_example}.

For expressing the data that is represented by an \eacs{RDF} model, several serialization formats are available. The \eacs{RDF} recommendation is based on \emph{RDF/XML} which maps the \eacs{RDF} model to an \eacs{XML} document~\cite{RDF_XML}. The representation of the above example in \emph{RDF/XML} can be seen in figure~\ref{fig:rdfxml_example}. As \emph{RDF/XML} is a rather verbose format which may be difficult to read for humans, the \eacs{N3} (\emph{Notation3}) representation for \eacs{RDF} is available which was developed with human-readability in mind~\cite{Notation3}. \emph{Notation3} incorporates some syntax features that go beyond the expressive power of \eacs{RDF}. A subset of \emph{Notation3} named \eacs{Turtle} (\emph{Terse RDF Triple Language}) is available that is limited to the features required to map \eacs{RDF} models~\cite{Turtle}. Figure~\ref{fig:turtle_example} shows the above example in \eacs{Turtle} syntax.

\begin{figure} % TODO syntax highlighting
\begin{lstlisting}
<?xml version="1.0"?>
<rdf:RDF xmlns:xsd="http://www.w3.org/2001/XMLSchema#" xmlns:geo="http://example.org/geo#" xmlns:rdf="http://www.w3.org/1999/02/22-rdf-syntax-ns#" xmlns:weather="http://example.org/weather#">
	<rdf:Description rdf:about="http://example.org/weather#CurrentWeather">
		<weather:hasTemperatureValue rdf:datatype="http://www.w3.org/2001/XMLSchema#float">18.2</weather:hasTemperatureValue>
		<weather:hasHumidityValue rdf:datatype="http://www.w3.org/2001/XMLSchema#float">.72</weather:hasHumidityValue>
		<geo:hasLocation rdf:resource="http://example.org/geo#Vienna" />
	</rdf:Description>
</rdf:RDF>
\end{lstlisting}
\caption{\eacs{RDF} example from figure~\ref{fig:rdf_example} encoded in \emph{RDF/XML} syntax}
\label{fig:rdfxml_example}
\end{figure}

\begin{figure}
\begin{lstlisting}
@prefix rdf: <http://www.w3.org/1999/02/22-rdf-syntax-ns#> .
@prefix weather: <http://example.org/weather#> .
@prefix geo: <http://example.org/geo#> .
@prefix xsd: <http://www.w3.org/2001/XMLSchema#> .

weather:CurrentWeather weather:hasTemperatureValue "18.2"^^xsd:float ;
    weather:hasHumidityValue ".72"^^xsd:float .

weather:CurrentWeather geo:hasLocation geo:Vienna .
\end{lstlisting}
\caption{\eacs{RDF} example from figure~\ref{fig:rdf_example} encoded in \eacs{Turtle} syntax}
\label{fig:turtle_example}
\end{figure}

\eacs{RDF} \emph{schema} (\eacs{RDFS}) is a recommendation by the \eacs{W3C} that builds upon \eacs{RDF}~\cite{RDFS}. It introduces a set of concepts and properties that permits the definition of simple ontologies. 

All things described by \eacs{RDF} are instances of \texttt{rdfs:Resource}, concepts are instances of \texttt{rdfs:Class}, and properties are instances of \texttt{rdfs:Property}. Other concepts introduced by \eacs{RDFS} are \texttt{rdfs:Literal}, \texttt{rdfs:Datatype} and \texttt{rdfs:XMLLiteral}.

The property \texttt{rdfs:domain} states that any resource that has a given property is an instance of one or more classes; the property \texttt{rdfs:range} states that the value of a property is an instance of one or more classes. \texttt{rdf:type} (often abbreviated with \texttt{a}) is used to state that a resource is an instance of a class. Hierarchies of classes can be constructed using the property \texttt{rdfs:subClassOf}: \texttt{C1 rdfs:subClassOf C2} states that any instance of \texttt{C2} is also an instance of \texttt{C1}. \texttt{rdfs:subPropertyOf} is an equivalent that is used for declaring hierachies of properties. Other properties defined by \eacs{RDFS} are \texttt{rdfs:label} and \texttt{rdfs:comment}.

\begin{figure}
\centering
\includegraphics[width=.9\textwidth]{figures/diagrams/rdfs_example.pdf}
\caption{Example of a simple \eacs{RDFS} model}
\label{fig:rdfs_example}
\end{figure}

Figure~\ref{fig:rdfs_example} shows the example from figure~\ref{fig:rdf_example}, enriched by some elements that are defined by \eacs{RDFS}. The data model introduces two classes, \texttt{weather:WeatherState} and \texttt{geo:Location}. \texttt{weather:CurrentWeather} and \texttt{geo:Vienna} are instances of \texttt{weather:WeatherState} and \texttt{geo:Location}, respectively. \texttt{geo:hasLocation} is a property with range \texttt{weather:WeatherState} and \texttt{geo:Location}.

\eacs{RDFS} comes with reasoning support~\cite{RDF_semantics}; e.g. in the above example, the statements

\begin{verbatim}
weather:CurrentWeather rdf:type weather:WeatherState .
geo:Vienna rdf:type geo:Location .
\end{verbatim}

can be removed without loss of knowledge. A reasoner can deduct them from these statements:

\begin{verbatim}
weather:CurrentWeather geo:hasLocation geo:Vienna .
geo:location rdfs:domain weather:WeatherState .
geo:location rdfs:range geo:Location .
\end{verbatim}
However, as the expressive power of \eacs{RDFS} is rather limited compared to what \eacs{OWL} has to offer, reasoning in \eacs{RDFS} does not have the significance it has in \eacs{OWL}.

Many of the concepts and properties defined are included in the \emph{Web Ontology Language} (\eacs{OWL}), a more expressive ontology language than \eacs{RDFS} based on \eacs{RDF} and \eacs{RDFS}. \eacs{OWL} is developed by the \emph{OWL Working Group}\footnote{\href{http://www.w3.org/2007/OWL/wiki/OWL\_Working\_Group}{http://www.w3.org/2007/OWL/wiki/OWL\_Working\_Group}} of the \eacs{W3C}. \eacs{OWL} 1 was first published in July 2002 as a \emph{working draft} and became a \eacs{W3C} \emph{recommendation} in February 2004\footnote{Both \emph{working draft} and \emph{recommendation} are \emph{maturity levels} proposed by the \eacs{W3C} that indicate the state of their technical reports~\cite{w3c-process}.}~\cite{OWL1}; the first \emph{working draft} of \eacs{OWL} 2 in March 2009 and the \eacs{W3C} \emph{recommendation} of \eacs{OWL} 2 in October 2009 with a second edition being finally released in December 2012~\cite{OWL}.
\eacs{OWL} 2 remains fully compatible to \eacs{OWL} 1, i.e. all \eacs{OWL} 1 ontologies are \eacs{OWL} 2 ontologies as well, with unchanged semantics.

Compared to \eacs{RDFS}, \eacs{OWL} introduces the following elements (among others which are omitted here):

% TODO object properties, datatype properties

% TODO simple examples?
\begin{itemize}
  \item The properties \texttt{owl:equivalentClass} is used to state that two classes are equivalent while \texttt{owl:allDisjointClasses} states that there is no individual that is an instance of more than one class from a set of classes.
  \item Similarly, the property \texttt{owl:sameAs} defines two individuals to be the same individual and \texttt{owl:differentFrom} states that two individuals can never be the same individual.
  \item Using the properties \texttt{owl:intersectionOf}, \texttt{owl:unionOf}, and \texttt{owl:complementOf} can be used to describe complex classes in a notation borrowed from set theory (e.g. if there are two classes, \texttt{Man} and \texttt{Woman}, the class \texttt{Person} can be defined as the class union of them).
  \item Using the properties \texttt{owl:allValuesFrom}, \texttt{owl:someValuesFrom}, and \texttt{owl:hasValue} classes can be defined based on the values of their properties.
  \item Using the properties \texttt{owl:minCardinality}, \texttt{owl:maxCardinality}, and \texttt{owl:cardinality}, the cardinality of properties per class can be limited.
  \item Properties can have various characteristics: A property can be inverse property of another property (\texttt{owl:inverseOf}) and two properties can be disjoint (\texttt{owl:propertyDisjointWith}). A property can be reflexive (it relates everything to itself, \texttt{owl:ReflexiveProperty}), irreflexive (no individual can be related to itself, \texttt{owl:IrreflexiveProperty}), functional (every individual can be linked to at most one other individual, \texttt{owl:FunctionalProperty}) or inverse functional (the inverse property is functional, \texttt{owl:InverseFunctionalProperty}).
\end{itemize}

% TODO citation: the first ontology war (open world assumption)
% TODO citation: monotonic reasoning
Reasoning in \texttt{OWL} respects the \emph{open world assumption}. If some statement cannot be inferred, it is not allowed to assume that this statement is false. Hence, reasoning in \texttt{OWL} is \emph{monotonic}: Adding more information to a model cannot cause anything become false that has previously known to be true, and vice versa.

% TODO citations (OWL reasoning specifications)

\eacs{OWL} ontologies are often designed using semantic editors like \protege\footnote{\href{http://protege.stanford.edu/}{http://protege.stanford.edu/}}. Common reasoners for \eacs{OWL} include \emph{Pellet}\footnote{\href{http://clarkparsia.com/pellet/}{http://clarkparsia.com/pellet/}}, \emph{RacerPro}\footnote{\href{http://semanticweb.org/wiki/RacerPro}{http://semanticweb.org/wiki/RacerPro}}, \emph{FaCT++}\footnote{\href{http://owl.man.ac.uk/factplusplus/}{http://owl.man.ac.uk/factplusplus/}} and \emph{HermiT}\footnote{\href{http://www.hermit-reasoner.com/}{http://www.hermit-reasoner.com/}}. They all implement reasoning as specified by the \eacs{OWL} specification. Implementation specific differences affect details that are not covered by the \eacs{OWL} specification. \protege and \emph{Pellet} are used to develop the weather data model for \thinkhome in chapter~\ref{ch:thinkhomeweather_ontology}.

% TODO blank nodes

\subsection{ThinkHome}

% TODO more detailed description (in the next chapter):
% repeat statements from introduction
% knowledge base:
% * name all categories of data in the knowledge base together with a short description
% * OWL, JENA
% agents:
% * name all types of agents with a short description
% * JADE, JADEX, JACK

\section{Ontologies for weather data}
\label{sec:weather_ontologies}

This section presents a selection of the ontologies that have been designed to cover the domain of weather data. Possibly one of them can be used for the weather data model of \thinkhome. If not, their advantages and disadvantages are analyzed in order to avoid their shortcomings in \thinkhome and benefit from their advantages.

% TODO other ontologies not mentioned here?

\subsection{Semantic Sensor Web}

% TODO

\subsection{SemSOS}

% TODO

\subsection{NextGen}

% TODO

\subsection{SWEET}

% TODO

\section{Related ontologies}
\label{sec:related_ontologies}

This section discusses some ontologies that cover domains that are related to weather data, e.g. location data, temporal data, or units of measurements. These are candidates for being reused by the weather data model of \thinkhome.

\subsection{Location data}
\label{subsec:location_ontologies}

% TODO define 'namespace' somewhere

\begin{figure}
\centering
\includegraphics[width=.9\textwidth]{figures/diagrams/wgs84_example.pdf}
\caption{Example of the use of the \emph{Basic Geo (WGS84 lat/long) Vocabulary}; all concepts and properties in the \texttt{geo} namespace are part of this vocabulary.}
\label{fig:wgs84_example}
\end{figure}

To handle geographical location data, the \emph{W3C Semantic Web Interest Group}\footnote{\href{http://www.w3.org/2001/sw/interest/}{http://www.w3.org/2001/sw/interest/}} developed the \emph{Basic Geo (WGS84 lat/long) Vocabulary}\cite{wgs84_vocabulary}. It introduces a concept called \emph{Spatial thing} and its attributes \egls{lat}, \egls{long} and \egls{alt} according to the \emph{WGS-84 geodetic reference system}\cite{WGS84}. Figure~\ref{fig:wgs84_example} shows an example of how the vocabulary is used.

However, the vocabulary contains some data properties that are incorrectly defined to be annotation properties. Thus, they must be redefined to be data properties whenever used in an \eacs{OWL} ontology.

\subsection{Date and time}
\label{subsec:date_ontologies}

\begin{figure}
\centering
\includegraphics[width=.9\textwidth]{figures/diagrams/owl_time_example1.pdf}
\caption{Example of the use of the \emph{OWL-Time} for describing an instant (July 16, 2012 at 14:30).}
\label{fig:owl_time_example1}
\end{figure}

\begin{figure}
\centering
\includegraphics[width=.5\textwidth]{figures/diagrams/owl_time_example2.pdf}
\caption{Example of the use of the \emph{OWL-Time} for the description of the interval of 90 minutes (one hour and 30 minutes).}
\label{fig:owl_time_example2}
\end{figure}

For specifying temporal properties, the \emph{World Wide Web consortium} (\emph{W3C}) offers a \emph{working draft} of \emph{OWL-Time}\cite{owl-time}, a \emph{Time Ontology in OWL}. It defines the concept called \Egls{temporal entity} that can either be an \Egls{instant} or an \Egls{interval}. Using appropriate attributes, the properties of a \Egls{temporal entity} are specified. Although \emph{OWL-Time} is in the state of a \emph{working draft} since September 2006, the main concepts and attributes that will be reused by other ontologies are likely to remain unchanged in future releases of that ontology.

Figure~\ref{fig:owl_time_example1} shows an example of \emph{OWL-Time} being used to specify an instant while figure~\ref{fig:owl_time_example2} demonstrates how a time interval is specified using \emph{OWL-Time}.

\subsection{Units of measurements}
\label{subsec:unit_ontologies}

As the use of units of measurement is a topic that occurs in many ontologies, there are several different approaches to cope with it.

\begin{figure}
\centering
\includegraphics[width=.4\textwidth]{figures/diagrams/nounits_example.pdf}
\caption{Example of a data model lacking units of measurement; the model represents an estate with a length of \SI{50}{metres} and a width of \SI{30}{metres}.}
\label{fig:nounits_example}
\end{figure}

\begin{figure}
\centering
\includegraphics[width=.9\textwidth]{figures/diagrams/muo_example.pdf}
\caption{Example of the use of \muo for the introduction of units of measurements.}
\label{fig:muo_example}
\end{figure}

The \emph{Measurement Units Ontology}\footnote{\href{http://idi.fundacionctic.org/muo/muo-vocab.html}{http://idi.fundacionctic.org/muo/muo-vocab.html}}\cite{MUO} is a simple and light-weight approach to enrich measurement values in an ontology with appropriate units. Some of the most important units are pre-defined, others can easily be added when needed. The \emph{Measurement Units Ontology} (\muo) is still work in progress, however it is not expected that it might change heavily in the future. Everything that will be reused by other ontologies will remain unchanged. Hence, \muo can be imported into any ontology without problems.

Figure~\ref{fig:nounits_example} shows the representation of an estate with its length and width specified using the datatype properties \texttt{hasLength} and \texttt{hasWidth}. In figure~\ref{fig:muo_example}, \muo is introduced to specify that both length and width are measured in metres: \texttt{hasLength} and \texttt{hasWidth} become object properties which link blank nodes to the individual \texttt{Estate}; \texttt{hasLength} and \texttt{hasWidth} are now both sub-properties of \texttt{muo:Quality value}. Each blank node has two properties, \texttt{muo:numerical value} states the literal value while \texttt{muo:measured in} gives the unit of measurement for the literal value. The unit (\texttt{meter} in this case) is represented by an instance of \texttt{muo:Unit of measurement}.

\begin{figure}
\centering
\includegraphics[width=.9\textwidth]{figures/diagrams/om_example.pdf}
\caption{Example of the use of \eacs{OM} for the introduction of units of measurements.}
\label{fig:om_example}
\end{figure}

The \emph{Ontology of Units of Measure and Related Concepts} (\eacs{OM})\footnote{\href{http://www.wurvoc.org/vocabularies/om-1.6/}{http://www.wurvoc.org/vocabularies/om-1.6/}}\cite{OM} is another promising approach for adding measurement units to \emph{OWL} that even provides features like the conversion between different units for the same quantities and representation and checking of formulas. Figure~\ref{fig:om_example} shows an example of the usage of \eacs{OM}. In this example, the specification the length of the estate is centered around a blank node which is an instance of \texttt{om:Length} which is a subconcept of \texttt{om:Quantity}. The blank node links to \texttt{Estate} using the property \texttt{om:phenomenon} and to another blank node via the property \texttt{om:value} which is an instance of \texttt{om:Measure}. This instance has a datatype property of type \texttt{om:numerical\_value} which specifies the numeric value of the estate's length, while the object property \texttt{om:unit} links 
to an instance of \texttt{om:Unit} named \texttt{om:meter}. The width of the estate is stated in a similar way.

\eacs{OM} is a rather large ontology (nearly \SI{2.4}{\mebi\byte} in \emph{RDF/XML syntax}) compared to the \thinkhomeweather ontology and related to the purpose it would fulfil in \thinkhomeweather; furthermore, at the time when the article about OM was published, development of \thinkhomeweather had already completed. Hence, \eacs{OM} was not taken into account for being used in the \thinkhomeweather ontology.

Besides the \emph{Measurement Units Ontology} and the \emph{Ontology of Units of Measure and Related Concepts}, several other ontologies have been examined, but all of them have shortcomings that render their use in the \thinkhomeweather ontology impossible. These ontologies are:

\begin{itemize}
  \item \textbf{\eacs{SWEET}}: Besides concepts, attributes and individuals for atmospherical phenomena, \eacs{SWEET} also comes with support for literals more precisely specified by units. However, as mentioned above, \eacs{SWEET} is an ontology that is inappropriate for use in \thinkhomeweather.
  
  \item \textbf{\eacs{QUDT}}: \eacs{QUDT}\footnote{\href{http://www.qudt.org/}{http://www.qudt.org/}}, a set of ontologies for \emph{Quantities, Units, Dimensions and Data Types in OWL and XML}, is a promising approach for adding support for units to an \emph{OWL} ontology. However, \eacs{QUDT} does not work in \protege together with the \emph{Pellet} reasoner. Using \eacs{QUDT} would require several changes to its \emph{OWL} files. Hence, it cannot be used in the \thinkhomeweather ontology.
  
  \item \textbf{\eacs{QUOMOS}}: The \emph{\eacs{OASIS} Quantities and Units of Measure Ontology Standard} (\eacs{QUOMOS})\footnote{\href{https://www.oasis-open.org/committees/tc\_home.php?wg\_abbrev=quomos}{https://www.oasis-open.org/committees/tc\_home.php?wg\_abbrev=quomos}} is a project that aims at developing ``an ontology for quantities, systems of measurement units, and base dimensions for use across multiple industries''. However, as no deliverables have been released at the time of writing, it cannot be used in \thinkhomeweather.
  
  \item \textbf{\eacs{OBO} Foundry Initiative}: The \emph{\eacs{OBO} Foundry Initiative}\footnote{\href{http://obofoundry.org/}{http://obofoundry.org/}} is an initiative that aims at collecting ontologies for use in the biomedical domain. The list of \emph{The Open Biological and Biomedical Ontologies} (abbreviated by \eacs{OBO}) includes an ontology for units of measurements\footnote{\href{http://obofoundry.org/cgi-bin/detail.cgi?id=unit}{http://obofoundry.org/cgi-bin/detail.cgi?id=unit}}\footnote{\href{http://code.google.com/p/unit-ontology/}{http://code.google.com/p/unit-ontology/}}. This ontology apparently covers most of the units that are used in the \thinkhomeweather ontology. However, it lacks any documentation and hence cannot be used in \thinkhomeweather.
\end{itemize}

\section{Conclusion}

As section~\ref{sec:weather_ontologies} discusses, unfortunately no existing ontology covers the domain of weather data in a way suitable for using it as a data model for \thinkhome. Hence, a completely new ontology is created in chapter~\ref{ch:thinkhomeweather_ontology}. However, some of the ontologies or parts of them may be reused in that new ontology; section~\ref{sec:integration} discusses this topic. Some of the ontologies that are mentioned in section~\ref{sec:related_ontologies} (the \emph{Basic Geo (WGS84 lat/long) Vocabulary}, \emph{OWL-Time} and the \emph{Measurement Units Ontology}) will be reused in any case when a model for data outside the domain of weather data is required, e.g. for location data, temporal data, or units of measurements.
