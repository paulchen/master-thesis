\chapter{Existing work}
\label{ch:existing_work}

The first part of this chapter covers the foundations the thesis builds upon. It gives introductions into all revelant topics, e.g. ontologies, \eacs{RDF} and \eacs{OWL}. Furthermore, it discusses the basic principles of \thinkhome which are required in order to fully understand the ideas behind the weather data model which is designed in later chapters.

% TODO repetition: conclude, conclusion
The second part sheds light on a selection of existing work regarding weather data in smart homes and ontologies for weather data. Additionally, some ontologies which may be reused in an ontology for weather data are discussed. The final section of this chapter reviews all ontologies that are presented and identifies elements can be used for \thinkhome.

\section{Foundations}

This chapter presents the foundations the work in this thesis builds upon: The concept of an ontology, the \emph{Web Ontology Language} (\eacs{OWL}) and \thinkhome.

\subsection{Ontologies}

% TODO parts of ontologies: concepts, attributes, relations, individuals
% TODO perhaps: function terms, restrictions, rules, axioms, events
% TODO reasoning
% TODO reuse, vocabulary
In computer science, an ontology represents knowledge as a set of concepts in a certain domain and relationships between pairs of concepts.

% TODO citations

Ontologies are encoded using formal languages. There are many of these \emph{ontology languages} such as \eacs{KIF} (\emph{Knowledge Interchange Format}), \emph{DAML+OIL} -- a successor to \eacs{DAML} (\emph{DARPA Agent Markup Language}) and \eacs{OIL} (\emph{Ontology Interchange Language})) that combines features of both --, \eacs{RDFS} (\emph{RDF Schema}), and \eacs{OWL}. The latter, the \emph{Web Ontology Language}, is used in the \thinkhome project; thus, section~\ref{subsec:owl} gives a brief introduction into \eacs{OWL}.

Over the years, many ontologies have been developed. Some of them define concepts and relations that can be found across many knowledge domains. These ontologies are often imported into other ontologies in order to minimize the effort of creating new ontologies and to simplify interoperability of ontologies. Examples for such ontologies that are implemented in \eacs{OWL} or \emph{RDF Schema} are:
\begin{itemize} % TODO citations
  \item \textbf{DOAP}: … % TODO
  \item \textbf{Dublin Core}: … % TODO
  \item \textbf{FOAF}: … % TODO
  \item \textbf{SIOC}: … % TODO
  \item \textbf{SKOS}: … % TODO
  \item \textbf{UMBEL}: … % TODO
\end{itemize}

\subsection{OWL}
\label{subsec:owl}

The \emph{Resource Description Framework} (\eacs{RDF}) is a standard model for knowledge representation~\cite{RDF}. It is specified in a set of recommendations by the \emph{World Wide Web Consortium} (\eacs{W3C})\footnote{\href{http://www.w3.org/}{http://www.w3.org/}}.

The data model of \eacs{RDF} incorporates element types named \emph{resources} and \emph{properties} which can also be found in ontologies: A resource is some thing that is described by \eacs{RDF}, it corresponds to an individual as it is found in an ontology. Each resource can have an arbitrary number of properties, i.e. \emph{attributes} that associate \emph{literate} values (e.g. numerical values, strings) to the resource or \emph{relations} that link this resource to other resources. Resources and properties are expressed using \emph{statements} (\emph{triples}) which consist of three parts called \emph{subject}, \emph{predicate} and \emph{object}. To identify resources and properties, \eacs{RDF} uses \eacss{URI} (\emph{Unified Resource Locators}). \eacss{URI} are strings defined by \emph{\eacs{RFC} 3986}~\cite{rfc3986} that uniquely identify things.

\begin{figure}
\centering
\includegraphics[width=.9\textwidth]{figures/diagrams/rdf_example.pdf}
\caption{Example of a simple \eacs{RDF} model}
\label{fig:rdf_example}
\end{figure}

Figure~\ref{fig:rdf_example} depicts a simple example for a piece of knowledge from the domain of weather data expressed using \eacs{RDF}: The resource \texttt{weather:CurrentWeather} represents the current state of the weather. The property \texttt{geo:hasLocation} links the \texttt{weather:CurrentWeather} to the resource \texttt{geo:Vienna} which represents the city of Vienna, Austria; i.e. \texttt{weather:CurrentWeather} describes the weather for Vienna. \texttt{weather:weatherState} has two more properties, \texttt{weather:hasTemperatureValue} and \texttt{weather:hasHumidityValue}, which link two literal values to the resource: a temperature value of \SI{18.2}{\celsius} and a humidity value of $72 \%$. The type \texttt{xsd:float} of both literals is defined by \emph{XML Schema}~\cite{xml-schema,xml-schema-datatypes}, one of the several \eacs{XML} schema languages available that define the structure of \eacs{XML} documents.

The complete \eacs{URI} of \texttt{weather:CurrentWeather} is \texttt{http://example.org/weather\#CurrentWeather} and the \eacs{URI} of \texttt{geo:Vienna} is \texttt{http://example.org/geo\#Vienna}. As in \eacs{XML}, parts of \eacss{URI} may be replaced by \emph{prefixes} to avoid frequent recurrence of the same strings. For the above example, the prefix \texttt{weather} has been defined to replace the string \texttt{http://example.org/weather\#} and \texttt{geo} replaces \texttt{http://example.org/geo\#}. This results in the identifiers \texttt{weather:CurrentWeather} and \texttt{geo:Vienna} that can be found in figure~\ref{fig:rdf_example}.

For expressing the data that is represented by an \eacs{RDF} model, several serialization formats are available. The \eacs{RDF} recommendation is based on \emph{RDF/XML} which maps the \eacs{RDF} model to an \eacs{XML} document~\cite{RDF_XML}. The representation of the above example in \emph{RDF/XML} can be seen in figure~\ref{fig:rdfxml_example}. As \emph{RDF/XML} is a rather verbose format which may be difficult to read for humans, the \eacs{N3} (\emph{Notation3}) representation for \eacs{RDF} is available which was developed with human-readability in mind~\cite{Notation3}. \emph{Notation3} incorporates some syntax features that go beyond the expressive power of \eacs{RDF}. A subset of \emph{Notation3} named \eacs{Turtle} (\emph{Terse RDF Triple Language}) is available that is limited to the features required to map \eacs{RDF} models~\cite{Turtle}. Figure~\ref{fig:turtle_example} shows the above example in \eacs{Turtle} syntax.

\begin{figure} % TODO syntax highlighting
\begin{lstlisting}
<?xml version="1.0"?>
<rdf:RDF xmlns:xsd="http://www.w3.org/2001/XMLSchema#" xmlns:geo="http://example.org/geo#" xmlns:rdf="http://www.w3.org/1999/02/22-rdf-syntax-ns#" xmlns:weather="http://example.org/weather#">
	<rdf:Description rdf:about="http://example.org/weather#CurrentWeather">
		<weather:hasTemperatureValue rdf:datatype="http://www.w3.org/2001/XMLSchema#float">18.2</weather:hasTemperatureValue>
		<weather:hasHumidityValue rdf:datatype="http://www.w3.org/2001/XMLSchema#float">.72</weather:hasHumidityValue>
		<geo:hasLocation rdf:resource="http://example.org/geo#Vienna" />
	</rdf:Description>
</rdf:RDF>
\end{lstlisting}
\caption{\eacs{RDF} example from figure~\ref{fig:rdf_example} encoded in \emph{RDF/XML} syntax}
\label{fig:rdfxml_example}
\end{figure}

\begin{figure}
\begin{lstlisting}
@prefix rdf: <http://www.w3.org/1999/02/22-rdf-syntax-ns#> .
@prefix weather: <http://example.org/weather#> .
@prefix geo: <http://example.org/geo#> .
@prefix xsd: <http://www.w3.org/2001/XMLSchema#> .

weather:CurrentWeather weather:hasTemperatureValue "18.2"^^xsd:float ;
    weather:hasHumidityValue ".72"^^xsd:float .

weather:CurrentWeather geo:hasLocation geo:Vienna .
\end{lstlisting}
\caption{\eacs{RDF} example from figure~\ref{fig:rdf_example} encoded in \eacs{Turtle} syntax}
\label{fig:turtle_example}
\end{figure}

\eacs{RDF} \emph{schema} (\eacs{RDFS}) is a recommendation by the \eacs{W3C} that builds upon \eacs{RDF}~\cite{RDFS}. It introduces a set of concepts and properties that permits the definition of simple ontologies. 

All things described by \eacs{RDF} are instances of \texttt{rdfs:Resource}, concepts are instances of \texttt{rdfs:Class}, and properties are instances of \texttt{rdfs:Property}. Other concepts introduced by \eacs{RDFS} are \texttt{rdfs:Literal}, \texttt{rdfs:Datatype} and \texttt{rdfs:XMLLiteral}.

The property \texttt{rdfs:domain} states that any resource that has a given property is an instance of one or more classes; the property \texttt{rdfs:range} states that the value of a property is an instance of one or more classes. \texttt{rdf:type} (often abbreviated with \texttt{a}) is used to state that a resource is an instance of a class. Hierarchies of classes can be constructed using the property \texttt{rdfs:subClassOf}: \texttt{C1 rdfs:subClassOf C2} states that any instance of \texttt{C2} is also an instance of \texttt{C1}. \texttt{rdfs:subPropertyOf} is an equivalent that is used for declaring hierachies of properties. Other properties defined by \eacs{RDFS} are \texttt{rdfs:label} and \texttt{rdfs:comment}.

\begin{figure}
\centering
\includegraphics[width=.9\textwidth]{figures/diagrams/rdfs_example.pdf}
\caption{Example of a simple \eacs{RDFS} model}
\label{fig:rdfs_example}
\end{figure}

Figure~\ref{fig:rdfs_example} shows the example from figure~\ref{fig:rdf_example}, enriched by some elements that are defined by \eacs{RDFS}. The data model introduces two classes, \texttt{weather:WeatherState} and \texttt{geo:Location}. \texttt{weather:CurrentWeather} and \texttt{geo:Vienna} are instances of \texttt{weather:WeatherState} and \texttt{geo:Location}, respectively. \texttt{geo:hasLocation} is a property with range \texttt{weather:WeatherState} and \texttt{geo:Location}.

\eacs{RDFS} comes with reasoning support~\cite{RDF_semantics}; e.g. in the above example, the statements

\begin{verbatim}
weather:CurrentWeather rdf:type weather:WeatherState .
geo:Vienna rdf:type geo:Location .
\end{verbatim}

can be removed without loss of knowledge. A reasoner can deduct them from these statements:

\begin{verbatim}
weather:CurrentWeather geo:hasLocation geo:Vienna .
geo:location rdfs:domain weather:WeatherState .
geo:location rdfs:range geo:Location .
\end{verbatim}
However, as the expressive power of \eacs{RDFS} is rather limited compared to what \eacs{OWL} has to offer, reasoning in \eacs{RDFS} does not have the significance it has in \eacs{OWL}.

Many of the concepts and properties defined are included in the \emph{Web Ontology Language} (\eacs{OWL}), a more expressive ontology language than \eacs{RDFS} based on \eacs{RDF} and \eacs{RDFS}. \eacs{OWL} is developed by the \emph{OWL Working Group}\footnote{\href{http://www.w3.org/2007/OWL/wiki/OWL\_Working\_Group}{http://www.w3.org/2007/OWL/wiki/OWL\_Working\_Group}} of the \eacs{W3C}. \eacs{OWL} 1 was first published in July 2002 as a \emph{working draft} and became a \eacs{W3C} \emph{recommendation} in February 2004\footnote{Both \emph{working draft} and \emph{recommendation} are \emph{maturity levels} proposed by the \eacs{W3C} that indicate the state of their technical reports~\cite{w3c-process}.}~\cite{OWL1}; the first \emph{working draft} of \eacs{OWL} 2 in March 2009 and the \eacs{W3C} \emph{recommendation} of \eacs{OWL} 2 in October 2009 with a second edition being finally released in December 2012~\cite{OWL}. \eacs{OWL} 2 remains fully compatible to \eacs{OWL} 1, i.e. all \eacs{OWL} 1 ontologies are also \eacs{OWL} 2 ontologies, with unchanged semantics.

Compared to \eacs{RDFS}, \eacs{OWL} introduces the following elements (among others which are omitted here):

% TODO simple examples?
\begin{itemize}
  \item The properties \texttt{owl:equivalentClass} is used to state that two classes are equivalent while \texttt{owl:allDisjointClasses} states that there is no individual that is an instance of more than one class from a set of classes.
  \item Similarly, the property \texttt{owl:sameAs} defines two individuals to be the same individual and \texttt{owl:differentFrom} states that two individuals can never be the same individual.
  \item Using the properties \texttt{owl:intersectionOf}, \texttt{owl:unionOf}, and \texttt{owl:complementOf} can be used to describe complex classes in a notation borrowed from set theory (e.g. if there are two classes, \texttt{Man} and \texttt{Woman}, the class \texttt{Person} can be defined as the class union of them).
  \item Using the properties \texttt{owl:allValuesFrom}, \texttt{owl:someValuesFrom}, and \texttt{owl:hasValue} classes can be defined based on the values of their properties.
  \item Using the properties \texttt{owl:minCardinality}, \texttt{owl:maxCardinality}, and \texttt{owl:cardinality}, the cardinality of properties per class can be limited.
  \item Properties can have various characteristics: A property can be inverse property of another property (\texttt{owl:inverseOf}) and two properties can be disjoint (\texttt{owl:propertyDisjointWith}). A property can be reflexive (it relates everything to itself, \texttt{owl:ReflexiveProperty}), irreflexive (no individual can be related to itself, \texttt{owl:IrreflexiveProperty}), functional (every individual can be linked to at most one other individual, \texttt{owl:FunctionalProperty}) or inverse functional (the inverse property is functional, \texttt{owl:InverseFunctionalProperty}).
\end{itemize}

% TODO citation: the first ontology war (open world assumption)
% TODO citation: monotonic reasoning
Reasoning in \texttt{OWL} respects the \emph{open world assumption}. If some statement cannot be inferred, it is not allowed to assume that this statement is false. Hence, reasoning in \texttt{OWL} is \emph{monotonic}: Adding more information to a model cannot cause anything become false that has previously known to be true, and vice versa.

% TODO citations (OWL reasoning specifications)

\eacs{OWL} ontologies are often designed using semantic editors like \protege\footnote{\href{http://protege.stanford.edu/}{http://protege.stanford.edu/}}. Common reasoners for \eacs{OWL} include \emph{Pellet}\footnote{\href{http://clarkparsia.com/pellet/}{http://clarkparsia.com/pellet/}}, \emph{RacerPro}\footnote{\href{http://semanticweb.org/wiki/RacerPro}{http://semanticweb.org/wiki/RacerPro}}, \emph{FaCT++}\footnote{\href{http://owl.man.ac.uk/factplusplus/}{http://owl.man.ac.uk/factplusplus/}} and \emph{HermiT}\footnote{\href{http://www.hermit-reasoner.com/}{http://www.hermit-reasoner.com/}}. They all implement reasoning as specified by the \eacs{OWL} specification. Implementation specific differences affect details that are not covered by the \eacs{OWL} specification. \protege and \emph{Pellet} are used to develop the weather data model for \thinkhome in chapter~\ref{ch:thinkhomeweather_ontology}.

% TODO blank nodes

\subsection{ThinkHome}

% TODO

\section{Ontologies for weather data}

% TODO weather data

\subsection{Semantic Sensor Web}

% TODO

\subsection{SemSOS}

% TODO

\subsection{NextGen}

% TODO

\subsection{SWEET}

% TODO

\section{Related ontologies}

% TODO

\subsection{SWIG Basic Geo (WGS84 lat/long) Vocabulary}

% TODO

\subsection{OWL-Time}

% TODO

% TODO Units ontologies

\section{Conclusion}
