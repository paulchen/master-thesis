\chapter{Conclusion}
\label{ch:conclusion}
% TODO unsourced?
% TODO uncomment
% \chapterquotation{If you want a happy ending, that depends, of course, on where you stop your story.}{Orson Welles}

% TODO extend: Ich finde die Conclusion gehört nochmals überarbeitet. Hierbei sollten nicht nur die Chapters sowie ihr Inhalt nochmals präsentiert werden, sondern vielmehr ein Resümee gezogen werden, was mit dieser Diplomarbeit erreicht wurde, sowie nochmals der "main outcome" in Kurzfassung beschrieben werden. Der Inhaltsüberblick über die einzelnen Chapter kann gerne drin bleiben, jedoch muss die Conclusion um die genannte Zusammenfassung erweitert werden.


\section{Summary}

This thesis a weather data model based on an \eacs{OWL} ontology, a Java application for importing data into that model, and all prerequisites leading to these two outcomes.

After chapter~\ref{ch:intro} gives an introduction into the motivation behind the thesis and describes the problem statement, the intended goal and the methodological approach, chapter~\ref{ch:existing_work} discusses the foundations the thesis builds upon: Ontologies, ontology related technologies like \eacs{RDF}, \eacs{RDFS}, or \eacs{OWL}, \thinkhome, already existing ontologies for weather data, and ontologies that cover information related to the domain covered by \smarthomeweather. Chapter~\ref{ch:weather_data} then focuses on weather data that is available from both locally installed weather sensors as well as from weather services accessible via Internet. This chapter then identifies a set of weather phenomena which \smarthomeweather shall cover and determines further details about the domain of weather data as used in the present context.
Of the weather services being discussed, \yrno is selected for providing a reference implementation for the import of weather data into \smarthomeweather in a later step. Chapter~\ref{ch:development_approaches} sheds light on five different approaches towards the development of new ontologies from scratch. The approaches are compared to each other and one of them, \methontology, is identified as the one that fits the requirements for designing \smarthomeweather best.

Eventually, chapter~\ref{ch:smarthomeweather_ontology} applies \methontology to \smarthomeweather and describes every step in a detailed manner. As the data model itself does not contain any weather data, in chapter~\ref{ch:weather_importer} the \emph{Weather Importer}, a Java application, is developed which accesses \yrno to retrieve weather data for a certain location, transforms the data being obtained and adds them to the data model of \smarthomeweather. Furthermore, the \emph{Weather Importer} includes a comprehensive set of \emph{JUnit} test cases which ensure that \smarthomeweather and the \emph{Weather Importer} work as expected.

\vspace{1em}

\section{Outlook}

The current version of \smarthomeweather comes with a few shortcomings that require future work in order to resolve them. Not all of these problems lie in the scope of \smarthomeweather. The main problems arising are:

\begin{itemize}
  \item There are a few situations that expose bugs in \protege and the \emph{Pellet} reasoner. While it is possible to develop ontologies using these technologies which cover a certain domain, it is hardly avoidable to run into bugs that manifest themselves in the form of incomprehensible error messages. In that case, the ontology needs to be modified slightly to work around these bugs. Unfortunately, during the development of \smarthomeweather, it has not been possible to track these bugs down in order to find their reason and to fix them. Future work that resolves these bugs may ease the work with \protege, \emph{Pellet}, and \smarthomeweather.
  
  \item At the time of writing, \emph{OWL-Time}~\cite{owl-time} (see section~\ref{subsec:date_ontologies}) has not reached the state of being a \emph{\eacs{W3C} recommendation}~\cite{w3c-process}; since it was first published more than six years ago, it is a \emph{working draft}. Although it can be assumed that the core concepts and relations defined by \emph{OWL-Time} will not change regarding their syntax and semantics, using technologies that are not yet published as being in a stable state always leaves a stale taste. % TODO rephrase "to leave a stale taste"
  
  \item Similar problems arise from the use of the \emph{Basic Geo (\eacs{WGS84} lat/long) Vocabulary}~\cite{wgs84_vocabulary}. This technology has not even been submitted to the \emph{\eacs{W3C} recommendation track} for standardisation. Furthermore, no work on the \emph{\eacs{WGS84} vocabulary} itself has been done since 2006. Further work by the \emph{\eacs{W3C} Geospatial Incubator Group} did not lead to any standards (see~\ref{subsec:location_ontologies}).
\end{itemize}

Furthermore, \smarthomeweather suffers performance issues regarding the time required for reasoning. In test runs that were conducted after development, complete reasoning in \protege using the \emph{Pellet reasoner} of the ``empty'' ontology (not containing any individuals denoting weather data) took between $15$ and $30$ seconds and between $45$ and $60$ seconds for the ontology that contains weather data imported using the \emph{Weather Importer} from chapter~\ref{ch:weather_importer} (on the PC used for development which is equipped with a \emph{Intel Q6600 CPU}~\cite{intel_q6600} running \emph{Ubuntu Linux}~\cite{ubuntu}).

One reason for these performance issues is the use of the \muo ontology which increases the reasoning time by about $30 \%$ (see section~\ref{sec:ontology_imports}). Abandonment of this ontology would speed up refactoring, though this would introduce problems regarding literal values without a unit. There are other ontologies which may be used instead of the \muo ontology (see section~\ref{subsec:unit_ontologies}), such as the \eacs{OM} ontology. However, as the \eacs{OM} ontology adds about the same level of complexity to the ontology, it is certain that it increases the reasoning time of the smart home's knowledge base to the same extent as \muo. It is questionable whether there is a unit ontology that allows faster reasoning than \muo.

Further performance optimisations may be possible by modifying the internal structure of \smarthomeweather without changing name and semantics of externally accessed concepts. E.g. concepts such as \Egls{weather report}, \Egls{weather state}, and \Egls{weather phenomenon} together with their respective sub-concepts remain part of the ontology while the definitions of these concepts are modified to allow faster reasoning. However, at the present time it is unknown to what degree performance gains are possible using this approach.

\vspace{1cm}

Nevertheless, in its current state, \smarthomeweather represents an ontology that fully complies with its specification of covering current and future weather data as far as possible. Eventually, the employment of \smarthomeweather in environments other than smart homes is also imaginable, whereever current and future weather data is used.
