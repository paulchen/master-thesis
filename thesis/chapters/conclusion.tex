\chapter{Conclusion}
\label{ch:conclusion}
% TODO unsourced?
% TODO uncomment
% \chapterquotation{If you want a happy ending, that depends, of course, on where you stop your story.}{Orson Welles}

\section{Summary}

This thesis describes a weather data model based on an \eacs{OWL} ontology, a Java application for importing data into that model, and all prerequisites leading to these two outcomes.

After Chapter~\ref{ch:intro} gives an introduction into the motivation behind the thesis and describes the problem statement, the intended goal, and the methodological approach, Chapter~\ref{ch:existing_work} discusses the foundations the thesis builds upon: Ontologies, ontology related technologies like \eacs{RDF}, \eacs{RDFS}, or \eacs{OWL}, \thinkhome, already existing ontologies for weather data, and ontologies that cover information related to the domain covered by \smarthomeweather. Chapter~\ref{ch:weather_data} then focuses on weather data that is available from both locally installed weather sensors as well as from weather services accessible via Internet. This chapter then identifies a set of weather phenomena which \smarthomeweather shall cover and determines further details about the domain of weather data as used in the present context.
Of the weather services being discussed, \yrno is selected for providing a reference implementation for the import of weather data into \smarthomeweather in a later step. Chapter~\ref{ch:development_approaches} sheds light on five different approaches towards the development of new ontologies from scratch. The approaches are compared to each other and one of them, \methontology, is identified as the one that fits the requirements for designing \smarthomeweather best.

Eventually, Chapter~\ref{ch:smarthomeweather_ontology} applies \methontology to \smarthomeweather and describes every step in a detailed manner. As the data model itself does not contain any weather data, in Chapter~\ref{ch:weather_importer} the \emph{Weather Importer}, a Java application, is developed which accesses \yrno to retrieve weather data for a certain location, transforms the data being obtained, and adds them to the data model of \smarthomeweather. Furthermore, the \emph{Weather Importer} includes a comprehensive set of \emph{JUnit} test cases which ensure that \smarthomeweather and the \emph{Weather Importer} work as expected.

\vspace{1em}

As the overall goal of this thesis, \smarthomeweather represents a comprehensive ontological model for current and future weather data which is limited to aspects that are reasonable for the use in the context smart homes. It is centered around a set of five top-level concepts, \Egls{weather condition}, \Egls{weather phenomenon}, \Egls{weather report}, \Egls{weather source}, and \Egls{weather state}. Instances of these elements represent the current weather situation together with a forecast for the upcoming 24 hours. Depending on their respective values, instances representing weather elements (i.e.\ instances of \Egls{weather phenomenon}) are grouped into categories such as \egls{room temperature} or \egls{heavy rain}, allowing simple and straight-forward identification of certain weather properties. Furthermore, some combinations of instances of \Egls{weather phenomenon} are combined into a set of instances of \Egls{weather state} to help allow simple identification of certain weather situations, e.g.\ \Egls{fair weather} or \Egls{severe weather}.

\smarthomeweather is built with simple and efficient \eacs{OWL} reasoning in mind; together with \eacs{SWRL} rules and \eacs{SPARQL} queries, the reasoning within \smarthomeweather can provide answers to a predefined set of competency questions that cover various weather-related aspects in and around a dwelling. Additionally, due to the extent of weather data being covered, \smarthomeweather may be able to answer other questions that have not yet been considered.

Besides the work on \smarthomeweather itself, the thesis carries out extensive research regarding methodologies for developing ontologies. Five different approaches towards creating new ontologies from scratch are outlined, their characteristics are identified, and their suitability for applying them to the domain of \smarthomeweather is evaluated. While \methontology turns out to be best-fitting approach in the context of \smarthomeweather, the other methodologies are also well-known approaches which would probably lead to results of similar quality as \methontology does.

Nine different Internet weather services are evaluated regarding the ability to provide viable data to a smart home. During this evaluation, many problems regarding weather services are enumerated which make their use difficult.  Using the example of the \emph{KNX} fieldbus, available weather sensors are enumerated. Taking the knowledge about weather services and sensors into account, a set of weather elements is compiled; knowledge about the state of any of these weather elements helps controlling smart homes. While most of these weather elements (atmospheric pressure, cloud coverage, dew point, precipitation, relative humidity, solar radiation, temperature, and wind) can be taken from weather services or sensors, the dew point value can also be derived from values of other elements; the position of the sun is determined algorithmically from time and location.

\smarthomeweather tries to reuse existing ontologies wherever possible. However, many ontologies have been found that are incomplete, unsuitable, or lack documentation; many projects appear to have been put on ice. Hence, the set of ontologies used by \smarthomeweather melts down to \emph{OWL-Time} (for temporal specifications), the \emph{Basic Geo (WGS84 lat/long) Vocabulary} (for geographic locations), and \muo (for units of measurement).

\section{Outlook}

\smarthomeweather provides two areas where future work can done: These are the elimination of its current shortcomings and the search for further uses of the data it provides.

\subsubsection{Shortcomings in the current version}

The current version of \smarthomeweather comes with a few shortcomings that require future work in order to resolve them. Not all of these problems lie in the scope of \smarthomeweather. The main problems arising are:

\begin{itemize}
  \item There are a few situations that expose bugs in \protege and the \emph{Pellet} reasoner. While it is possible to develop ontologies using these technologies which cover a certain domain, it is hardly avoidable to run into bugs that manifest themselves in the form of incomprehensible error messages. In that case, the ontology needs to be modified slightly to work around these bugs. Unfortunately, during the development of \smarthomeweather, it has not been possible to track these bugs down in order to find their reason and to fix them. Future work that resolves these bugs may ease the work with \protege, \emph{Pellet}, and \smarthomeweather.
  
  \item At the time of writing, \emph{OWL-Time}~\cite{owl-time} (see Section~\ref{subsec:date_ontologies}) has not reached the state of being a \emph{\eacs{W3C} recommendation}~\cite{w3c-process}; since it was first published more than six years ago, it has remained to be a \emph{working draft}. Although it can be assumed that the core concepts and relations defined by \emph{OWL-Time} will not change regarding their syntax and semantics, it is still work in progress and therefore may change or vanish without prior notice.
  
  \item Similar problems arise from the use of the \emph{Basic Geo (\eacs{WGS84} lat/long) Vocabulary}~\cite{wgs84_vocabulary}. This technology has not even been submitted to the \emph{\eacs{W3C} recommendation track} for standardisation. Furthermore, no work on the \emph{\eacs{WGS84} vocabulary} itself has been done since 2006. Further work by the \emph{\eacs{W3C} Geospatial Incubator Group} did not lead to any standards (see Section~\ref{subsec:location_ontologies}).
\end{itemize}

Furthermore, \smarthomeweather suffers performance issues regarding the time required for reasoning. In test runs that were conducted after development, complete reasoning in \protege using the \emph{Pellet reasoner} of the ``empty'' ontology (not containing any individuals denoting weather data) took between $15$ and $30$ seconds and between $45$ and $60$ seconds for the ontology that contains weather data imported using the \emph{Weather Importer} from Chapter~\ref{ch:weather_importer} (on the PC used for development which is equipped with a \emph{Intel Q6600 CPU}~\cite{intel_q6600} running \emph{Ubuntu Linux}~\cite{ubuntu}).

One reason for these performance issues is the use of the \muo ontology which increases the reasoning time by about $30 \%$ (see Section~\ref{sec:ontology_imports}). Abandonment of this ontology would speed up refactoring, though this would introduce problems regarding literal values without a unit. There are other ontologies which may be used instead of the \muo ontology (see Section~\ref{subsec:unit_ontologies}), such as the \eacs{OM} ontology. However, as the \eacs{OM} ontology adds about the same level of complexity to the ontology, it is certain that it increases the reasoning time of the smart home's knowledge base to the same extent as \muo. It is questionable whether there is a unit ontology that allows faster reasoning than \muo.

Further performance optimisations may be possible by modifying the internal structure of \smarthomeweather without changing name and semantics of externally accessed concepts. E.g. concepts such as \Egls{weather report}, \Egls{weather state}, and \Egls{weather phenomenon} together with their respective sub-concepts remain part of the ontology while the definitions of these concepts are modified to allow faster reasoning. However, at the present time it is unknown to what degree performance gains are possible using this approach.

Nevertheless, its current state, \smarthomeweather represents an ontology that fully complies with its specification of covering current and future weather data as far as possible. \smarthomeweather may even provide answers to many questions that have not been considered during its development. Eventually, the employment of \smarthomeweather in environments other than smart homes is also imaginable, wherever current and future weather data is used.

\subsubsection{Further uses of data provided by \smarthomeweather}

\smarthomeweather also provides a context for future work to further improve the control of smart homes. One possible starting point is the identification of further aspects in smart homes that are related to the weather, particularly in conjunction with other sources of data. These aspects may be covered by another set of questions similar to the competency questions presented in Section~\ref{sec:weather_information}; some possible examples are:

\begin{itemize}
  \item Assuming that the price for electrical energy varies over time (smart metres capturing the power consumption of a building in intervals of a few seconds, minutes, or hours gain increasing popularity\footnote{In Austria, 95 percent of all households are expected to be equipped with smart metres by 2019~\cite{smart_metres_austria}.}), how can energy consumption and costs be minimised through most efficiently using the power provided by solar panels?
  
  \item How can knowledge about the times of presence and absence of the building's inhabitants together with a weather forecast over 24 hours lead to more efficient \eacs{HVAC} control?
  
  \item Can the building learn from the influences of weather on the building (e.g.\ sunshine heating up a room) to more efficiently control processes in the future (e.g.\ turn off the heating in a room in advance the sun may heat it up soon)?
\end{itemize}

Due to the extent of weather data covered by \smarthomeweather, the ontology may be able to provide data for many or even all future use cases. Whatever future research may result in -- smart homes and especially weather-related control will remain an interesting topic for many years.
