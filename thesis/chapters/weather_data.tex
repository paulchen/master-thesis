\chapter{Weather data}
\label{ch:weather_data}

Based on the insights acquired from existing work in Chapter~\ref{ch:existing_work}, this chapter aims at compiling a set of weather data elements which are either necessary for providing useful data on exterior influences to smart home systems or which would add benefit to the data provided. Furthermore, possible sources are evaluated with respect to their suitability for the given context.

\section{Weather information}
\label{sec:weather_information}

In order to identify which data is required for the \smarthomeweather ontology, it is necessary to define the scope that shall be covered by the ontology. When designing an ontology, requirements analysis is often centered around a set of \emph{competency questions}~\cite{Ontology101,UscholdKing,GruningerFox,Methontology}. If the ontology is able to provide answers to all of these competency questions, its requirements are met.

Below are the questions that have been identified to be adequate competency questions for \smarthomeweather; the list stems from analysis of the processes at a smart home that may be influenced by weather.

\begin{itemize}
  \item What is the current weather situation?
  \item What will the weather situation be in one hour, in two hours, …, in 24 hours?
  \item What is the current temperature, humidity, wind speed, …?
  \item What will be the temperature, humidity, wind speed, … in one hour, in two hours, …, in 24 hours?
  \item What will be the minimum temperature, humidity, … over the next 24 hours? What are the maximum values?
  \item Will the weather change? Will the temperature, humidity, … rise or fall?
  \item Does it rain? Will it rain in the next hours? Will it rain today?
  \item Will there be sunshine today? 
  \item Do we need to irrigate the garden?
  \item Will there be severe weather?
  \item Will temperature drop/stay below \SI{0}{\celsius}?
  \item When can we open windows and when do we have to keep them shut?
  \item When do we need sun protection?
  \item When will it outside be colder than inside the house? When will it be warmer?
\end{itemize}

These competency questions will again be used in Section~\ref{sec:ontology_specification} when the \emph{Ontology requirements specification document} is created.

The idea of providing a smart home with future weather data is to enable the system to prepare for upcoming weather situations. There are Internet weather services that provide forecasts for several days (see Section~\ref{sec:internet_services}). However, the further the time described by a forecast lies in the future, the more inaccurate the forecast becomes~\cite{forecast_error1,forecast_error2}. Additionally, no decisions in a smart home have been identified that require the availability of a forecast about a time more than a few hours in the future. Hence, the period of 24 hours has been chosen as a compromise between the deteriorating accuracy of weather forecasts over time and the time period smart homes require weather data for.

The above competency questions can be answered when the (predicted) state of the weather for particular points of time is known. The state of the weather is given by measurement values of certain \emph{weather elements}. These weather elements are temperature, relative and absolute humidity, dew point temperature, wind speed, wind direction, precipitation, cloud coverage, and others.~\cite{GlossaryOfMeteorology}

A set of measurement value for one or more points of time is called \emph{weather state} from now on.

Apart from weather elements, Internet weather services often provide some information that will be called \emph{weather condition} from now on. Generally speaking, the weather condition is a one-word description of the current weather situation. Examples for the general weather conditions are ``Sun'', ``Rain'', and ``Fog''. Some weather conditions can be split up into several conditions, e.g.\ ``It is overcast and raining'' into ``Overcast'' and ``Rain''.

Sources for weather data are weather sensors and weather forecasts. As smart homes perform no weather forecasting on its own, forecast data is gathered from weather services via Internet. While data about the current weather state can be obtained from both weather sensors and Internet services, the only source for data about future weather states are Internet services.

Section~\ref{sec:weather_sensors} covers data that can be obtained from weather sensors; Section~\ref{sec:weather_services} discusses weather data that can be fetched from Internet services. Based on the findings in these two sections Section~\ref{sec:weather_conclusion} presents a set of weather elements that are incorporated into \smarthomeweather.

\section{Sensor data}
\label{sec:weather_sensors}

While there are stand-alone solutions for fetching weather data from sensors, in many \emph{home automation} systems \emph{fieldbus} systems are used.

\subsection{\emph{Fieldbus} systems}
\label{sec:fieldbus}

Over the past few years, several of \emph{fieldbus} systems have emerged in the context of \emph{home automation}. A \emph{fieldbus} is a network system for real-time distributed control~\cite{fieldbus}, standardised as \emph{IEC~61158}~\cite{IEC61158}. A \emph{Programmable Logic Controller} (\eacs{PLC}), a computer which is designed specifically for automation purposes, can utilise a \emph{fieldbus} to retrieve sensor data from \emph{sensors} and send control commands to \emph{actuators}. Among the sensors that are available for \emph{fieldbus} systems, there exist sensors for a wide variety of weather elements such as temperature or humidity. 

There are several competing standards for \emph{fieldbus} systems, including:
\begin{itemize}
  \item \textbf{KNX:} \emph{KNX} is an international standard (\emph{ISO/IEC 14543-3}~\cite{knx,knx_standard}) which is the successor to three previously used \emph{fieldbus} systems, namely \emph{European Home Systems Protocol} (\eacs{EHS}), \emph{BatiBUS}, and \emph{European Installation Bus} (\eacs{EIB}). It supports the use of several physical communication media such as \emph{twisted pair} wiring, \emph{power line} networking, \emph{Ethernet}, \emph{infrared}, or \emph{radio}. There is a wide range of devices that can be used for controlling a \emph{KNX} network.
  
  \item \textbf{LonWorks:} Developed by \emph{Echelon Corporation}~\cite{echelon} in 1990, \emph{LonWorks} (\emph{Local Operating Network})~\cite{lonworks} is a widely used fieldbus standard; since 2008, the technology is standardised as \emph{ISO/IEC~14908}~\cite{ISO14908}. \emph{LonWorks} supports \emph{twisted pair} cabling and \emph{power line} networking and can be controlled by any general-purpose processor which can use the \emph{LonTalk} protocol that is employed by \emph{LonWorks}.
  
  \item \textbf{BACnet:} \eacs{BACnet} (\emph{building automation and control networks})~\cite{bacnet} is another communications protocol which is internationally standardised in \emph{ISO~16484-5}~\cite{ISO16484}. It employs several communication means such as \emph{Ethernet}, \emph{BACnet/IP}, \emph{Point-to-point connections} over \emph{RS-232}, or \emph{LonWorks}' \emph{LonTalk}.
  
  \item \textbf{LCN:} \eacs{LCN} (\emph{Local Control Network})~\cite{LCN} is a proprietary home automation system developed by \emph{Issendorf KG} in 1992. \eacs{LCN} is organised into \emph{\eacs{LCN} modules} that exchange data over a single data wire and the neutral wire power supply.
\end{itemize}

\subsection{\emph{KNX} sensors}

Each of the \emph{fieldbus} systems mentioned in Section~\ref{sec:fieldbus} provides a variety of sensors. \emph{KNX}, which is an open standard, is in this section used as an example to illustrate the weather sensors that are available for \emph{fieldbus} systems.

\emph{KNX} weather sensors are available for
\begin{itemize}
  \item atmospheric pressure (\emph{barometer}),
  \item brightness (\emph{photometer}),
  \item humidity (\emph{hygrometer}),
  \item precipitation (\emph{rain gauge}),
  \item solar radiation (\emph{pyranometer)},
  \item temperature (\emph{thermometer}), and
  \item wind direction and wind speed (\emph{wind vane} and \emph{anemometer}, respectively).
\end{itemize}

There exist \emph{weather stations} that provide sensors for combinations of the above weather elements.

Other sensors that are available under the \emph{KNX} standard include sensors for water, fluid level, smoke, \ce{CO2} concentration, or air pollution. However, these are not relevant in the context of \smarthomeweather as they do not provide weather data.


\section{Service data}
\label{sec:weather_services}

This section presents the details of a number of popular weather services that are available over the Internet. In a first step, a number of aspects are identified that are relevant for the usage of a weather service in the context of smart homes. The weather services are then evaluated regarding these aspects.

\subsection{Available Internet services}
\label{sec:internet_services}

There is a tremendous amount of services providing weather data over the Internet\footnote{No website has been found that compiles a comprehensive list of Internet weather services; however, there are many websites that discuss various services~\cite{WeatherServices1,WeatherServices2,WeatherServices3}}. However, only a small number of these services is suitable for usage in the context of \smarthomeweather. The services differ regarding the way data is provided, the data format, the area being covered, the terms of use etc.

The access to Internet weather services in \smarthomeweather is implemented in Java, hence an important question about a certain weather service is whether in can easily be accessed from within a Java program.

For evaluation of Internet weather services, the following aspects are examined:

\begin{itemize}
  \item \textbf{Coverage area:} Which part of the world is covered by the weather service?\\
  In order to keep things simple, a service covering a larger part of the world is preferred over a service covering a smaller part.
  
  \item \textbf{Data format:} In what format is weather data being delivered? Can it be easily parsed and processed?\\
  A data format that is easier to handle on client-side is preferred over a data format that requires more complicated handling. \eacs{XML}~\cite{XML} can be handled natively within many programming languages including Java, using \eacs{DOM} (\emph{Document Object Model})~\cite{DOM} or \eacs{SAX} (\emph{Simple \acs{API} for \acs{XML}})~\cite{SAX}. \eacs{JSON} (\emph{JavaScript Object Notation})~\cite{rfc4627} may require an additional library in some languages such as Java\footnote{\eacs{JSON} libraries for Java include \emph{JSON-lib}~\cite{json-lib}, \emph{FlexJSON}~\cite{flexjson}, and \emph{Gson}~\cite{gson}}, while \emph{PHP} comes with built-in support for \eacs{JSON}~\cite{php-json}. There may even be data formats that require the development of a new parser.
  
  \item \textbf{Data access:} How does the access to weather data work? How is data requested and how is the answer being received?\\
  The less complicated a request is, the more suitable a weather service is for \smarthomeweather. Requests in \eacs{HTTP} (\emph{Hypertext Transfer Protocol})~\cite{rfc2616} are preferred due to the simplicity of performing requests on the client side in Java using the \emph{Apache HttpComponents project}~\cite{apache_hc}.
  
  \item \textbf{Access restrictions, terms of use:} Is the service available freely or is the access restricted? Are credentials (e.g.\ a username and a password or an access key) required for access? If yes, can the credentials be obtained in a simple way or does that entail a complicated procedure? Are there any access fees for academic or commercial use?\\
  A service being less restricted is preferred over a service coming with more restrictions.
  
  \item \textbf{Documentation:} Is there any documentation for this service? Does it cover all aspects or are there some features that are undocumented?\\
  Of course, a service without documentation is unsuitable. A better documented service is always preferred over a service that is less documented.
  
  \item \textbf{Stability:} Can one expect the service to remain unchanged over a reasonable amount of time (e.g.\ several years)? If there will be future changes, will they be announced? How long will they be announced in advance?\\
  A stable service is preferred over an unstable one. A better handling of changes is preferred over a worse handling of changes.
  
  \item \textbf{Weather elements:} Which weather elements (e.g.\ temperature, relative humidity, dew point etc.) are covered by the service?\\
  A weather service covering a wider range of weather elements is preferred over a service covering a smaller range.
  
  \item \textbf{Time frame:} Are forecasts available? If yes, how detailed are these forecasts? How far into the future are forecasts available? What is the interval between two forecasts?\\
  Forecasts for at least 48 hours are mandatory. Over the next 48 hours, there should be at least six forecasts with an interval of at most eight hours between two consecutive forecasts. A weather service covering a longer period than 48 hours into the future and/or more than six forecasts within the next 48 hours is preferred.
  
  Although the competency questions for \smarthomeweather only ask for weather data over the upcoming 24 hours, a weather service used for \smarthomeweather is expected to provide forcasts for at least 48 hours into the future. This allows or improves interpolation of missing values (see Section~\ref{subsec:importer_fetch} for details).
  
  \item \textbf{Weather updates:} How often is the weather data being updated?\\
  The data should be updated at least every six hours. A service having a shorter update interval is preferred over a service having a longer interval.
\end{itemize}

In the following sections, some of the most popular Internet weather services are evaluated. The selection is a set of services that seemed to be popular at the time the evaluation was performed.

In addition to the aspects listed above, some general information is provided (operator and web page). The following sections aim at determining which Internet weather service best fits the given requirements. Furthermore, based on which weather elements are provided by various services, a set of weather elements is determined that will be used in the ontology. In case data about a weather element is not available from any weather service, a sensor can at least provide current data about this element.

\paragraph{Weather services being evaluated}

Table~\ref{table:weather_data1} lists all weather services that have been evaluated together with their operators, web pages, and coverage areas. \emph{Google Weather API} is discontinued~\cite{google_weather}; it is included here because \emph{Google} announced its shutdown after the time this evaluation was conducted. The \emph{Google Weather \acs{API}} serves as an example of a weather service that is unsuitable for usage in \smarthomeweather as it is discontinued. \eacs{DWD} (\emph{Deutscher Wetterdienst})~\cite{dwd} and \emph{NWS}~\cite{nws} are the only services that do not provide worldwide forecast data.

Table~\ref{table:weather_data3} lists the  weather services together with the data format they use to provide their data. For the case of Java, \eacs{XML}-based formats including \eacs{RSS} feeds (\emph{Rich Site Summary})~\cite{RSS} are preferred because of the built-in support for \eacs{XML} by \eacs{DOM} and \eacs{SAX}. There are libraries for providing \eacs{JSON} and \eacs{JSONP} (\emph{\acs{JSON} with padding})~\cite{jsonp} support for Java. \eacs{CSV} (\emph{comma-separated values})~\cite{rfc4180} is a format that can be parsed easily in any language, although using \eacs{CSV} leads to more implementation costs than \eacs{XML}. There are no libraries available that provide Java support for the \eacs{SYNOP} format (\emph{Surface Synoptic Observations}~\cite{SYNOP} which is used by \eacs{DWD}; hence this format is unsuitable to be used in a Java application.

\eacs{METAR} (\emph{Meteorological Aerodrome Report})~\cite{metar} is a format for reporting weather data~\cite{metar} that is standardised by the \emph{International Civil Aviation Organisation} (\eacs{ICAO})~\cite{ICAO}. It is primarily used in aviation to provide weather data to pilots of aircraft. There are a few parsers available, e.g.\ \emph{PyMETAR}~\cite{pymetar} for Python or an implementation for Java~\cite{metar-java}.

Most services provide their data via \eacs{HTTP}~\cite{rfc2616}, hence simple access from Java applications is possible using existing libraries such as the \emph{Apache HttpComponents project}~\cite{apache_hc}. \eacs{DWD} uses access via \eacs{FTP} (\emph{File Transfer Protocol})~\cite{rfc959} while \eacs{NWS} (\emph{National Weather Service}) employs \eacs{REST} (\emph{Representational State Transfer})~\cite{REST} and \eacs{SOAP} (\emph{Simple Object Access Protocol})~\cite{SOAP} for the access to its data. \eacs{FTP}, \eacs{REST}, and \eacs{SOAP} can all be used in Java applications, but their use may lead to higher implementation costs compared to the use of \eacs{HTTP}.

Regarding the terms of use that are summarised in Table~\ref{table:weather_data4}, there are some services that require the creation of an account (\eacs{DWD}, \emph{World Weather Online}, \emph{Weather Underground}, and \emph{Weather.com}) while others don't (\emph{Yahoo! Weather}, \emph{Google Weather Feed}, \yrno, and \eacs{METAR}). Some services provide all of their data freely (\eacs{DWD}, \yrno, \eacs{METAR}, or \eacs{NWS}), others provide the data freely only for non-commercial purposes (\emph{Yahoo! Weather}, \emph{World Weather Online}, and \emph{Weather Underground}); for some services, the terms of use are unknown (\emph{Google Weather Feed}, \emph{Weather.com}).

Except the ones not providing information about updates (\emph{Yahoo! Weather}, \emph{World Weather Online}, \emph{Weather Underground}, and \emph{Weather.com}) and the one being discontinued (\emph{Google Weather Feed}), all weather services declare to perform regular updates to their weather services (see Table~\ref{table:weather_data5}). Any changes are announced either on the web pages, via \eacs{RSS} feeds, or via email.

Table~\ref{table:weather_data6} shows which data is available from the weather services. While temperature is available from all services, some services provide only data about a few other weather elements (\emph{Google Weather Feed} and \emph{Weather Underground}); from some services, only limited forecasts are available (\eacs{DWD}, \emph{Yahoo! Weather}, and \emph{World Weather Online}, \emph{Weather.com}). Some services limit the availability of data via their freely accessible interface (\emph{World Weather Online} and \emph{Weather Underground}).

As seen in Table~\ref{table:weather_data6}, \eacs{METAR} only provides current weather data; some services provide forecasts for several days, but issue only one forecast per day (\emph{Yahoo! Weather}, \emph{Google Weather Feed}, and \emph{Weather Underground}). All weather services provide frequent updates (at least every few hours).

Nearly all weather services discussed provide comprehensive documentation regarding their interfaces and use; the only exception is the \emph{Google Weather Feed} which was an inofficial \eacs{API} and therefore has never had any official documentation.

\begin{table}
\centering
\begin{threeparttable}[b]
\begin{tabular}{|p{.26\textwidth}|p{.3\textwidth}|p{0.12\textwidth}|p{0.18\textwidth}|}
  \hline
  \textbf{Weather service} & \textbf{Operator} & \textbf{Web page} & \textbf{Coverage area} \\
  \hline\hline
  \textbf{DWD} & Deutscher Wetterdienst (\eacs{DWD}, ``German weather service'') & \cite{dwd} & Worldwide (current weather data); Germany and large cities around the world (forecasts)  \\
  \hline
  \textbf{Google Weather Feed} & Google Inc. & none\tnote{1} & Worldwide \\
  \hline
  \textbf{\acs{METAR}} & Airports around the world & \cite{metar_source} & Worldwide \\
  \hline
  \textbf{\acs{NWS}} & National Weather Service\tnote{2} & \cite{nws} & Only US \\
  \hline
  \textbf{Weather.com} & The Weather Channel, LLC & ~\cite{weather_com} & Worldwide \\
  \hline
  \textbf{Weather Underground} & Weather Underground & \cite{weather_underground} & Worldwide \\
  \hline
  \textbf{World Weather Online} & World Weather Online & \cite{worldweatheronline} & Worldwide  \\
  \hline
  \textbf{Yahoo! Weather} & Yahoo! Inc. & \cite{yahoo_weather} & Worldwide  \\
  \hline
  \textbf{yr.no} & Meteorologisk institutt (Norwegian Meteorological Institute), Norsk rikskringkasting AS (NRK, Norwegian Broadcasting Corporation) & ~\cite{yrno} & Worldwide \\
  \hline
\end{tabular}
\begin{tablenotes}
\item[1] not publicly advertised \eacs{API}
\item[2] part of \eacs{NOAA} (\emph{National Oceanic and Atmospheric Administration})~\cite{noaa}
\end{tablenotes}
\end{threeparttable}
\caption[Names, operators, web pages, and coverage areas of weather services]{Names, operators, web pages, and coverage areas of all weather services that have been evaluated.}
\label{table:weather_data1}
\end{table}

\begin{table}
\centering
\begin{threeparttable}[b]
\begin{tabular}{|l||l|l|l|l|l|l||l|l|l|l|l|}
  \hline
  ~ & \multicolumn{6}{c||}{\textbf{Data format}} & \multicolumn{5}{c|}{\textbf{Data access}} \\
  \hline
  ~ & \rotatebox{90}{\textbf{\eacs{CSV}}} & \rotatebox{90}{\textbf{\eacs{JSON}}} & \rotatebox{90}{\textbf{\eacs{JSONP}}} & \rotatebox{90}{\textbf{\eacs{RSS}}} & \rotatebox{90}{\textbf{\eacs{XML}}} & \rotatebox{90}{\textbf{Custom~}} & \rotatebox{90}{\textbf{\eacs{FTP}}} & \rotatebox{90}{\textbf{\eacs{HTTP}}} & \rotatebox{90}{\textbf{\eacs{REST}}} & \rotatebox{90}{\textbf{\eacs{SOAP}}} & \rotatebox{90}{\textbf{Custom}} \\
  \hline\hline
  \textbf{DWD} & $\times$ & $\times$ & $\times$ & $\times$ & $\times$ & \checkmark\tnote{1} & \checkmark & $\times$ & $\times$ & $\times$ & $\times$ \\
  \hline
  \textbf{Google Weather Feed} & $\times$ & $\times$ & $\times$ & $\times$ & \checkmark & $\times$ & $\times$ & \checkmark & $\times$ & $\times$ & $\times$ \\
  \hline
  \textbf{\acs{METAR}} & $\times$ & $\times$ & $\times$ & $\times$ & $\times$ & \checkmark\tnote{2} & $\times$ & $\times$ & $\times$ & $\times$ & \checkmark\tnote{3} \\
  \hline
  \textbf{\acs{NWS}} & $\times$ & $\times$ & $\times$ & $\times$ & \checkmark & $\times$ & $\times$ & $\times$ & \checkmark & \checkmark & $\times$ \\
  \hline
  \textbf{Weather.com} & $\times$ & \checkmark & $\times$ & $\times$ & \checkmark & $\times$ & $\times$ & \checkmark & $\times$ & $\times$ & $\times$ \\
  \hline
  \textbf{Weather Underground} & $\times$ & \checkmark & $\times$ & $\times$ & \checkmark & $\times$ & $\times$ & \checkmark & $\times$ & $\times$ & $\times$ \\
  \hline
  \textbf{World Weather Online} & \checkmark & \checkmark & \checkmark & $\times$ & \checkmark & $\times$ & $\times$ & \checkmark & $\times$ & $\times$ & $\times$ \\
  \hline
  \textbf{Yahoo! Weather} & $\times$ & $\times$ & $\times$ & \checkmark & $\times$ & $\times$ & $\times$ & \checkmark & $\times$ & $\times$ & $\times$ \\
  \hline
  \textbf{yr.no} & $\times$ & $\times$ & $\times$ & $\times$ & \checkmark & $\times$ & $\times$ & \checkmark & $\times$ & $\times$ & $\times$ \\
  \hline
\end{tabular}
\begin{tablenotes}
\item[1] \eacs{DWD} uses \eacs{SYNOP} (\emph{surface synoptic observations})~\cite{SYNOP} for current weather data, a data format frequently used for weather observations specified by the \eacs{WMO} (\emph{World Meteorological Organization})~\cite{WMO}; furthermore, \eacs{DWD} uses a data format which is not machine-readable consisting of weather maps, tables, and texts for forecasts.
\item[2] \eacs{METAR} uses its own format standardised by \eacs{ICAO} which is not human readable.
\item[3] \eacs{METAR}s are available in \eacs{HTML} format from \eacs{NOAA}'s web page\tnote{2} (\emph{National Oceanic and Atmospheric Administration}). Various other data sources are available.
\end{tablenotes}
\end{threeparttable}
\caption[Data formats and data access protocols of weather services]{Data formats and protocols for data access of weather services.}
\label{table:weather_data3}
\end{table}
  
\begin{table}
\centering
\begin{tabular}{|p{.19\textwidth}|p{0.71\textwidth}|}
  \hline
  \textbf{Weather service} & \textbf{Access restrictions, terms of use} \\
  \hline\hline
  \textbf{DWD} & An account is mandatory for access. Account creation is a matter of minutes and new accounts do not need to be approved by \eacs{DWD}. The usage of \eacs{DWD} is free within smart home projects.\\
  \hline
  \textbf{Google Weather Feed} & No account required; terms of use unknown (does not have an explicitely stated licence due to being an inofficial interface) \\
  \hline
  \textbf{\acs{METAR}} & Freely available without restrictions \\
  \hline
  \textbf{\acs{NWS}} & Freely available without restrictions \\
  \hline
  \textbf{Weather.com} & An account must be created in a complicated process; licence unclear \\
  \hline
  \textbf{Weather Underground} & Account creation is mandatory; available for personal, non-commercial use (public \eacs{API}); custom weather services are available \\
  \hline
  \textbf{World Weather Online} & An account must be created for accessing both the Free \eacs{API} and the Premium \eacs{API}. The Free \eacs{API} is free of charge for personal and commercial use, credits must be given to World Weather Online. A Premium \eacs{API} is available at a charge.\\
  \hline
  \textbf{Yahoo! Weather} & Free of charge for individuals and non-profit organizations, attribution required. No statement about commercial use anywhere in the documentation.\\
  \hline
  \textbf{yr.no} & No account required; data licenced under \emph{CC-BY 3.0}~\cite{ccby30}\\
  \hline
\end{tabular}
\caption[Access restrictions and terms of use for weather services]{Access restrictions and terms of use for weather services.}
\label{table:weather_data4}
\end{table}

\begin{table}
\centering
\begin{threeparttable}[b]
\begin{tabular}{|p{.19\textwidth}|p{0.71\textwidth}|}
  \hline
  \textbf{Weather service} & \textbf{Stability} \\
  \hline\hline
  \textbf{DWD} & A few changes every weeks which are announced via email at least one week in advance; nothing about the data required by \smarthomeweather has been changed during the last twelve months.\\
  \hline
  \textbf{Google Weather Feed} & Discontinued~\cite{google_weather} \\
  \hline
  \textbf{\acs{METAR}} & Stable.\tnote{1} \\
  \hline
  \textbf{\acs{NWS}} & Small changes every few months with announcements via \eacs{RSS} that do not effect the core parts of the interface. \\
  \hline
  \textbf{Weather.com} & Unknown \\
  \hline
  \textbf{Weather Underground} & The has been a recent \eacs{API} change; the stability is unknown. \\
  \hline
  \textbf{World Weather Online} & Unknown \\
  \hline
  \textbf{Yahoo! Weather} & Unknown \\
  \hline
  \textbf{yr.no} & New releases of the \eacs{API} one or two times per year; the old \eacs{API} remains in operation a few months after the release of a new one. Announcements are made on the web page. A less frequently changing \eacs{API} with \emph{long term support} is available.\\
  \hline
\end{tabular}
\begin{tablenotes}
\item[1] \emph{METAR} is standardised by \eacs{ICAO}~\cite{ICAO}. It was introduced in 1968 and has been modified a number of times since. However, most elements remained the same since introduction and can be expected to remain unchanged in the long run.
\end{tablenotes}
\end{threeparttable}
\caption[Stability of weather services]{Stability of weather services.}
\label{table:weather_data5}
\end{table}

\begin{table}
\centering
\begin{threeparttable}[b]
\begin{tabular}{|l|l||c|c|c|c|c|c|c|c|c|}
  \hline
  ~ & ~ & \rotatebox{90}{\textbf{DWD}} & \rotatebox{90}{\textbf{Google Weather Feed}} & \rotatebox{90}{\textbf{\acs{METAR}}} & \rotatebox{90}{\textbf{\acs{NWS}}} & \rotatebox{90}{\textbf{Weather.com}} & \rotatebox{90}{\textbf{Weather Underground\tnote{1}}} & \rotatebox{90}{\textbf{World Weather Online\tnote{1}~~~}} & \rotatebox{90}{\textbf{Yahoo! Weather}} & \rotatebox{90}{\textbf{yr.no}} \\
  \hline\hline
  \multirow{10}{*}{\rotatebox{90}{\textbf{Current weather}}} & \textbf{Cloud coverage} & \checkmark & $\times$ & \checkmark & \checkmark & $\times$ & $\times$ & $\times$ & $\times$ & \checkmark \\
  \cline{2-11}
  ~ & \textbf{Condition} & \checkmark & \checkmark & $\times$ & $\times$ & \checkmark & \checkmark & $\times$ & $\times$ & \checkmark \\
  \cline{2-11}
  ~ & \textbf{Dew point} & $\times$ & $\times$ & \checkmark & \checkmark & \checkmark & $\times$ & \checkmark & $\times$ & $\times$ \\
  \cline{2-11}
  ~ & \textbf{Humidity} & \checkmark & \checkmark & $\times$ & $\times$ & \checkmark & $\times$ & \checkmark & \checkmark & \checkmark \\
  \cline{2-11}
  ~ & \textbf{Precipitation} & \checkmark & $\times$ & \checkmark & \checkmark & \checkmark & $\times$ & \checkmark & \checkmark & \checkmark \\
  \cline{2-11}
  ~ & \textbf{Pressure} & \checkmark & $\times$ & \checkmark & $\times$ & \checkmark & $\times$ & \checkmark & \checkmark & \checkmark \\
  \cline{2-11}
  ~ & \textbf{Sunrise, sunset} & $\times$ & $\times$ & $\times$ & $\times$ & \checkmark & $\times$ & $\times$ & \checkmark & \checkmark \\
  \cline{2-11}
  ~ & \textbf{Temperature} & \checkmark & \checkmark & \checkmark & \checkmark & \checkmark & \checkmark & \checkmark & \checkmark & \checkmark \\
  \cline{2-11}
  ~ & \textbf{Visibility} & $\times$ & $\times$ & \checkmark & $\times$ & \checkmark & $\times$ & $\times$ & \checkmark & $\times$ \\
  \cline{2-11}
  ~ & \textbf{Wind} & \checkmark & $\times$ & \checkmark & \checkmark & \checkmark & $\times$ & \checkmark & \checkmark & \checkmark \\
  \hline\hline
  \multirow{10}{*}{\rotatebox{90}{\textbf{Future weather}}} & \textbf{Cloud coverage} & $\times$ & $\times$ & $\times$ & \checkmark & $\times$ & $\times$ & $\times$ & $\times$ & \checkmark \\
  \cline{2-11}
  ~ & \textbf{Condition} & \checkmark & \checkmark & $\times$ & $\times$ & \checkmark & \checkmark & $\times$ & \checkmark & \checkmark \\
  \cline{2-11}
  ~ & \textbf{Dew point} & $\times$ & $\times$ & $\times$ & \checkmark & $\times$ & $\times$ & \checkmark & $\times$ & $\times$ \\
  \cline{2-11}
  ~ & \textbf{Humidity} & $\times$ & $\times$ & $\times$ & $\times$ & \checkmark & $\times$ & \checkmark & $\times$ & \checkmark \\
  \cline{2-11}
  ~ & \textbf{Precipitation} & $\times$ & $\times$ & $\times$ & \checkmark & \checkmark & $\times$ & \checkmark & $\times$ & \checkmark \\
  \cline{2-11}
  ~ & \textbf{Pressure} & $\times$ & $\times$ & $\times$ & $\times$ & $\times$ & $\times$ & \checkmark & $\times$ & \checkmark \\
  \cline{2-11}
  ~ & \textbf{Sunrise, sunset} & $\times$ & $\times$ & $\times$ & $\times$ & \checkmark & $\times$ & $\times$ & $\times$ & \checkmark \\
  \cline{2-11}
  ~ & \textbf{Temperature} & \checkmark & $\times$ & $\times$ & \checkmark & \checkmark & \checkmark & \checkmark & \checkmark & \checkmark \\
  \cline{2-11}
   ~ & \textbf{Visibility} & $\times$ & $\times$ & $\times$ & $\times$ & $\times$ & $\times$ & $\times$ & $\times$ & $\times$ \\
  \cline{2-11}
  ~ & \textbf{Wind} & $\times$ & $\times$ & $\times$ & \checkmark & \checkmark & $\times$ & \checkmark & $\times$ & \checkmark \\
  \hline\hline
  \multicolumn{2}{|l||}{\textbf{Forecast range (days)}} & 3 & 3 & 0 & $> 7$ & 5 & 5 & 5 & 2& 9 \\
  \hline
  \multicolumn{2}{|l||}{\textbf{Forecasts per day}} & 1-2 & 1 & - & 4 & 1 & 1 & 1 & 1 & 4 \\
  \hline
  \multicolumn{2}{|l||}{\textbf{Update interval (hours)}} & $< 3$ & $< 3$ & $0.5$ & 1 & 1 & $< 3$ & $< 3$ & 3-4 & $< 3$ \\
  \hline
\end{tabular}
\begin{tablenotes}
\item[1] Further data is available via non-public interfaces which are adapted on customer request.
\end{tablenotes}
\end{threeparttable}
\caption[Weather data provided by weather services]{Weather data provided by weather services.}
\label{table:weather_data6}
\end{table}

\subsection{Summary}

\begin{table}
\centering
\begin{tabular}{|p{.26\textwidth}|p{.3\textwidth}|p{.34\textwidth}|}
  \hline
  \textbf{Weather service} & \textbf{Advantage(s)} & \textbf{Disadvantage(s)} \\
  \hline \hline
  \textbf{DWD} & - & Data format (\eacs{SYNOP}); forecasts lack important weather elements \\
  \hline
  \textbf{Google Weather Feed} & Simple data format (\eacs{XML}) & Unofficial, undocumented \eacs{API}; unknown terms of use; only a small number of available weather elements; discontinued \\
  \hline
  \textbf{\acs{METAR}} & - & Data format (\eacs{METAR}); no forecasts \\
  \hline
  \textbf{\acs{NWS}} & Simple data format (\eacs{XML}) & No worldwide coverage \\
  \hline
  \textbf{Weather.com} & Data formats (\eacs{XML}, \eacs{JSON}) & Complicated account creation \\
  \hline
  \textbf{Weather Underground} & Data formats (\eacs{XML}, \eacs{JSON}) & Public \eacs{API} lacks important data \\
  \hline
  \textbf{World Weather Online} & Wide range of data formats & Free \eacs{API} lacks important data \\
  \hline
  \textbf{Yahoo! Weather} & Simple data format (\eacs{RSS}) & Low number of forecasts; forecasts lack important weather elements \\
  \hline
  \textbf{yr.no} & Simple data format (\eacs{XML}); available under \emph{CC-BY 3.0}~\cite{ccby30} & - \\
  \hline
\end{tabular}
\caption[Advantages and disadvantages of Internet weather services]{Advantages and disadvantages of Internet weather services.}
\label{table:weather_services}
\end{table}

Table~\ref{table:weather_services} summarises the main advantages and disadvantages of the weather services discussed in Section~\ref{sec:weather_services}. Based on that evaluation, it is determined that \yrno clearly suits the requirements of smart homes and \smarthomeweather best. It provides current data and four forecasts per day over a period of nine days. Data is accessible in \eacs{XML} format via \eacs{HTTP} and is licensed under \emph{CC-BY 3.0}~\cite{ccby30}. Data includes details about temperature, wind direction and speed, chance and intensity of precipitation, atmospheric pressure, relative humidity, cloud coverage, and fog.

Thus, \yrno is used in Chapter~\ref{ch:weather_importer} for developing the reference implementation of a program that obtains weather data for a certain location and feeds it into the \smarthomeweather ontology. See Section~\ref{sec:weather_data_yr_no} for details on how to query \yrno's weather \eacs{API}.

Some weather services provide data about the position of the sun, e.g.\ the times of sunrise and sunset. However, \smarthomeweather does not rely on this data and chooses a different approach to determine the position of the sun (see Section~\ref{sec:sun_position}).

\section{Weather data \eacs{API} of \yrno}
\label{sec:weather_data_yr_no}

This section gives an overview on how requests to the weather \eacs{API} of \yrno work.

For an arbitrary request, latitude and longitude must be specified (in degrees; northern latitudes and eastern longitudes are represented by positive values)~\cite{yrno_documentation}. Additionally, the altitude above sea level (in metres) may be specified for locations outside of Norway. Based on this input data, a \emph{URL} of the format

\vspace{3mm}\noindent\begin{tikzpicture}\node[draw,rounded corners=0pt,inner sep=0pt,shade,top color=white,bottom color=listingBottom]{\begin{minipage}{\textwidth}\hspace{2mm}\begin{minipage}{.95\textwidth}\vspace{3mm}
\begin{minted}{text}
http://api.yr.no/weatherapi/locationforecast/1.8/?lat=<latitude>;lon=<longitude>
\end{minted}
\vspace{3mm}\end{minipage}\end{minipage}};\end{tikzpicture}

or

\vspace{1mm}\noindent\begin{tikzpicture}\node[draw,rounded corners=0pt,inner sep=0pt,shade,top color=white,bottom color=listingBottom]{\begin{minipage}{\textwidth}\hspace{2mm}\begin{minipage}{.95\textwidth}\vspace{3mm}
\begin{minted}{text}
http://api.yr.no/weatherapi/locationforecast/1.8/?lat=<latitude>;lon=<longitude>;msl=<altitude>
\end{minted}
\vspace{3mm}\end{minipage}\end{minipage}};\end{tikzpicture}

is constructed, e.g.

\vspace{1mm}\noindent\begin{tikzpicture}\node[draw,rounded corners=0pt,inner sep=0pt,shade,top color=white,bottom color=listingBottom]{\begin{minipage}{\textwidth}\hspace{2mm}\begin{minipage}{.95\textwidth}\vspace{3mm}
\begin{minted}{text}
http://api.yr.no/weatherapi/locationforecast/1.8/?lat=48.21;lon=16.37;msl=171
\end{minted}
\vspace{3mm}\end{minipage}\end{minipage}};\end{tikzpicture}

for the city of Vienna, Austria (N~\SI {48.21}{\degree }, E~\SI {16.37}{\degree }, \SI {171}{\metre } above \eacs {MSL}). An \eacs{HTTP} \emph{GET} request to this \eacs{URL} returns an \eacs{XML} document conforming to the \eacs{XML} \emph{Schema}~\cite{xml-schema} definition that can be found online~\cite{yrno_schema}.

The structure of this \eacs{XML} document is shown in Listing~\ref{listing:yrno_structure} (attributes are omitted for better readability).

\begin{mintlisting}
\begin{tikzpicture}\node[draw,rounded corners=0pt,inner sep=0pt,shade,top color=white,bottom color=listingBottom]{\begin{minipage}{\textwidth}\hspace{2mm}\begin{minipage}{.95\textwidth}\vspace{3mm}
\begin{minted}{xml}
<weatherdata>
	<meta>
		<model />
	</meta>
	<product>
		/* ... */
	</product>
</weatherdata>
\end{minted}
\vspace{3mm}\end{minipage}\end{minipage}};\end{tikzpicture}

\caption[Structure of the \eacs{XML} document returned by the \eacs{API} of \yrno]{Structure of the \eacs{XML} document returned by the \eacs{API} of \yrno; attributes are omitted for better readability.}
\label{listing:yrno_structure}
\end{mintlisting}

The attributes of the \texttt{<model>} element describe when the forecast has been created, when it will be updated for the next time, and what date and time of the first and the last forecast returned are.

There is an arbitrary number of \texttt{<time>} elements that are children of the \texttt{<product>} element. Every \texttt{<time>} element represents the weather forecast for a certain period of time. Each \texttt{<time>} element has a \texttt{<location>} element that has a child element for each weather property. A typical \texttt{<time>} element is depicted in Listing~\ref{listing:yrno_time_element}.

\begin{mintlisting}
\begin{tikzpicture}\node[draw,rounded corners=0pt,inner sep=0pt,shade,top color=white,bottom color=listingBottom]{\begin{minipage}{\textwidth}\hspace{2mm}\begin{minipage}{.95\textwidth}\vspace{3mm}
\begin{minted}{xml}
<time datatype="forecast" from="2013-06-24T12:00:00Z" to="2013-06-24T12:00:00Z">
	<location altitude="171" latitude="48.2100" longitude="16.3700">
		<temperature id="TTT" unit="celcius" value="15.7"/>
		<windDirection id="dd" deg="303.4" name="NW"/>
		<windSpeed id="ff" mps="6.7" beaufort="4" name="Laber bris"/>
		<humidity value="67.5" unit="percent"/>
		<pressure id="pr" unit="hPa" value="1016.0"/>
		<cloudiness id="NN" percent="100.0"/>
		<fog id="FOG" percent="0.0"/>
		<lowClouds id="LOW" percent="100.0"/>
		<mediumClouds id="MEDIUM" percent="89.1"/>
		<highClouds id="HIGH" percent="43.0"/>
	</location>
</time>
\end{minted}
\vspace{3mm}\end{minipage}\end{minipage}};\end{tikzpicture}

\caption[A \texttt{<time>} element returned by the \eacs{API} of \yrno (1)]{A typical \texttt{<time>} element returned by the \eacs{API} of \yrno.}
\label{listing:yrno_time_element}
\end{mintlisting}

In total, there are $41$ elements that are allowed to be children of the \texttt{<location>} element. While all of the are used by \yrno for locations within Norway and for some other places within Scandinavia, \yrno only uses a subset of these elements for other locations around the globe; these are texttt{pressure}, \texttt{precipitation}, \texttt{cloudiness}, \texttt{lowClouds}, \texttt{mediumClouds}, \texttt{highClouds}, \texttt{temperature}, \texttt{dewpointTemperature}, \texttt{humidity}, \texttt{windDirection}, \texttt{wind\hspace{0pt}Speed}, and \texttt{symbol}. These are exactly the elements which are relevant for \smarthomeweather.

None of the child elements is required. However, for most places of the world, the \eacs{XML} document contains \texttt{<location>} elements having two different sets of child elements:

\begin{itemize}
  \item Some \texttt{<location>} elements have the child elements \texttt{temperature}, \texttt{windDirection}, \texttt{windSpeed}, \texttt{humidity}, \texttt{pressure}, \texttt{cloudiness}, \texttt{fog}, \texttt{lowClouds}, \texttt{mediumClouds}, and \texttt{highClouds} (as in Listing~\ref{listing:yrno_time_element}). The values of the attributes from and to of the enclosing \texttt{<time>} element are equal. \texttt{<time>} elements that contain such a \texttt{<location>} element describe the weather situation for a single point of time.
  
  \item Some \texttt{<location>} elements have the child elements \texttt{precipitation} and \texttt{symbol}; the values of the attributes \texttt{from} and \texttt{to} of the enclosing \texttt{<time>} element differ by three to six hours. These \texttt{<time>} elements describe the weather situation for a period of time. See Listing~\ref{listing:yrno_time_element2} for an example of such an element.
\end{itemize}

\begin{mintlisting}
\begin{tikzpicture}\node[draw,rounded corners=0pt,inner sep=0pt,shade,top color=white,bottom color=listingBottom]{\begin{minipage}{\textwidth}\hspace{2mm}\begin{minipage}{.95\textwidth}\vspace{3mm}
\begin{minted}{xml}
<time datatype="forecast" from="2013-06-24T06:00:00Z" to="2013-06-24T12:00:00Z">
	<location altitude="171" latitude="48.2100" longitude="16.3700">
		<precipitation unit="mm" value="2.5"/>
		<symbol id="LIGHTRAIN" number="9"/>
	</location>
</time>
\end{minted}
\vspace{3mm}\end{minipage}\end{minipage}};\end{tikzpicture}

\caption[A \texttt{<time>} element returned by the \eacs{API} of \yrno (2)]{A \texttt{<time>} element returned by the \eacs{API} of \yrno describing a period of time.}
\label{listing:yrno_time_element2}
\end{mintlisting}

The documentation of \yrno does not explain why this distinction was made; in \smarthomeweather, in both cases the data from \yrno are transformed to descriptions of weather situations for periods of time (cf. Section~\ref{subsec:importer_fetch}.

The content of the \eacs{XML} document covers a period of nine days, starting at the current day.

\section{Position of the sun}
\label{sec:sun_position}

Besides location-based weather data, some weather services including \yrno offer an interface for retrieving sunrise and sunset data~\cite{yrno_sunrise}. Given a position in latitude and longitude together with a date, the times of rise and set of sun and moon are provided. The elevation angle of the sun at solar noon is also given.

In \smarthomeweather, the position of the sun is specified by azimuth angle and direction. For data about the sun's position to be of any value, it is necessary that the position is known for every \emph{Weather state}. Although it may be possible to calculate the sun's position from sunrise and sunset times and the elevation sun's angle at noon, this leads to unnecessary development effort.

Furthermore, most of the weather services that are available via Internet do not offer any data about the sun's position. Hence, it was concluded that the data provided by weather services describing the position of the sun are inappropriate for the use within the \smarthomeweather ontology. Instead, the sun's position is provided by one of some well-known algorithms.

There are several algorithms available for calculating the sun's position (specified by zenith and azimuth angles) at a certain location given by latitude and longitude at a certain time; two of them are:

\begin{itemize}
  \item The \emph{Solar Position Algorithm} (\eacs{SPA})~\cite{SPA_algorithm} provides results for zenith and azimuth angles with uncertainties of less than \num{0.0003} degrees in the period from 2000~BC to 6000~AD. However, the algorithm which is complicated to implement. Hence, it is used only in contexts that require values of that precision.
  
  \item An algorithm that can be implemented more easily is the \emph{\acs{PSA} algorithm}~\cite{PSA_algorithm}, named after \emph{Plataforma Solar de Almería}~\cite{PSA} (a centre for the exploration of the use of solar energy situated in the Province of Almería in Spain) where the algorithm was developed. The results provided by this algorithm differ from the actual values more than the results calculated by the \eacs{SPA}: For the period between 1999 and 2015, the differences per value are smaller than \num{0.01} degrees. As this level of accuracy is sufficient for use in a smart home, the \emph{\acs{PSA} algorithm} is suitable for use by the \emph{Weather importer} application (see Section~\ref{subsec:importer_fetch}). A ready-to-use implementation of the \emph{\acs{PSA} algorithm} in C++ is available; this implementation can easily be ported to Java.
\end{itemize}

\section{Conclusion}
\label{sec:weather_conclusion}

The findings in this chapter provide the basis for the design of \smarthomeweather in Chapter~\ref{ch:smarthomeweather_ontology} and the development of a reference implementation for the import of weather data from an Internet weather service into \smarthomeweather in Chapter~\ref{ch:weather_importer}: Section~\ref{sec:weather_information} presents the competency questions the ontology shall answer. Section~\ref{sec:weather_sensors} and Section~\ref{sec:weather_services} discuss which data is available from local weather sensors and Internet weather services. In Section~\ref{sec:weather_services} \yrno is determined to be the weather service which suits the requirements of smart homes best; Section~\ref{sec:weather_data_yr_no} describes its \eacs{API} in detail. Eventually, Section~\ref{sec:sun_position} presents the \eacs{PSA} algorithm which is used to provide \smarthomeweather with data about the position of the sun.

Although \yrno is selected to develop a reference implementation for the import of weather data, the source of weather data shall remain replaceable. The ontology is designed in a way that makes switching to another weather service simple. If \yrno is to be replaced by another weather service, the ontology remains unchanged; only the program that imports weather data must be adapted. If the weather service used does not support one or more weather elements specified in the ontology, they are simply omitted in the program's output.

The ontology supports the following weather elements (in no particular order):
\begin{itemize}
  \item Temperature,
  \item relative humidity,
  \item dew point,
  \item cloud cover,
  \item chance and intensity of precipitation,
  \item speed and direction of wind,
  \item atmospheric pressure,
  \item solar radiation,
  \item the position of the sun, and
  \item the overall weather condition.
\end{itemize}

Except the position of the sun, all elements are available from many Internet weather services and weather sensors. This data is not obtained from any sensor or service; instead, it is calculated using the \eacs{PSA} algorithm (see Section~\ref{sec:sun_position}.

In case a weather service does not provide a value for the dew point, there are simple means available for calculating that value from the values of temperature and relative humidity. E.g. the following formula leads to a dew point value with an accuracy sufficient for smart homes~\cite{dew_point_calculation}:

\[t_d \approx t-\left(\frac{100-RH}{5}\right)\]

$t_d$ represents the dew point temperature, $t$ the air temperature, and $RH$ the relative humidity in percent (i.e.\ in the interval $[0,100]$). Accordingly, the value of relative humidity can be calculated from the values of dew point and temperature using the same method (the case of a missing temperature value, but values of dew point and relative humidity is unlikely).

As the atmospheric pressure decreases with increasing altitude above sea level, it is necessary to convert the pressure value observed by a sensor to the equivalent pressure at sea level using the \emph{Barometric formula}~\cite{us76_standard_atmosphere}.
This is not necessary for many Internet weather services (including \yrno) as these often report a value that has already been converted.

To ensure the ontology can be used in the desired manner, it is necessary that the competency questions from Section~\ref{sec:weather_information} can be answered by the ontology using the provided input data. As seen in Table~\ref{table:competency_questions} which shows which competency questions can be answered by using which weather element(s), all required data is available. Once the development of \smarthomeweather is completed, Section~\ref{sec:ontology_evaluation} will discuss whether the ontology can actually answer the competency questions.

\begin{table}
\centering
\begin{tabular}{|p{.45\textwidth}|p{.45\textwidth}|}
\hline
\textbf{Competency question} & \textbf{Weather element(s)} \\
\hline \hline
What is the current weather situation? & \emph{all available elements} \\
\hline
What will the weather situation be in one hour, in two hours, …, in 24 hours? & \emph{all available elements} \\
\hline
What is the current temperature, humidity, wind speed, …? & \emph{the corresponding weather element} \\
\hline
What will be the temperature, humidity, wind speed, … in one hour, in two hours, …, in 24 hours? & \emph{the corresponding weather element} \\
\hline
What will be the minimum temperature, humidity, … over the next 24 hours? What about maximum values? & \emph{the corresponding weather element} \\
\hline
Will the weather change? Will the temperature, humidity, … rise or fall? & \emph{the corresponding weather element} \\
\hline
Does it rain? Will it rain in the next hours? Will it rain today? & Precipitation \\
\hline
Will there be sunshine today? & Cloud coverage, solar radiation, sun position \\
\hline
Do we need to irrigate the garden? & Precipitation, cloud coverage, solar radiation \\
\hline
Will there be severe weather? & Wind, precipitation, temperature \\
\hline
Will temperature drop/stay below \SI{0}{\celsius}? & Temperature \\
\hline
When can we open windows and when do we have to keep them shut? & Precipitation, wind \\
\hline
When do we need sun protection? & Solar radiation, cloud coverage, sun position \\
\hline
When will it outside be colder than inside the house? When will it be warmer? & Temperature \\
\hline
\end{tabular}
\caption[Assignment of competency questions to weather element(s)]{Assignment of competency questions to weather element(s).}
\label{table:competency_questions}
\end{table}

