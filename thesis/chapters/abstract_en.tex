\chapter*{Abstract}

In the past few years, the idea of creating smart homes has gained popularity.
A smart home possesses some kind of intelligence that allows it to take work from the its inhabitants.
The overall goal is to increase the inhabitants' comfort, but energy use and costs will be reduced as well.

The aim of the \textit{ThinkHome} project \cite{CR2011-TH_Journal} \cite{CR2010-DEST_ThinkHome} is to design smart homes that are able to decrease energy consumption.
For making control decisions, \textit{ThinkHome} is capable of processing input data from various sources.
In the context of the \textit{ThinkHome} project, this thesis aims at constructing an OWL ontology for weather information, containing data about both current conditions and weather forecasts.
The data described by this ontology will enable the \textit{ThinkHome} system to make decisions based on current and future weather conditions.

At first, the thesis will determine in which particular ways weather data can be used within \textit{ThinkHome}. Furthermore, possible sources for weather data will be analyzed. Primary sources will be weather services that are that are accessible via Internet. Optionally, local weather stations can provide further data about current weather conditions. A set of Internet-based sources for weather information will be reviewed for their suitability for use within the \textit{ThinkHome} project.

Afterwards, existing ontologies will be reviewed for their structure, advantages and disadvantages in order to acquire ideas being suitable to be re-used. Several well-known approaches for building new ontologies from scratch will be discussed in detail.

The thesis will follow Methontology \cite{Methontology}, the best-fitting of these approaches, to build an \textit{OWL} ontology that covers both the weather data being available and the concepts required to perform weather-related tasks within \textit{ThinkHome} while always keeping the possibility of simple and efficient \textit{OWL reasoning} in mind.

Based on that ontology, a weather module for the \textit{ThinkHome} project will be developed. That module will gather data from weather services and local weather stations and transform it into a representation that complies to the \textit{ThinkHome} \textit{OWL} ontology. Furthermore, an interface between the weather module and \textit{ThinkHome}'s multi-agent system will be created. That interface will enable the multi-agent system to use weather information in order to make appropriate control decisions.
