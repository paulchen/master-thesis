\chapter{Introduction}
\label{ch:intro}

\section{Motivation}

Over the last years, the idea of designing and building smart homes gained increasing popularity. According to the \emph{Intertek Research \& Testing Centre}\footnote{\href{http://www.its-rtc.com/}{http://www.its-rtc.com/}} who carried out a smart home project named \emph{DTI Smart Home Project}, a smart home is ``a dwelling incorporating a communications network that connects the key electrical appliances and services, and allows them to be remotely controlled, monitored or accessed. Remotely in this context can mean both within the dwelling and from outside the dwelling''~\cite{SmartHomeDefinition,SmartHomeResearch}. By utilizing these components, a smart home can be equipped with some kind of intelligence which enables the home to take work from its inhabitants. The goals that become reachable by that approach are diverse: For instance, smart homes may be aimed at making their inhabitants' lives easier or at reducing the overall energy consumption without any loss of comfort. % TODO some more uses?

For instance, a smart home can take control of the \eacs{HVAC} (heating, ventilation and air conditioning) system of a building. Instead of the inhabitants who have to care about maintaining an appropriate room temperature and the supply of a sufficient amount of fresh air, the smart home manages these tasks. Moreover, a smart home can regulate indoor and outdoor illumination, control home entertainment systems or monitor the building's state for unusual activity. The list of possible applications of smart homes is endless.

In order to fulfill its purpose, a smart home requires three elements: Intelligent control, home automation products and an internal network. Intelligent control represents a gateway for managing the systems. Home automation products are components distributed in the building that either monitors or measures physical states (i.e. \emph{sensors}), performs certain actions (i.e. \emph{actors}) or both. Examples for sensors are thermometers, microphones, cameras or motion detectors; actors may be switches, dimmers, room temperature controllers or window blind actuators. All components are connected using an internal network that is used by sensors to report their measurements and by actors to receive their command inputs.

The areas served by smart home systems can be categorized into six categories: Environment (\eacs{HVAC}, water management, lighting, enery management, metering), security (alarms, motions detectors, environmental detectors), home entertainment (audio visual equipment, Internet), domestic appliances (cooking, cleaning, mainenance alerts), information and communication (phone, Internet) and health (telecare, home assistance).

As smart homes comprise a large field of research and development, there exist several projects that are aimed at developing infrastructures for smart homes. Some of these projects are:
\begin{itemize}
  \item \textbf{Project 1}: …
  \item \textbf{Project 2}: …
  \item \textbf{Project 3}: …
\end{itemize}

This thesis belongs to a smart home project named \thinkhome\footnote{\href{https://www.auto.tuwien.ac.at/projectsites/thinkhome/overview.html}{https://www.auto.tuwien.ac.at/projectsites/thinkhome/overview.html}} which is developed at the \emph{Institute of Computer Aided Automation} at the \emph{Vienna University of Technology}~\cite{CR2011-TH_Journal,CR2010-DEST_ThinkHome}.

% TODO what is thinkhome

ThinkHome can process input data from various sources. However, weather data is not yet supported. Hence, at the moment \thinkhome is unable to make control decisions based on current weather conditions such as temperature, humidity, precipitation or sunshine. Furthermore, \thinkhome has no knowledge about future weather conditions such as upcoming sudden changes of weather which may require specific control decisions in advance.

\section{Problem statement and goal}

The aim of this thesis is the development of a data model for weather data which will be utilized by \thinkhome. Apart from current weather data, this model will cover future weather data over a time range suitable for the use within \thinkhome. This enables \thinkhome to use current and future states of the weather as a source of knowledge for making control decisions. While data about the current weather state can be obtained from various sensors that are installed at the building, data about future weather states must be gathered from weather services that provide forecasts for the desired period. There is a wide range of weather services available over the Internet which can be utilized for this purpose.

Besides the data model itself, an application is to be developed which imports weather data from local weather sensors, Internet weather services or both. To provide a reference implementation, a certain weather service is used. The application is designed in a modular way to simplify the use of different weather services.

Both the data model and the weather data enable \thinkhome to make control decisions based on current and future weather conditions.

\section{Methodological Approach}

% TODO identify requirements?

% TODO Some conditions are defined that a weather service must adhere to for being suitable for use within the \textit{ThinkHome} project; e.g. the services must provide both current data and forecasts, the data retrieved must be machine-readable and the services' terms and conditions must allow the use for smart homes.

% TODO After that review process has been finished, a set of weather properties (e.g. temperature and humidity) is defined. Afterwards, that properties are structured into an \textit{OWL} ontology that is designed for making \textit{OWL} reasoning possible. One of the approaches described in \cite{Ontology101} and \cite{SoftwareEngineeringOntology} may be used.

There are several existing approaches for providing weather data to smart homes as well as several approaches for covering weather data using (\emph{OWL}) ontologies. These approaches analyzed in order to determine whether they are suitable for being reused for \thinkhome. However, as chapter~\ref{ch:existing_work} discusses, none of them suits \thinkhome's requirements. Thus, the approaches previously analyzed are reviewed for their structure, their advantages and their disadvantages.

Furthermore, a set of weather services that are available via Internet is reviewed, regarding the type of data they provide and whether they suit the requirements of a smart home. Based on these requirements and the data provided by weather sensors and Internet weather services, the data which can be provided to \thinkhome is identified, i.e. a set of weather properties (e.g. temperature or humidity) and the time range that will be covered. One of the services is selected that will later be used to develop a reference implementation for the import of weather data.

As none of the existing ontologies can be used for \thinkhome, a new ontology is designed. The development process follows one of a set of well-known approaches for ontology development. Therefore, a set of approaches is selected and evaluated. The best suitable approach is identified.

This approach is then applied to the problem domain of weather data. The result is an ontology that satisfies \thinkhome's requirements as far as possible. Furthermore, a reference implementation for the import of weather data from services that are available via Internet is developed. This implementation uses an object-oriented model that stores the weather data and is responsible for converting input data into an appropriate format for \thinkhome. A modular design is preferred in order to simplify the adoption of a different weather service when necessary.

Finally, all results are critically evaluated, possible future modifications, improvements and extensions are discussed.

\section{Outline}

Chapter~\ref{ch:existing_work} discusses existing work regarding integration of weather data into \smarthome systems, ontologies for weather data and ontologies that may be imported by a weather ontology for \thinkhome.

Chapter~\ref{ch:weather_data} examines various sources for weather data and determines the scope of weather data that is relevant to \thinkhome and that will be used in the ontology.

In chapter~\ref{ch:development_approaches} various well-known approaches for developing ontologies are presented. Their suitability for designing an ontology for \thinkhome is evaluated and the best-suited approach is identified.

Chapter~\ref{ch:thinkhomeweather_ontology} describes the process of developing the ontology.

In chapter~\ref{ch:weather} a Java application is presented that gathers weather data from appropriate services and provides them to the \thinkhome ontology.

Finally, chapter~\ref{ch:conclusion} concludes about the insights from the previous chapters, summarizes all findings and gives an outlook about future work.

Appendix~\ref{ch:listings} contains all tables and listings that are omitted from the previous chapters as they are only required for reference and completeness, but not for understanding the topics covered by the previous chapters.

Appendix~\ref{main} contains the Glossary of Terms.
