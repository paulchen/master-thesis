\chapter{Introduction}
\label{ch:intro}

\section{Motivation}
\label{sec:motivation}

Over the recent years, the idea of designing and building \emph{smart homes} has gained increasing popularity. A \emph{smart home} can be seen as a building being equipped with some kind of intelligence which enables the building to support its inhabitants. For instance, a smart home can take control of the \eacs{HVAC} (heating, ventilation and air conditioning) system of a building. Instead of the inhabitants having to care about maintaining an appropriate room temperature and the supply of a sufficient amount of fresh air, the smart home itself may manage these tasks automatically on its own. Moreover, a smart home may adjust indoor illumination, control home entertainment systems or monitor the building's state for unusual activity. The list of possible applications of smart homes is endless.

According to the \emph{Intertek Research \& Testing Centre}~\cite{intertek} which carried out a smart home project named \emph{DTI Smart Home Project}, a smart home is ``a dwelling incorporating a communications network that connects the key electrical appliances and services, and allows them to be remotely controlled, monitored or accessed. Remotely in this context can mean both within the dwelling and from outside the dwelling''~\cite{SmartHomeDefinition,SmartHomeResearch}. The overall objectives that can be targeted using that approach are diverse: For instance, smart homes may be aimed at increasing their inhabitants' comfort or at reducing the overall energy consumption without any loss of comfort.

The areas served by smart home systems can be categorized into six categories: Environment (\eacs{HVAC}, water management, lighting, energy management, metering), security (alarms, motions detectors, environmental detectors), home entertainment (audio visual equipment, Internet), domestic appliances (cooking, cleaning, maintenance alerts), information and communication (phone, Internet) and health (telecare, home assistance).

In order to fulfil its purpose, a smart home requires three elements: Intelligent control, home automation products, and an internal network. Intelligent control represents a gateway for managing the systems. Home automation products are components distributed in the building that either monitors or measures physical states (i.e. \emph{sensors}), performs certain actions (i.e. \emph{actors}) or both. Examples for sensors are thermometers, microphones, cameras or motion detectors; actors may be switches, dimmers, room temperature controllers or window blind actuators. All components are connected using an internal network that is used by sensors to report their measurements and by actors to receive their command inputs.

As smart homes comprise a large field of research and development, there exist several projects that are aimed at developing infrastructures for smart homes. Some of these projects are:
\begin{itemize}
  \item \textbf{Mozer's adaptive house}: The \emph{Adaptive House} is an approach that is based on continuous monitoring the building's inhabitants regarding what actions they take and when. Based on observations collected over a certain period of time, the \emph{Adaptive House} tries to predict the inhabitants' behaviour in order to perform monitored actions (e.g. controlling the building's illumination) automatically.~\cite{adaptivehouse}
  \item \textbf{Georgia Tech Aware Home}: The \emph{Georgia Tech Aware Home} is centered around a pre-defined set of use cases that are part of various scenarios such as busy families, ageing in place, or children with special needs. Throughout the building, monitors, input interfaces etc. are placed which provide assistance to the inhabitants. The \emph{Tech Aware Home} is a rather static approach that does not include the idea of learning from the inhabitants.~\cite{techawarehome}
  \item \textbf{Gator Tech Smart Home}: The \emph{Gator Tech Smart Home} is an approach geared towards simplified evolution of the home regarding the addition of new technologies and modifications to existing application domains. The goal is the creation of \emph{assisted environments} that utilises available sensors and actors to improve the building's inhabitant's comfort and to take routine work away from them~\cite{gator_tech}.
  \item \textbf{eHome}: \emph{eHome} is a project implemented for research purposes in a two-room apartment that is equipped with actors for tasks like turning light on and off, moving curtains etc. Three user interfaces (laptop, TV set and mobile phone) are available for performing the available tasks. \emph{eHome} is not a ``smart home'' in the strict sense, instead ``automated home'' is a better suitable term.~\cite{ehome}
  \item \textbf{House\_n}: \eacs{MIT}'s \emph{House\_n} is an approach based on what its designers call ``subtle reminders''. By the inhabitants' current behaviour, the system tries to predict what they are about to do. Based on these predictions and measurement data from various sensors, \emph{House\_n} provides support in terms of textual hints and images which are presented on simple displays mounted on windows, doors, cupboards, etc. \emph{House\_n} intentionally abstains from the use of any actors to not patronise its inhabitants.~\cite{housen}
\end{itemize}

One of the downsides that is common among many current smart home systems is the failure to exploit their full potential~\cite{HomeAutomationChallengesOpportunities,HomeMaestro,StateOfHomeNetworking}. The number of parameters such a system must take into account and the variety of processes being controlled often lead to systems with a high degree of complexity. The actions taken by the system become inexplicable while optimisations or customisations imply a huge effort and are therefore avoided. The system proves to be not as powerful or flexible as desired. Users are often disappointed by bad experiences with smart home systems and decide to refrain from their use.

The idea that lead to this thesis emerged from a smart home project named \thinkhome~\cite{ThinkHomeWeb} which is developed at the \emph{Institute of Computer Aided Automation} at the \emph{Vienna University of Technology}~\cite{CR2011-TH_Journal,CR2010-DEST_ThinkHome}. \thinkhome proposes an approach towards a smart home system that overcomes the aforementioned problems in order to provide sustainable buildings with minimized energy consumption. A building which incorporates \thinkhome is ought to make understandable control decisions which maintain energy efficiency and comfort without patronising its inhabitants.

The \thinkhome ecosystem consists of two main components: a comprehensive knowledge base and a multi-agent system. The knowledge base contains both static data (data that is seldom changed, e.g. the structure of the building, user preferences, control rules) as well as dynamic data (data that changes frequently, e.g. the current state of the building, the current state of external influences, recent measurements from sensors, current states of actors). It is implemented using a set of ontologies which define a common data model and provide a shared vocabulary. By exploiting the reasoning capabilities provided by these ontologies, the multi-agent system is enabled to make control decisions based on either predefined rules or self-learned experience.

The use of well-established semantic web technologies for its knowledge base and field-tested methods from AI, \thinkhome is able to provide a both flexible and powerful infrastructure which is suitable for dealing with the complexity of smart homes in an efficient and manageable way.

Like other smart home ecosystems, \thinkhome is designed to process input data from various sources. One source is weather data which enables \thinkhome's multi-agent system to make control decisions based on current weather conditions such as temperature, humidity, precipitation, or sunshine. Furthermore, \thinkhome needs knowledge about future weather conditions such as upcoming sudden changes of weather which may be required by specific control decisions in advance.

This \emph{Predictive Control} has been the subject of numerous articles in the recent years~\cite{predictive_control1,predictive_control2,predictive_control3}; however, none of these approaches builds upon a knowledge base that is built using an ontological model.

\section{Problem statement and goal}

The aim of this thesis is the development of a data model for weather data which will be utilized by smart home systems. Apart from current weather data, this model covers future weather data over a time range suitable for the use within a smart home. This enables smart homes to use current and future weather states as a source of knowledge for making control decisions. While data about the current weather state can be obtained from various sensors that are installed at the building, data about future weather states must be obtained from weather services that provide forecasts for the desired period. There is a wide range of weather services available over the Internet which can be utilized for this purpose.

Besides the data model itself, an application is developed which imports weather data from local weather sensors, Internet weather services, or both. To provide a reference implementation, one particular weather service is selected and utilised. The application is designed in a modular way to simplify the use of different weather services.

The data model and the weather data enable smart homes to make control decisions based on current and future weather conditions.

\section{Methodological approach}

There are several existing approaches for providing weather data to smart homes as well as several approaches for covering weather data using (\emph{OWL}) ontologies. These approaches are analysed in order to determine whether they are suitable for being reused in the context of smart homes. However, as chapter~\ref{ch:existing_work} discusses, none of them meets the appropriate requirements. Thus, the approaches previously analysed are reviewed for their structure, their advantages and their disadvantages.

Furthermore, a set of weather services that are available via Internet is reviewed, regarding the type of data they provide and whether they suit the requirements of a smart home, e.g. the services must provide both current data and forecasts, the data retrieved must be machine-readable and the services' terms and conditions must allow the usage in smart homes. Based on these findings and the data provided by weather sensors and Internet weather services, the data which can be provided to smart homes is identified, i.e. a set of weather properties (e.g. temperature or humidity) and the time range that will be covered. One of the services is selected that will later be used to develop a reference implementation for the import of weather data.

As none of the existing ontologies qualifies to be used in smart homes, a new \eacs{OWL} ontology is designed, always keeping its intended use and simple and efficient reasoning in mind. The development process of this ontology -- which is named \emph{SmartHomeWeather} -- follows one of a set of well-known approaches for ontology development. Therefore, a set of approaches is selected and evaluated. The best suitable approach is identified.

This approach is then applied to the problem domain of weather data. The result is an ontology that satisfies requirements found at smart homes as far as possible. Furthermore, a reference implementation for the import of weather data from local services services that are available via Internet is developed. This implementation uses an object-oriented model that stores the weather data and is responsible for converting input data into an appropriate format for \smarthomeweather. A modular design is preferred in order to simplify the adoption of a different weather service when necessary.

Finally, all results are critically evaluated and possible future modifications, improvements and extensions are discussed.

\section{Outline}

Apart from the introduction, this thesis is structured in the following manner:

Chapter~\ref{ch:existing_work} discusses existing work regarding integration of weather data into smart home systems, ontologies for weather data and ontologies that may be imported by \smarthomeweather. Furthermore, \thinkhome is presented as an example of a smart home infrastructure that comprises an knowledge base built using ontologies. \eacs{RDF}, \eacs{RDFS} and \eacs{OWL} are the technical foundations of this thesis and are covered by this chapter as well.

Chapter~\ref{ch:weather_data} examines various sources for weather data and determines the scope of weather data that is relevant to smart homes. Based on these findings, the range of weather data parameters that will be used in the ontology is settled.

In chapter~\ref{ch:development_approaches} various well-known approaches for developing ontologies are presented. Their suitability for designing the \smarthomeweather ontology is evaluated and the best-suited approach is identified. The latter is described in all details relevant in the context of this thesis.

Chapter~\ref{ch:smarthomeweather_ontology} describes the process of actually designing the \smarthomeweather ontology.

In chapter~\ref{ch:weather_importer} a Java application is presented that obtains weather data from appropriate services and sensors and provides them to the \smarthomeweather ontology.

Finally, chapter~\ref{ch:conclusion} concludes about the insights from the previous chapters, summarizes all findings and gives an outlook about possible future work.

Appendix~\ref{ch:listings} contains all tables and listings that are omitted from the previous chapters as they are only included for reference and completeness, but not for understanding the topics covered by the previous chapters.

Appendix~\ref{main} contains the glossary of terms.
