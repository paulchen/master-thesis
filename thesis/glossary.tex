\newglossaryentry{temperature}{
  name={temperature},
  description={TODO},
  see={weather phenomenon,room temperature,above room temperature,heat,below room temperature,frost,cold}
}
\newglossaryentry{room temperature}{
  name={room temperature},
  description={\emph{Temperature} of least 20 and at most 25 degrees Celsius},
  see={temperature}
}
\newglossaryentry{above room temperature}{
  name={abobe room temperature},
  description={\emph{Temperature} of more than 25 and less than or equal to 30 degrees Celsius},
  see={temperature}
}
\newglossaryentry{below room temperature}{
  name={below room temperature},
  description={\emph{Temperature} of more than or equal to 10 and less than 20 degrees Celsius},
  see={temperature}
}
\newglossaryentry{cold}{
  name={cold},
  description={\emph{Temperature} of more than or equal to 0 and less than 10 degrees Celsius},
  see={temperature}
}
\newglossaryentry{frost}{
  name={frost},
  description={\emph{Temperature} below 0 degrees Celsius},
  see={temperature}
}
\newglossaryentry{heat}{
  name={heat},
  description={\emph{Temperature} above 30 degrees Celsius},
  see={temperature}
}


% TODO references
\newglossaryentry{sun position}{
  name={sun position},
  description={The position of the sun, given by its elevation above horizon and its direction},
  see={day,night,twilight,solar twilight,nautical twilight,astronomical twilight,civil twilight,sun from east,sun from south,sun from west,sun from north,twilight,weather phenomenon}
}
\newglossaryentry{day}{
  name={day},
  description={During the day, the sun is exactly at or above the horizon},
  see={sun position}
}
\newglossaryentry{night}{
  name={night},
  description={The sun is more than 18 degrees below horizon},
  see={sun position}
}
\newglossaryentry{solar twilight}{
  name={solar twilight},
  description={During solar twilight, the sun is above horizon, but less than 6 degrees above},
  see={sun position,day}
}
\newglossaryentry{twilight}{
  name={twilight},
  description={During twilight, the sun is not at or above horizon and at most 18 degrees below horizon},
  see={sun position,astronomical twilight,civil twilight,nautical twilight}
}
\newglossaryentry{civil twilight}{
  name={civil twilight},
  description={During civil twilight, the sun is not at or above horizon and at most 6 degrees below horizon},
  see={sun position,twilight}
}
\newglossaryentry{nautical twilight}{
  name={nautical twilight},
  description={During nautical twilight, the sun is more than 6 degrees and no more than 12 degrees below horizon},
  see={sun position,twilight}
}
\newglossaryentry{astronomical twilight}{
  name={astronomical twilight},
  description={During astronomical twilight, the sun is more than 12 degrees and no more than 18 degrees below horizon},
  see={sun position,twilight}
}
\newglossaryentry{sun from north}{
  name={sun from north},
  description={The direction of the sun's position of more than 315 or at most 45 degrees},
  see={sun position}
}
\newglossaryentry{sun from east}{
  name={sun from east},
  description={The direction of the sun's position of more than 45 and at most 135 degrees},
  see={sun position}
}
\newglossaryentry{sun from south}{
  name={sun from south},
  description={The direction of the sun's position of more than 135 and at most 225 degrees},
  see={sun position}
}
\newglossaryentry{sun from west}{
  name={sun from west},
  description={The direction of the sun's position of more than 225 and at most 315 degrees},
  see={sun position}
}


% TODO reference
\newglossaryentry{atmospheric pressure}{
  name={atmospheric pressure},
  description={TODO},
  see={weather phenomenon,very low pressure,low pressure,average pressure,high pressure,very high pressure}
}
\newglossaryentry{very low pressure}{
  name={very low pressure},
  description={\emph{Atmospheric pressure} of less than 998\hecto\pascal},
  see={atmospheric pressure}
}
\newglossaryentry{low pressure}{
  name={low pressure},
  description={\emph{Atmospheric pressure} of at least 998 and less than 1008\hecto\pascal},
  see={atmospheric pressure}
}
\newglossaryentry{average pressure}{
  name={average pressure},
  description={\emph{Atmospheric pressure} of at least 1008 and less than 1018\hecto\pascal},
  see={atmospheric pressure}
}
\newglossaryentry{high pressure}{
  name={high pressure},
  description={\emph{Atmospheric pressure} of at least 1018 and less than 1028\hecto\pascal},
  see={atmospheric pressure}
}
\newglossaryentry{very high pressure}{
  name={very high pressure},
  description={\emph{Atmospheric pressure} of at least 1028\hecto\pascal},
  see={atmospheric pressure}
}


% TODO reference
\newglossaryentry{wind}{
  name={wind},
  description={TODO},
  see={directional wind,calm,light wind,strong wind,storm,hurricane,weather phenomenon}
}
\newglossaryentry{directional wind}{
  name={directional wind},
  description={Wind including a direction},
  see={wind,north wind,south wind,west wind,east wind}
}
\newglossaryentry{north wind}{
  name={north wind},
  description={Wind approximately coming from the north, e.g. originating from a direction of at least 315 or less than 45 degrees},
  see={wind,directional wind}
}
\newglossaryentry{east wind}{
  name={east wind},
  description={Wind approximately coming from the east, e.g. originating from a direction of at least 45 and less than 135 degrees},
  see={wind,directional wind}
}
\newglossaryentry{south wind}{
  name={south wind},
  description={Wind approximately coming from the south, e.g. originating from a direction of at least 135 and less than 225 degrees},
  see={wind,directional wind}
}
\newglossaryentry{west wind}{
  name={wind wind},
  description={Wind approximately coming from the wind, e.g. originating from a direction of at least 225 and less than 315 degrees},
  see={wind,directional wind}
}
\newglossaryentry{calm}{
  name={calm},
  description={Wind speed below 1\metre\per\second},
  see={wind}
}
\newglossaryentry{light wind}{
  name={light wind},
  description={Wind with a speed of at least 1 and less than 10\metre\per\second},
  see={wind}
}
\newglossaryentry{strong wind}{
  name={strong wind},
  description={Wind with a speed of at least 10 and less than 20\metre\per\second},
  see={wind}
}
\newglossaryentry{storm}{
  name={storm},
  description={Wind with a speed of at least 20\metre\per\second},
  see={wind}
}
\newglossaryentry{hurricane}{
  name={hurricane},
  description={Wind with a speed of at least 32\metre\per\second},
  see={wind}
}


\newglossaryentry{cloud cover}{
  name={cloud cover},
  description={Concept describing the current cloud coverage in \emph{okta} (integer numbers from 0 to 9). Sub-concepts are \emph{Clear sky}, \emph{Partly cloudy}, \emph{Mostly cloudy}, \emph{Overcast} and \emph{Unknown cloud cover}},
  see={okta,clear sky,partly cloudy,mostly cloudy,overcast,unknown cloud cover,weather phenomenon}
}
\newglossaryentry{okta}{
  name={okta},
  description={Unit of measurement that specifies the amount of cloud cover on a range from 0 (\emph{Clear sky}) to 8 (\emph{Overcast}); a value of 9 represents an \emph{Unknown cloud cover}},
  see={clear sky,cloud cover,mostly cloudy,overcast,partly cloudy}
}
\newglossaryentry{clear sky}{
  name={clear sky},
  description={Cloud coverage is reported as to be 0$\:$\emph{okta}}
  see={okta,cloud cover}
}
\newglossaryentry{partly cloudy}{
  name={partly cloudy},
  description={Cloud coverage is reported as to be 1, 2, 3 or 4$\:$\emph{okta}}
  see={okta,cloud cover}
}
\newglossaryentry{mostly cloudy}{
  name={mostly cloudy},
  description={Cloud coverage is reported as to be 5, 6, or 7$\:$\emph{okta}}
  see={okta,cloud cover}
}
\newglossaryentry{overcast}{
  name={overcast},
  description={Cloud coverage is reported as to be 8$\:$\emph{okta}}
  see={okta,cloud cover}
}
\newglossaryentry{unknown cloud cover}{
  name={unknown cloud cover},
  description={Cloud coverage is reported as to be 9$\:$\emph{okta}}
  see={okta,cloud cover}
}

\newglossaryentry{weather report}{
  name={weather report},
  description={Summarizes all data aquired at a certain \emph{observation time} for a certain \emph{location} about the current weather or the weather some time in the future. Exactly one \emph{weather state} is linked to each \emph{weather report}.},
  see={weather state,current weather report,forecast weather report,observation time,location}
}
\newglossaryentry{current weather report}{
  name={current weather report},
  description={\emph{Weather report} describing the current weather},
  see={weather report}
}
\newglossaryentry{weather report from sensor}{
  name={current weather report from sensor},
  description={A \emph{weather report} containing data obtained from one or more \emph{sensor sources}},
  see={weather report,sensor source}
}
\newglossaryentry{weather report from service}{
  name={current weather report from service},
  description={A \emph{weather report} containing data obtained from an Internet \emph{service source}},
  see={weather report,service source}
}
\newglossaryentry{current weather report from sensor}{
  name={current weather report from sensor},
  description={\emph{Current weather report} compiled from data from a \emph{sensor source}},
  see={weather report,current weather report,sensor source}
}
\newglossaryentry{current weather report from service}{
  name={current weather report from service},
  description={\emph{Current weather report} compiled from data from a \emph{service source}},
  see={weather report,current weather report,service source}
}
\newglossaryentry{forecast weather report}{
  name={forecast weather report},
  description={\emph{Weather report} for some time after its \emph{observation time}},
  see={weather report,observation time,long range weather report,short range weather report,medium range weather report}
}
\newglossaryentry{short range weather report}{
  name={short range weather report},
  description={\emph{Weather report} describing a forecast for a point of time at most 3 hours in the future relative to the \emph{observation time}},
  see={weather report,forecast weather report,observation time}
}
\newglossaryentry{medium range weather report}{
  name={medium range weather report},
  description={\emph{Weather report} describing a forecast for a point of time more than 3 hours and at most 12 hours in the future relative to the \emph{observation time}},
  see={weather report,forecast weather report,observation time}
}
\newglossaryentry{long range weather report}{
  name={long range weather report},
  description={\emph{Weather report} describing a forecast for a point of time more than 12 hours in the future relative to the \emph{observation time}},
  see={weather report,forecast weather report,observation time}
}
\newglossaryentry{observation time}{
  name={observation time},
  description={Time when the weather data was collected from the weather sensor or the Internet weather service},
  see={weather report}
}
\newglossaryentry{start time}{
  name={start time},
  description={The time a \emph{Weather report} is valid from, relative to the \emph{Weather report}'s \emph{Observation time}},
  see={weather report,start time,observation time}
}
\newglossaryentry{end time}{
  name={end time},
  description={The time a \emph{Weather report} is valid until, relative to the \emph{Weather report}'s \emph{Observation time}},
  see={weather report,start time,observation time}
}
\newglossaryentry{priority}{
  name={priority},
  description={An integer value indicating which \emph{weather report} for a certain period of time is to be preferred over another \emph{weather report} for the same period of time.},
  see={weather report}
}


\newglossaryentry{dew point}{
  name={dew point},
  description={TODO},
  see={weather phenomenon}
}


\newglossaryentry{humidity}{
  name={humidity},
  description={TODO},
  see={weather phenomenon,very dry,dry,normal humidity,moist,very moist}
}
\newglossaryentry{very dry}{
  name={very dry},
  description={Relative humidity of less than 30 percent},
  see={humidity}
}
\newglossaryentry{dry}{
  name={dry},
  description={Relative humidity of at least 30 and less than 40 percent},
  see={humidity}
}
\newglossaryentry{normal humidity}{
  name={normal humidity},
  description={Relative humidity of at least 40 and at most 70 percent},
  see={humidity}
}
\newglossaryentry{moist}{
  name={moist},
  description={Relative humidity of more than 70 and at most 80 percent},
  see={humidity}
}
\newglossaryentry{very moist}{
  name={very moist},
  description={Relative humidity of more than percent},
  see={humidity}
}



% TODO probability of precipitation
\newglossaryentry{precipitation}{
  name={precipitation},
  description={TODO},
  see={no rain,light rain,medium rain,heavy rain,extremely heavy rain,weather phenomenon}
}
\newglossaryentry{no rain}{
  name={no rain},
  description={No precipitation (because the chance or the intensity of precipitation is 0)},
  see={precipitation}
}
\newglossaryentry{light rain}{
  name={light rain},
  description={More than 0 and at most 5\milli\metre\per\hour \emph{precipitation} per hour.},
  see={precipitation}
}
\newglossaryentry{medium rain}{
  name={medium rain},
  description={More than 5 and at most 20\milli\metre\per\hour \emph{precipitation} per hour.},
  see={precipitation}
}
\newglossaryentry{heavy rain}{
  name={heavy rain},
  description={More than 20 and at most 50\milli\metre\per\hour \emph{precipitation} per hour},
  see={precipitation}
}
\newglossaryentry{extremely heavy rain}{
  name={extremely heavy rain},
  description={More than 50 and at most 100\milli\metre\per\hour \emph{precipitation} per hour},
  see={precipitation}
}
\newglossaryentry{tropical storm rain}{
  name={tropical storm rain},
  description={More than 100\milli\metre\per\hour \emph{precipitation} per hour},
  see={precipitation}
}


\newglossaryentry{solar radiation}{
  name={solar radiation},
  description={TODO},
  see={weather phenomenon,no radiation,low radiation,medium radiation,high radiation,very high radiation}
}
\newglossaryentry{no radiation}{
  name={no radiation},
  description={No solar radiation (0\watt\per\squaren\metre)},
  see={solar radiation}
}
\newglossaryentry{low radiation}{
  name={low radiation},
  description={Solar radiation of more than 0\watt\per\squaren\metre and less than 250\watt\per\squaren\metre},
  see={solar radiation}
}
\newglossaryentry{medium radiation}{
  name={medium radiation},
  description={Solar radiation of at least 250\watt\per\squaren\metre and less than 500\watt\per\squaren\metre},
  see={solar radiation}
}
\newglossaryentry{high radiation}{
  name={high radiation},
  description={Solar radiation of at least 500\watt\per\squaren\metre and less than 750\watt\per\squaren\metre},
  see={solar radiation}
}
\newglossaryentry{very high radiation}{
  name={very high radiation},
  description={Solar radiation of at least 750\watt\per\squaren\metre},
  see={solar radiation}
}


\newglossaryentry{location}{
  name={location},
  description={Geographical position a \emph{weather report} is valid for, given by \emph{altitude}, \emph{longitude} and \emph{latitude}, using the \emph{WGS84} reference model\cite{WGS84}},
  see={weather report,altitude,longitude,latitude}
}
\newglossaryentry{latitude}{
  name={latitude},
  description={Latitude of the \emph{location} a \emph{weather report} is valid for, given as a decimal number. Positive values refer to positions north of the equator, negative values refer to positions south of the equator},
  see={weather report,location}
}
\newglossaryentry{longitude}{
  name={longitude},
  description={Longitude of the \emph{location} a \emph{weather report} is valid for, given as a decimal number. Positive values refer to positions east of the prime meridian at Greenwich, negative values refer to positions west of the prime meridian},
  see={weather report,location}
}
\newglossaryentry{altitude}{
  name={altitude},
  description={Height above \emph{MSL} (\emph{mean sea level}) of the \emph{location} the \emph{weather report} is valid for, in metres},
  see={weather report,location}
}


\newglossaryentry{weather source}{
  name={weather source},
  description={A source for weather data, either a set of \emph{weather sensors} or an Internet \emph{weather service}},
  see={service source,sensor source}
}
\newglossaryentry{sensor source}{
  name={sensor source},
  description={A sensor offering some kind of weather data},
  see={weather source}
}
\newglossaryentry{service source}{
  name={service source},
  description={An Internet weather service offering some kind of weather data},
  see={weather source}
}

\newglossaryentry{weather condition}{
  name={weather condition},
  description={Overall state of the weather, represented by (a combination of) these individuals: \emph{Sun}, \emph{Light clouds}, \emph{Partly cloudy}, \emph{Cloudy}, \emph{Fog}, \emph{Rain}, \emph{Snow}, \emph{Sleet}, and \emph{Thunder}},
  see={weather state}
}
\newglossaryentry{weather phenomenon}{
  name={weather phenomenon},
  description={Represents a certain weather element. Relevant weather elements are \emph{Temperature}, \emph{Humidity}, \emph{Dew point}, \emph{Wind}, \emph{Precipitation}, \emph{Atmospheric pressure}, \emph{Cloud cover}, \emph{Solar radiation}, and the \emph{Sun position}},
  see={temperature,humidity,dew point,wind,precipitation,atmospheric pressure,cloud cover,solar radiation,sun position}
}

\newglossaryentry{weather state}{
  name={weather state},
  description={TODO},
  see={weather report,weather phenomenon}
}

