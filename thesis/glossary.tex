\newglossaryentry{temperature}{
  name={temperature},
  description={Sub-concept of \emph{Weather phenomenon} representing temperature. The temperature value is specified using the property \emph{has temperature value}. Sub-concepts are \emph{Frost}, \emph{Cold}, \emph{Below room temperature}, \emph{Room temperature}, \emph{Above room temperature}, and \emph{Heat}},
  see={weather phenomenon,has temperature value,frost,cold,below room temperature,room temperature,above room temperature,heat}
}
\newglossaryentry{room temperature}{
  name={room temperature},
  description={Sub-concept of \emph{Temperature} representing a temperature of least \num{20} and at most \SI{25}{\celsius}},
  see={temperature}
}
\newglossaryentry{above room temperature}{
  name={abobe room temperature},
  description={Sub-concept of \emph{Temperature} representing a temperature of more than \num{25} and less than or equal to \SI{30}{\celsius}},
  see={temperature}
}
\newglossaryentry{below room temperature}{
  name={below room temperature},
  description={Sub-concept of \emph{Temperature} representing a temperature of more than or equal to \num{10} and less than \SI{20}{\celsius}},
  see={temperature}
}
\newglossaryentry{cold}{
  name={cold},
  description={Sub-concept of \emph{Temperature} representing a temperature of more than or equal to \num{0} and less than \SI{10}{\celsius}},
  see={temperature}
}
\newglossaryentry{frost}{
  name={frost},
  description={Sub-concept of \emph{Temperature} representing a temperature below \SI{0}{\celsius}},
  see={temperature}
}
\newglossaryentry{heat}{
  name={heat},
  description={Sub-concept of \emph{Temperature} representing a temperature above \SI{30}{\celsius}},
  see={temperature}
}


\newglossaryentry{sun position}{
  name={sun position},
  description={Sub-concept of \emph{Weather phenomenon} describing the position of the sun, given by its elevation above horizon (specified by the property \emph{has sun elevation angle} and its direction \emph{has sun direction}. Sub-concepts are \emph{Day}, \emph{Solar twilight}, \emph{Twilight}, \emph{Civil twilight}, \emph{Nautical twilight}, \emph{Astronomical twilight}, and \emph{Night} for the elevation above horizon, and \emph{Sun from north}, \emph{Sun from east}, \emph{Sun from South}, and \emph{Sun from West} for the direction.},
  see={weather phenomenon,has sun elevation angle,has sun direction,day,solar twilight,twilight,civil twilight,nautical twilight,astronomical twilight,night,sun from north,sun from east,sun from south,sun from west}
}
\newglossaryentry{day}{
  name={day},
  description={Sub-concept of \emph{sun position} that represents the sun being exactly at or above the horizon},
  see={sun position}
}
\newglossaryentry{night}{
  name={night},
  description={Sub-concept of \emph{sun position} that represents the sun being more than \num{18} degrees below horizon},
  see={sun position}
}
\newglossaryentry{solar twilight}{
  name={solar twilight},
  description={Sub-concept of \emph{sun position} that represents the sun being above horizon, but less than \num{6} degrees above},
  see={sun position,day}
}
\newglossaryentry{sun below horizon}{
  name={sun below horizon},
  description={Sub-concept of \emph{sun position} that represents the sun being below the horizon},
  see={sun position,night,twilight}
}
\newglossaryentry{twilight}{
  name={twilight},
  description={Sub-concept of \emph{sun position} that represents the sun being below horizon and at most \num{18} degrees below horizon},
  see={sun position,civil twilight,nautical twilight,astronomical twilight}
}
\newglossaryentry{civil twilight}{
  name={civil twilight},
  description={Sub-concept of \emph{sun position} that represents the sun being below horizon and at most \num{6} degrees below horizon},
  see={sun position,twilight}
}
\newglossaryentry{nautical twilight}{
  name={nautical twilight},
  description={Sub-concept of \emph{sun position} that represents the sun being more than \num{6} degrees and no more than \num{12} degrees below horizon},
  see={sun position,twilight}
}
\newglossaryentry{astronomical twilight}{
  name={astronomical twilight},
  description={Sub-concept of \emph{sun position} that represents the sun being more than \num{12} degrees and no more than \num{18} degrees below horizon},
  see={sun position,twilight}
}
\newglossaryentry{sun from north}{
  name={sun from north},
  description={Sub-concept of \emph{sun position} that represents a sun direction of more than \num{315} or at most \num{45} degrees},
  see={sun position}
}
\newglossaryentry{sun from east}{
  name={sun from east},
  description={Sub-concept of \emph{sun position} that represents a sun direction of more than \num{45} and at most \num{135} degrees},
  see={sun position}
}
\newglossaryentry{sun from south}{
  name={sun from south},
  description={Sub-concept of \emph{sun position} that represents a sun direction of more than \num{135} and at most \num{225} degrees},
  see={sun position}
}
\newglossaryentry{sun from west}{
  name={sun from west},
  description={Sub-concept of \emph{sun position} that represents a sun direction of more than \num{225} and at most \num{315} degrees},
  see={sun position}
}



\newglossaryentry{atmospheric pressure}{
  name={atmospheric pressure},
  description={Sub-concept of \emph{Weather phenomenon} representing atmospheric pressure. The value is specified using the property \emph{has pressure value}. Sub-concepts are \emph{Very low pressure}, \emph{Low pressure}, \emph{Average pressure}, \emph{High pressure}, and \emph{Very high pressure}},
  see={weather phenomenon,has pressure value,very low pressure,low pressure,average pressure,high pressure,very high pressure}
}
\newglossaryentry{very low pressure}{
  name={very low pressure},
  description={Sub-concept of \emph{Atmospheric pressure} representing a pressure of less than \SI{998}{\hecto\pascal}},
  see={atmospheric pressure}
}
\newglossaryentry{low pressure}{
  name={low pressure},
  description={Sub-concept of \emph{Atmospheric pressure} representing a pressure of at least \num{998} and less than \SI{1008}{\hecto\pascal}},
  see={atmospheric pressure}
}
\newglossaryentry{average pressure}{
  name={average pressure},
  description={Sub-concept of \emph{Atmospheric pressure} representing a pressure of at least \num{1008} and less than \SI{1018}{\hecto\pascal}},
  see={atmospheric pressure}
}
\newglossaryentry{high pressure}{
  name={high pressure},
  description={Sub-concept of \emph{Atmospheric pressure} representing a pressure of at least \num{1018} and less than \SI{1028}{\hecto\pascal}},
  see={atmospheric pressure}
}
\newglossaryentry{very high pressure}{
  name={very high pressure},
  description={Sub-concept of \emph{Atmospheric pressure} representing a pressure of at least \SI{1028}{\hecto\pascal}},
  see={atmospheric pressure}
}

\newglossaryentry{wind}{
  name={wind},
  description={Sub-concept of \emph{Weather phenomenon} representing wind (speed and direction, the latter is optional). Wind speed is given using the property \emph{has wind speed}; wind direction is specified using \emph{has wind direction}. Sub-concepts are \emph{Directional wind}, \emph{North wind}, \emph{East wind}, \emph{South wind}, and \emph{West wind} for the wind direction, and \emph{Calm}, \emph{Light wind}, \emph{Strong wind}, \emph{Storm}, and \emph{Hurricane} for the wind speed},
  see={weather phenomenon,has wind speed,has wind direction,directional wind,north wind,east wind,south wind,west wind,calm,light wind,strong wind,storm,hurricane}
}
\newglossaryentry{directional wind}{
  name={directional wind},
  description={Sub-concept of \emph{Wind} including a direction},
  see={wind,north wind,east wind,south wind,west wind}
}
\newglossaryentry{north wind}{
  name={north wind},
  description={Sub-concept of \emph{Wind} representing wind approximately coming from the north, e.g. originating from a direction of at least \num{315} or less than \num{45} degrees},
  see={wind,directional wind}
}
\newglossaryentry{east wind}{
  name={east wind},
  description={Sub-concept of \emph{Wind} representing wind approximately coming from the east, e.g. originating from a direction of at least \num{45} and less than \num{135} degrees},
  see={wind,directional wind}
}
\newglossaryentry{south wind}{
  name={south wind},
  description={Sub-concept of \emph{Wind} representing wind approximately coming from the south, e.g. originating from a direction of at least \num{135} and less than \num{225} degrees},
  see={wind,directional wind}
}
\newglossaryentry{west wind}{
  name={wind wind},
  description={Sub-concept of \emph{Wind} representing wind  approximately coming from the wind, e.g. originating from a direction of at least \num{225} and less than \num{315} degrees},
  see={wind,directional wind}
}
\newglossaryentry{calm}{
  name={calm},
  description={Sub-concept of \emph{Wind} representing wind with a speed below \SI{1}{\metre\per\second}},
  see={wind}
}
\newglossaryentry{light wind}{
  name={light wind},
  description={Sub-concept of \emph{Wind} representing wind with a speed of at least \num{1} and less than \SI{10}{\metre\per\second}},
  see={wind}
}
\newglossaryentry{strong wind}{
  name={strong wind},
  description={Sub-concept of \emph{Wind} representing wind with a speed of at least \num{10} and less than \SI{20}{\metre\per\second}},
  see={wind}
}
\newglossaryentry{storm}{
  name={storm},
  description={Sub-concept of \emph{Wind} representing wind  with a speed of at least \SI{20}{\metre\per\second}},
  see={wind}
}
\newglossaryentry{hurricane}{
  name={hurricane},
  description={Sub-concept of \emph{Wind} representing wind with a speed of at least \SI{32}{\metre\per\second}},
  see={wind}
}


\newglossaryentry{cloud cover}{
  name={cloud cover},
  description={Sub-concept of \emph{Weather phenomenon} describing the current cloud coverage in \emph{okta} (integer numbers from \num{0} to \num{9}) which is given using the property \emph{has cloud cover}; the altitude of the cloud layer is given by the property \emph{has cloud altitude}. Sub-concepts are \emph{Clear sky}, \emph{Partly cloudy}, \emph{Mostly cloudy}, \emph{Overcast}, and \emph{Unknown cloud cover}},
  see={weather phenomenon,has cloud cover,has cloud altitude,okta,clear sky,partly cloudy,mostly cloudy,overcast,unknown cloud cover}
}
\newglossaryentry{okta}{
  name={okta},
  description={Unit of measurement that specifies the amount of \emph{cloud cover} on a range from \num{0} (\emph{Clear sky}) to \num{8} (\emph{Overcast}); a value of \num{9} represents an \emph{Unknown cloud cover}},
  see={cloud cover,clear sky,partly cloudy,mostly cloudy,overcast,unknown cloud cover}
}
\newglossaryentry{clear sky}{
  name={clear sky},
  description={Sub-concept of \emph{Cloud cover} representing cloud coverage that is reported to be \num{0}$\:$\emph{okta}},
   see={okta,cloud cover}
}
\newglossaryentry{partly cloudy}{
  name={partly cloudy},
  description={Sub-concept of \emph{Cloud cover} representing cloud coverage that is reported to be \num{1}, \num{2}, \num{3}, or \num{4}$\:$\emph{okta}},
  see={okta,cloud cover}
}
\newglossaryentry{mostly cloudy}{
  name={mostly cloudy},
  description={Sub-concept of \emph{Cloud cover} representing cloud coverage that is reported to be \num{5}, \num{6}, or \num{7}$\:$\emph{okta}},
  see={okta,cloud cover}
}
\newglossaryentry{overcast}{
  name={overcast},
  description={Sub-concept of \emph{Cloud cover} representing cloud coverage that is reported to be \num{8}$\:$\emph{okta}},
  see={okta,cloud cover}
}
\newglossaryentry{unknown cloud cover}{
  name={unknown cloud cover},
  description={Sub-concept of \emph{Cloud cover} representing cloud coverage that is reported to be \num{9}$\:$\emph{okta}},
  see={okta,cloud cover}
}

\newglossaryentry{weather report}{
  name={weather report},
  description={Concept that summarizes all data acquired at a certain \emph{observation time} for a certain \emph{location} from a \emph{weather source}, about the current weather or the weather some time in the future. Exactly one \emph{weather state} is linked to each \emph{weather report}},
  see={weather state,current weather report,forecast weather report,weather source,observation time,start time,end time,location}
}
\newglossaryentry{current weather report}{
  name={current weather report},
  description={Sub-concept of \emph{Weather report} describing the current weather},
  see={weather report}
}
\newglossaryentry{weather report from sensor}{
  name={weather report from sensor},
  description={Sub-concept of \emph{weather report}; contains data obtained from one or more \emph{sensor sources}},
  see={weather report,sensor source}
}
\newglossaryentry{weather report from service}{
  name={weather report from service},
  description={Sub-concept of \emph{weather report}; contains data obtained from an Internet \emph{service source}},
  see={weather report,service source}
}
\newglossaryentry{current weather report from sensor}{
  name={current weather report from sensor},
  description={Sub-concept of \emph{Current weather report}; compiled from data from a \emph{sensor source}},
  see={weather report,current weather report,sensor source}
}
\newglossaryentry{current weather report from service}{
  name={current weather report from service},
  description={Sub-concept of \emph{Current weather report}; compiled from data from a \emph{service source}},
  see={weather report,current weather report,service source}
}
\newglossaryentry{forecast weather report}{
  name={forecast weather report},
  description={Sub-concept of \emph{Weather report} for some time after its \emph{observation time}},
  see={weather report,observation time,short range weather report,medium range weather report,long range weather report}
}
\newglossaryentry{short range weather report}{
  name={short range weather report},
  description={Sub-concept of \emph{Weather report} describing a forecast for a point of time at most \num{3} hours in the future relative to the \emph{observation time}. Its sub-concepts are \emph{Forecast 1 hour weather report}, \emph{Forecast 2 hours weather report}, and \emph{Forecast 3 hours weather report}},
  see={weather report,forecast weather report,observation time}
}
\newglossaryentry{medium range weather report}{
  name={medium range weather report},
  description={Sub-concept of \emph{Weather report} describing a forecast for a point of time more than \num{3} hours and less than \num{12} hours in the future relative to the \emph{observation time}. Its sub-concepts are \emph{Forecast 6 hours weather report}, \emph{Forecast 9 hours weather report}, and \emph{Forecast 12 hours weather report}},
  see={weather report,forecast weather report,observation time}
}
\newglossaryentry{long range weather report}{
  name={long range weather report},
  description={Sub-concept of \emph{Weather report} describing a forecast for a point of time at least \num{12} hours in the future relative to the \emph{observation time}. Its sub-concepts are \emph{Forecast 15 hours weather report}, \emph{Forecast 18 hours weather report}, \emph{Forecast 21 hours weather report}, and \emph{Forecast 24 hours weather report}},
  see={weather report,forecast weather report,observation time}
}
\newglossaryentry{observation time}{
  name={observation time},
  description={The time when the weather data was collected from the weather sensor or the Internet weather service, given by an instance of the concept \emph{Instant} from \emph{OWL-Time}\cite{owl-time}},
  see={weather report,has observation time,instant}
}
\newglossaryentry{has observation time}{
  name={has observation time},
  description={An object property of \emph{Weather report} that specifies the report's \emph{observation time}},
  see={weather report,observation time}
}
\newglossaryentry{start time}{
  name={start time},
  description={The time a \emph{Weather report} is valid from, given by an instance of the concept \emph{Interval} from \emph{OWL-Time}\cite{owl-time} that specifies the interval between the report's \emph{observation time} and its \emph{start time}},
  see={weather report,has start time,end time,observation time,interval}
}
\newglossaryentry{end time}{
  name={end time},
  description={The time a \emph{Weather report} is valid until, given by an instance of the concept \emph{Interval} from \emph{OWL-Time}\cite{owl-time} that specifies the interval between the report's \emph{observation time} and its \emph{end time}},
  see={weather report,has end time,start time,observation time,interval}
}
\newglossaryentry{has start time}{
  name={has start time},
  description={An object property of \emph{weather report} that specifies the report's \emph{start time}},
  see={weather report,start time,has end time,observation time}
}
\newglossaryentry{has end time}{
  name={has end time},
  description={An object property of \emph{weather report} that specifies the report's \emph{end time}},
  see={weather report,end time,has start time,observation time}
}
\newglossaryentry{priority}{
  name={priority},
  description={An integer value indicating which \emph{weather report} for a certain period of time is to be preferred over another \emph{weather report} for the same period of time},
  see={weather report,has priority}
}
\newglossaryentry{has priority}{
  name={has priority},
  description={A data property of \emph{weather report} that specifies the report's \emph{priority}},
  see={weather report,priority}
}


\newglossaryentry{dew point}{
  name={dew point},
  description={Sub-concept of \emph{Weather phenomenon} that describes the dew point. The value is given using the property \emph{has dew point value}},
  see={weather phenomenon,has dew point value}
}

\newglossaryentry{humidity}{
  name={humidity},
  description={Sub-concept of \emph{Weather phenomenon} representing the relative humidity of the air; the humidity value is specified using the property \emph{has humidity value}. Sub-concepts are \emph{Very dry}, \emph{Dry}, \emph{Normal humidity}, \emph{Moist}, and \emph{Very moist}},
  see={weather phenomenon,has humidity value,very dry,dry,normal humidity,moist,very moist}
}
\newglossaryentry{very dry}{
  name={very dry},
  description={Sub-concept of \emph{Humidity} describing relative humidity of less than \num{30} percent},
  see={humidity}
}
\newglossaryentry{dry}{
  name={dry},
  description={Sub-concept of \emph{Humidity} describing relative humidity of at least \num{30} and less than \num{40} percent},
  see={humidity}
}
\newglossaryentry{normal humidity}{
  name={normal humidity},
  description={Sub-concept of \emph{Humidity} describing relative humidity of at least \num{40} and at most \num{70} percent},
  see={humidity}
}
\newglossaryentry{moist}{
  name={moist},
  description={Sub-concept of \emph{Humidity} describing relative humidity of more than \num{70} and at most \num{80} percent},
  see={humidity}
}
\newglossaryentry{very moist}{
  name={very moist},
  description={Sub-concept of \emph{Humidity} describing relative humidity of more than \num{80} percent},
  see={humidity}
}


\newglossaryentry{precipitation}{
  name={precipitation},
  description={Sub-concept of \emph{Weather phenomenon} that describes precipitation (intensity and probability). Intensity is specified by the property \emph{has precipitation intensity}, probability by \emph{has precipitation probability}. Sub-concepts are \emph{No rain}, \emph{Light rain}, \emph{Medium rain}, \emph{Heavy rain}, \emph{Extremely heavy rain}, and \emph{Tropical storm rain}},
  see={weather phenomenon,has precipitation intensity,has precipitation probability,no rain,light rain,medium rain,heavy rain,extremely heavy rain,tropical storm rain}
}
\newglossaryentry{no rain}{
  name={no rain},
  description={Sub-concept of \emph{Precipitation} representing absence precipitation (because either the intensity of precipitation is \num{0})},
  see={precipitation}
}
\newglossaryentry{light rain}{
  name={light rain},
  description={Sub-concept of \emph{Precipitation} representing a precipitation probability greater than \num{0} and an intensity of more than \num{0} and at most \SI{5}{\milli\metre\per\hour}},
  see={precipitation}
}
\newglossaryentry{medium rain}{
  name={medium rain},
  description={Sub-concept of \emph{Precipitation} representing a precipitation probability greater than \num{0} and an intensity of more than \num{5} and at most \SI{20}{\milli\metre\per\hour}},
  see={precipitation}
}
\newglossaryentry{heavy rain}{
  name={heavy rain},
  description={Sub-concept of \emph{Precipitation} representing a precipitation probability greater than \num{0} and an intensity of more than \num{20} and at most \SI{50}{\milli\metre\per\hour}},
  see={precipitation}
}
\newglossaryentry{extremely heavy rain}{
  name={extremely heavy rain},
  description={Sub-concept of \emph{Precipitation} representing a precipitation probability greater than \num{0} and an intensity of more than \num{50} and at most \SI{100}{\milli\metre\per\hour}},
  see={precipitation}
}
\newglossaryentry{tropical storm rain}{
  name={tropical storm rain},
  description={Sub-concept of \emph{Precipitation} representing a precipitation probability greater than \num{0} and an intensity of more than \SI{100}{\milli\metre\per\hour}},
  see={precipitation}
}


\newglossaryentry{solar radiation}{
  name={solar radiation},
  description={Sub-concept of \emph{Weather phenomenon} representing solar radiation; the value is specified using the property \emph{has solar radiation value}. Sub-concepts are \emph{No radiation}, \emph{Low radiation}, \emph{Mediumr radiation}, \emph{High radiation}, and \emph{Very high radiation}},
  see={weather phenomenon,has solar radiation value,no radiation,low radiation,medium radiation,high radiation,very high radiation}
}
\newglossaryentry{no radiation}{
  name={no radiation},
  description={Sub-concept of \emph{Solar radiation} describing the absence of any solar radiation (\SI{0}{\watt\per\square\metre})},
  see={solar radiation}
}
\newglossaryentry{low radiation}{
  name={low radiation},
  description={Sub-concept of \emph{Solar radiation} describing solar radiation of more than \SI{0}{\watt\per\square\metre} and less than \SI{250}{\watt\per\square\metre}},
  see={solar radiation}
}
\newglossaryentry{medium radiation}{
  name={medium radiation},
  description={Sub-concept of \emph{Solar radiation} describing solar radiation of at least \SI{250}{\watt\per\square\metre} and less than \SI{500}{\watt\per\square\metre}},
  see={solar radiation}
}
\newglossaryentry{high radiation}{
  name={high radiation},
  description={Sub-concept of \emph{Solar radiation} describing solar radiation of at least \SI{500}{\watt\per\square\metre} and less than \SI{750}{\watt\per\square\metre}},
  see={solar radiation}
}
\newglossaryentry{very high radiation}{
  name={very high radiation},
  description={Sub-concept of \emph{Solar radiation} describing solar radiation of at least \SI{750}{\watt\per\square\metre}},
  see={solar radiation}
}


\newglossaryentry{location}{
  name={location},
  description={An object property of \emph{weather report} that specifies the geographical position the \emph{weather report} is valid for, given by \emph{altitude}, \emph{longitude} and \emph{latitude}, using the \emph{WGS84} reference model\cite{WGS84}; links to an instance of the concept \emph{point} of the \emph{Basic Geo (WGS84 lat/long) Vocabulary}\cite{wgs84_vocabulary}},
  see={weather report,lat,long,alt}
}
\newglossaryentry{lat}{
  name={lat},
  description={Property defined by the \emph{Basic Geo (WGS84 lat/long) Vocabulary}\cite{wgs84_vocabulary} specifying the latitude of the \emph{location} a \emph{weather report} is valid for, given as a decimal number. Positive values refer to positions north of the equator, negative values refer to positions south of the equator},
  see={weather report,location}
}
\newglossaryentry{long}{
  name={long},
  description={Property defined by the \emph{Basic Geo (WGS84 lat/long) Vocabulary}\cite{wgs84_vocabulary} specifying the longitude of the \emph{location} a \emph{weather report} is valid for, given as a decimal number. Positive values refer to positions east of the prime meridian at Greenwich, negative values refer to positions west of the prime meridian},
  see={weather report,location}
}
\newglossaryentry{alt}{
  name={alt},
  description={Property defined by the \emph{Basic Geo (WGS84 lat/long) Vocabulary}\cite{wgs84_vocabulary} specifying the height above \emph{MSL} (\emph{mean sea level}) of the \emph{location} the \emph{weather report} is valid for, in metres},
  see={weather report,location}
}


\newglossaryentry{weather source}{
  name={weather source},
  description={A concept representing a source for weather data, either a set of \emph{weather sensors} or an Internet \emph{weather service}. Sub-concepts are \emph{Sensor source}, and \emph{Service source}},
  see={sensor source,service source}
}
\newglossaryentry{sensor source}{
  name={sensor source},
  description={A sub-concept of \emph{Weather Source} representing a sensor or a set of sensors offering some kind of weather data},
  see={weather source}
}
\newglossaryentry{service source}{
  name={service source},
  description={A sub-concept of \emph{Weather Source} representing an Internet weather service offering some kind of weather data},
  see={weather source}
}

\newglossaryentry{weather condition}{
  name={weather condition},
  description={A concept describing the overall state of the weather, represented by (a combination of) these individuals: \emph{Sun}, \emph{Light clouds}, \emph{Partly cloudy}, \emph{Cloudy}, \emph{Fog}, \emph{Rain}, \emph{Snow}, \emph{Sleet}, and \emph{Thunder}},
  see={weather state}
}
\newglossaryentry{weather phenomenon}{
  name={weather phenomenon},
  description={A concept that represents a certain weather element. Relevant weather elements are \emph{Atmospheric pressure}, \emph{Cloud cover}, \emph{Dew point}, \emph{Humidity}, \emph{Precipitation}, \emph{Solar radiation}, \emph{Sun position}, \emph{Temperature}, and \emph{Wind}},
  see={atmospheric pressure,cloud cover,dew point,humidity,precipitation,solar radiation,sun position,temperature,wind}
}

\newglossaryentry{weather state}{
  name={weather state},
  description={A concept that represents all data available about weather phenomena for a certain \emph{Weather report}. A set of instances of the concept \emph{Weather phenomenon} are linked to every instance of \emph{Weather state}},
  see={weather report,weather phenomenon}
}


\newglossaryentry{has source}{
  name={has source},
  description={Object property that links instances of \emph{Weather report} and \emph{Weather source}; inverse property of \emph{is source of}},
  see={weather report,weather source,is source of}
}
\newglossaryentry{is source of}{
  name={is source of},
  description={Object property that links instances of \emph{Weather report} and \emph{Weather source}; inverse property of \emph{has source}},
  see={weather report,weather source,has source}
}
\newglossaryentry{has weather state}{
  name={has weather state},
  description={Object property that links instances of \emph{Weather report} and \emph{Weather state}; inverse property of \emph{belongs to weather report}},
  see={weather report,weather state,belongs to weather report}
}
\newglossaryentry{belongs to weather report}{
  name={belongs to weather report},
  description={Object property that links instances of \emph{Weather report} and \emph{Weather state}; inverse property of \emph{has weather state}},
  see={weather report,weather state,has weather state}
}
\newglossaryentry{has condition}{
  name={has condition},
  description={Object property that links instances of \emph{Weather state} and \emph{Weather condition}; does not have an inverse property},
  see={weather state,weather condition}
}
\newglossaryentry{has weather phenomenon}{
  name={has weather phenomenon},
  description={Object property that links instances of \emph{Weather state} and \emph{Weather phenomenon}; inverse property of \emph{belongs to state}},
  see={weather state,weather phenomenon,belongs to state}
}
\newglossaryentry{belongs to state}{
  name={belongs to state},
  description={Object property that links instances of \emph{Weather state} and \emph{Weather phenomenon}; inverse property of \emph{has weather phenomenon}},
  see={weather state,weather phenomenon,has weather phenomenon}
}

\newglossaryentry{has cloud altitude}{
  name={has cloud altitude},
  description={Property that specifies the cloud altitude of a cloud layer represented by an instance of \emph{Cloud cover}. This property makes use of concepts and properties defined by the \emph{Measurement Units Ontology}\cite{MUO} to provide both a value and its unit},
  see={cloud cover}
}
\newglossaryentry{has cloud cover}{
  name={has cloud cover},
  description={Property that specifies the cloud coverage of a cloud layer represented by an instance of \emph{Cloud cover}. This property makes use of concepts and properties defined by the \emph{Measurement Units Ontology}\cite{MUO} to provide both a value and its unit},
  see={cloud cover}
}
\newglossaryentry{has dew point value}{
  name={has dew point value},
  description={Property that specifies the dew point value of an instance of \emph{Dew point}. This property makes use of concepts and properties defined by the \emph{Measurement Units Ontology}\cite{MUO} to provide both a value and its unit},
  see={dew point}
}
\newglossaryentry{has humidity value}{
  name={has humidity value},
  description={Property that specifies the humidity value of an instance of \emph{Humidity}. This property makes use of concepts and properties defined by the \emph{Measurement Units Ontology}\cite{MUO} to provide both a value and its unit},
  see={humidity}
}
\newglossaryentry{has precipitation probability}{
  name={has precipitation probability},
  description={Property that specifies the precipitation probability of an instance of \emph{Precipitation}. This property makes use of concepts and properties defined by the \emph{Measurement Units Ontology}\cite{MUO} to provide both a value and its unit},
  see={precipitation}
}
\newglossaryentry{has precipitation intensity}{
  name={has precipitation intensity},
  description={Property that specifies the precipitation intensity of an instance of \emph{Precipitation}. This property makes use of concepts and properties defined by the \emph{Measurement Units Ontology}\cite{MUO} to provide both a value and its unit},
  see={precipitation}
}
\newglossaryentry{has pressure value}{
  name={has pressure value},
  description={Property that specifies the pressure value of an instance of \emph{Atmospheric pressure}. This property makes use of concepts and properties defined by the \emph{Measurement Units Ontology}\cite{MUO} to provide both a value and its unit},
  see={atmospheric pressure}
}
\newglossaryentry{has solar radiation value}{
  name={has solar radiation value},
  description={Property that specifies the solar radiation value of an instance of \emph{Solar radiation}. This property makes use of concepts and properties defined by the \emph{Measurement Units Ontology}\cite{MUO} to provide both a value and its unit},
  see={solar radiation}
}
\newglossaryentry{has sun direction}{
  name={has sun direction},
  description={Property that specifies the sun's direction of an instance of \emph{Sun position}. This property makes use of concepts and properties defined by the \emph{Measurement Units Ontology}\cite{MUO} to provide both a value and its unit},
  see={sun position}
}
\newglossaryentry{has sun elevation angle}{
  name={has sun elevation angle},
  description={Property that specifies the sun's elevation angle (the sun's height above horizon) of an instance of \emph{Sun position}. This property makes use of concepts and properties defined by the \emph{Measurement Units Ontology}\cite{MUO} to provide both a value and its unit},
  see={sun position}
}
\newglossaryentry{has temperature value}{
  name={has temperature value},
  description={Property that specifies the temperature value of an instance of \emph{Temperature}. This property makes use of concepts and properties defined by the \emph{Measurement Units Ontology}\cite{MUO} to provide both a value and its unit},
  see={temperature}
}
\newglossaryentry{has wind direction}{
  name={has wind direction},
  description={Property that specifies the wind direction of an instance of \emph{Wind}. This property makes use of concepts and properties defined by the \emph{Measurement Units Ontology}\cite{MUO} to provide both a value and its unit},
  see={wind}
}
\newglossaryentry{has wind speed}{
  name={has wind speed},
  description={Property that specifies the wind speed of an instance of \emph{Wind}. This property makes use of concepts and properties defined by the \emph{Measurement Units Ontology}\cite{MUO} to provide both a value and its unit},
  see={wind}
}

\newglossaryentry{has previous weather state}{
  name={has previous weather state},
  description={Property that links an instance of \emph{Weather state} to the immediately preceding instance concerning their \emph{start time}s, if such an instance exists. Both this property and its reverse property \emph{has next weather state} are functional properties},
  see={weather state,has next weather state}
}
\newglossaryentry{has next weather state}{
  name={has next weather state},
  description={Property that links an instance of \emph{Weather state} to the immediately succeeding instance concerning their \emph{start time}s, if such an instance exists. Both this property and its reverse property \emph{has previous weather state} are functional properties},
  see={weather state,has previous weather state}
}

\newglossaryentry{instant}{
  name={instant},
  description={Sub-concept of \emph{Temporal entity} defined by \emph{OWL-Time}\cite{owl-time} that specifies an instant},
  see={temporal entity,interval}
}
\newglossaryentry{interval}{
  name={interval},
  description={Sub-concept of \emph{Temporal entity} defined by \emph{OWL-Time}\cite{owl-time} that specifies a period of time by its length (not start and end time)},
  see={temporal entity,instant}
}
\newglossaryentry{temporal entity}{
  name={temporal entity},
  description={Concept defined by \emph{OWL-Time}\cite{owl-time}; either an \emph{Instant} or an \emph{Interval}},
  see={instant,interval}
}

\newglossaryentry{calm weather}{
  name={calm weather},
  description={TODO},
  see={}
}
\newglossaryentry{clear weather}{
  name={clear weather},
  description={TODO},
  see={}
}
\newglossaryentry{cold weather}{
  name={cold weather},
  description={TODO},
  see={}
}
\newglossaryentry{fair weather}{
  name={fair weather},
  description={TODO},
  see={}
}
\newglossaryentry{dry weather}{
  name={dry weather},
  description={TODO},
  see={}
}
\newglossaryentry{moist weather}{
  name={moist weather},
  description={TODO},
  see={}
}
\newglossaryentry{hot weather}{
  name={hot weather},
  description={TODO},
  see={}
}
\newglossaryentry{airing weather}{
  name={airing weather},
  description={TODO},
  see={}
}
\newglossaryentry{cloudy weather}{
  name={cloudy weather},
  description={TODO},
  see={}
}
\newglossaryentry{pleasant temperature weather}{
  name={pleasant temperature weather},
  description={TODO},
  see={}
}
\newglossaryentry{severe weather}{
  name={severe weather},
  description={TODO},
  see={}
}
\newglossaryentry{thunderstorm}{
  name={thunderstorm},
  description={TODO},
  see={}
}
\newglossaryentry{very rainy weather}{
  name={very rainy weather},
  description={TODO},
  see={}
}
\newglossaryentry{windy weather}{
  name={windy weather},
  description={TODO},
  see={}
}
\newglossaryentry{no rain weather}{
  name={no rain weather},
  description={TODO},
  see={}
}
\newglossaryentry{sun protection weather}{
  name={sun protection weather},
  description={TODO},
  see={}
}
\newglossaryentry{rainy weather}{
  name={rainy weather},
  description={TODO},
  see={}
}
