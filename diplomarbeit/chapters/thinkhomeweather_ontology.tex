% TODO put all names of steps according to \methontology into \emph
% TODO ontology - naming conventions
% TODO mention - this chapter does not cover the ontology development process as it was done; instead, it presents the completed documents that were created in that processs.
% TODO mention second paper about methontology (the one that describes all artefacts created when building the chemical ontology)

% TODO check if the things mentioned here are properly covered in the chapters referred

% TODO rename: belongsToWeatherReport -> belongsToReport

% TODO elaborate steps more?

% TODO reference paper: Building legal ontologies with METHONTOLOGY and WebODE

The previous chapters covered all aspects that have to be analyzed in order to build a completely new ontology from scratch: Chapter \ref{ch:weather_data} discussed all details about weather data necessary for the \thinkhome weather ontology, what data will be used and where to obain it. Chapter \ref{ch:existing_work} gave an overview over existing work regarding ontologies in the domain of weather data. As none of the existing ontologies fit the needs of an ontology for \thinkhome, chapter  \ref{ch:development_approaches} analyzed some of the most popular approaches for building ontologies from scratch. Among those \methontology was identified as the best suitable approach. % TODO citation for methontology?

Based on these findings, this chapter describes the process of building the \thinkhome weather ontology in detail. The structure follows the steps in the ontology development process introduced by \methontology.

\section{Specification}

Specification, the first step proposed by \methontology, aims at creating an ontology specification document using natural language. That document must include at least the intended purpose of the ontology, the level of 
formality and the ontology's scope.

% TODO ensure there are no problematic page breaks inside this box (e.g. between the headline 'Scope:' and the first bullet point)
\begin{mdframed}
\setlength{\parindent}{0pt}
\MakeUppercase{\textbf{Ontology requirements specification document}}

\vspace{.5cm}

\textbf{Domain}: Weather data

\vspace{.2cm}

\textbf{Date}: 2012 % TODO

\vspace{.2cm}

\textbf{Purpose}: The ontology shall cover data about weather phenomena occurring at a certain location somewhere on Earth between the present and 24 hours in the future. Weather data will be aquired from both internet services as well as from weather sensors mounted to a building. This weather data will enable the \textit{ThinkHome} system to make decisions based on current and future weather conditions.

\vspace{.2cm}

\textbf{Level of formality}: semi-informal % TODO citations for uschold & gruninger, 1996; uschuld & king (enterprise ontology)

\vspace{.2cm}

\textbf{Scope}:

\begin{itemize}
  \item \emph{Weather phenomenon}: Represents a certain weather element. Relevant weather elements are temperature, humidity, dew point, wind speed and direction, precipitation intensity and probability, atmospheric pressure, cloud cover, sun radiation, and the sun's position.

  \item \emph{Weather condition}: Overall state of the weather: sun, light clouds, partly cloudy, cloudy, fog, rain, snow, sleet, thunder.
  \item \emph{Weather state}: Summarizes all weather phenomena for a certain time. 
  \item \emph{Weather report}: Summarizes all data aquired at a certain time about the current weather or the weather some time in the future. Exactly one weather state is linked to each weather report.
  \item \emph{Weather source}: Source where the data belonging to a weather report has been obtained from (either a internet weather service or a local weather sensor).
\end{itemize}

\vspace{.2cm}

\textbf{Sources of knowledge}: % TODO reference more literature (e.g. for types of twilight)
\begin{itemize}
  \item Weather data that is available from internet services.
  \item Weather data that is available from local weather sensors.
  \item Existing ontologies that cover the domain of weather data.
  \item Information about weather in general, e.g. the Glossary of Meteorology by the American Meteorological Society. % TODO citation, more literature
\end{itemize}

\textbf{Competency questions}:
\begin{itemize}
  \item What is the current weather situation?
  \item What will the weather situation be in one hour, in two hours, …, in 24 hours?
  \item What is the current temperature, humidity, wind speed, …?
  \item What will be the temperature, humidity, wind speed, … in one hour, in two hours, …, in 24 hours?
  \item What will be the minimum temperature, humidity, … over the next 24 hours? What about maximum values?
  \item Will the weather change? Will the temperature, humidity, … rise or fall?
  \item Does it rain? Will it rain in the next hours? Will it rain today?
  \item Will there be sunshine today? 
  \item Do we need to irrigate the garden?
  \item Will there be severe weather?
%  \item Will temperature drop/stay below $0°C$? % FIXME
  \item When can we open windows and when do we have to keep them shut?
  \item When do we need sun protection?
  \item When will it outside be colder than inside the house? When will it be warmer?
\end{itemize}

\end{mdframed}

\vspace{.5cm}

In the following sections, the \thinkhomeweather ontology is built in a way to meet all above requirements. Section \ref{ch:ontology_evaluation} evaluates if the ontology that has been developed fits the specification and what the ontology's shortcomings are.

\section{Knowledge Aquisition}

The step of knowledge aquisition was already described in the previous chapters \ref{ch:existing_work} and \ref{ch:weather_data}. These chapters disussed in detail:

\begin{itemize}
  \item Which weather data are relevant for \thinkhome?
  \item Which weather data are available from sensors and internet services? How can this data be acquired?
  \item Which data do not have any use for \thinkhomeweather due to being too complicated or because they cannot be processed in an ontology in a useful way?
  \item What knowledge about weather in general is required to build an appropriate ontology? % TODO check if this is available in the previous chapters
\end{itemize}

\section{Conceptualization}

In the conceptualization step, the domain knowledge is structured in a conceptial model that describes the problem and its solution in terms of the domain vocabulary identified in the specification step. At first, a complete Glossary of Terms is built that covers all concepts, instances, attributes and binary relations.

\subsection{Glossary of Terms}

% TODO mention: in alphabetical order
% TODO repeat header on every page?
% TODO references between glossary items; e.g. from 'civil twilight' to 'twilight'
% TODO add sub-concepts for weather state
% TODO add type column
% TODO add binary relations
\begin{longtable}{|p{0.25\textwidth}|p{0.7\textwidth}|}
  \hline
  \textbf{Name} & \textbf{Description} \\
  \hline\hline
  Above room temperature & \\
  \hline
  Altitude & \\
  \hline
  Astronomical twilight & \\
  \hline
  Atmospheric pressure & \\
  \hline
  Below room temperature & \\
  \hline
  Calm & \\
  \hline
  Civil twilight & \\
  \hline
  Clear sky & \\
  \hline
  Cloud & \\
  \hline
  Cloud cover & \\
  \hline
  Cold & \\
  \hline
  Condition weather state & \\
  \hline
  Current weather state & \\
  \hline
  Current weather state from sensor & \\
  \hline
  Current weather state from service & \\
  \hline
  Day & \\
  \hline
  Dew point & \\
  \hline
  Directional wind & \\
  \hline
  Dry & \\
  \hline
  East wind & \\
  \hline
  End time & \\
  \hline
  Extremely heavy rain & \\
  \hline
  Fog & \\
  \hline
  Forecast weather state & \\
  \hline
  Frost & \\
  \hline
  Heat & \\
  \hline
  Heavy rain & \\
  \hline
  High pressure & \\
  \hline
  High radiation & \\
  \hline
  Humidity & \\
  \hline
  Hurricane & \\
  \hline
  Latitude & \\
  \hline
  Light clouds & \\
  \hline
  Light rain & \\
  \hline
  Light wind & \\
  \hline
  Location & \\
  \hline
  Longitude & \\
  \hline
  Long range weather report & \\
  \hline
  Low pressure & \\
  \hline
  Low radiation & \\
  \hline
  Medium radiation & \\
  \hline
  Medium rain & \\
  \hline
  Medium range weather report & \\
  \hline
  Moist & \\
  \hline
  Mostly cloudy & \\
  \hline
  Nautical twilight & \\
  \hline
  Night & \\
  \hline
  No radiation & \\
  \hline
  No rain & \\
  \hline
  Normal humidity & \\
  \hline
  Normal pressure & \\
  \hline
  North wind & \\
  \hline
  Observation time & \\
  \hline
  Overcast & \\
  \hline
  Partly cloudy & \\
  \hline
  Precipitation & \\
  \hline
  Priority & \\
  \hline
  Rain & \\
  \hline
  Room temperature & \\
  \hline
  Sensor source & \\
  \hline
  Service source & \\
  \hline
  Short range weather report & \\
  \hline
  Sleet & \\
  \hline
  Snow & \\
  \hline
  Solar twilight & \\
  \hline
  South wind & \\
  \hline
  Start time & \\
  \hline
  Storm & \\
  \hline
  Strong wind & \\
  \hline
  Sun & \\
  \hline
  Sun from east & \\
  \hline
  Sun from north & \\
  \hline
  Sun from south & \\
  \hline
  Sun from west & \\
  \hline
  Sun position & \\
  \hline
  Sun radiation & \\
  \hline
  Temperature & \\
  \hline
  Thunder & \\
  \hline
  Timed weather report & \\
  \hline
  Tropical storm rain & \\
  \hline
  Unknown cloud cover & \\
  \hline
  Very dry & \\
  \hline
  Very high pressure & \\
  \hline
  Very high radiation & \\
  \hline
  Very low pressure & \\
  \hline
  Very moist & \\
  \hline
  Weather condition & \\
  \hline
  Weather phenomenon & \\
  \hline
  Weather report & \\
  \hline
  Weather report from sensor & \\
  \hline
  Weather report from service & \\
  \hline
  Weather report source & \\
  \hline
  Weather state & \\
  \hline
  West wind & \\
  \hline
  Wind & \\
  \hline
\end{longtable}

\subsection{Concept-classification trees}
\label{sec:concept_classification_trees}

All concepts from the Glossary of Terms can be organized into five categories. Each of these categories has one root concept. These are \emph{Weather condition}, \emph{Weather phenomenon}, \emph{Weather report}, \emph{Weather source} and \emph{Weather state}. All other concepts are sub-concepts of one of these five root concepts. The trees resulting from the structure are presented below.

\subsubsection{Weather condition}

A \emph{Weather condition} represents a simple verbal description of the weather state at a certain time. As it does not have any sub-concepts, the diagram showing this classification tree looks rather simple:

% TODO

\subsubsection{Weather phenomenon}

A \emph{Weather phenomenon} represents a certain weather element, e.g. temperature or humidity. Every specific weather element is a sub-concept of \emph{Weather phenomenon}. Hence, this tree evolves:

% TODO
Every of these sub-concepts again has sub-concepts for different states of the weather elements. For the sake of clarity, these concepts are not shown in the diagram above. See the diagrams below for details.

The only sub-concept of \emph{Weather phenomenon} that does not have any sub-concepts is \emph{Dew point}. The diagram showing a tree that consists of only one node for \emph{Dew point} is omitted.

\paragraph{Atmospheric pressure}
% TODO

\paragraph{Cloud cover}
% TODO

\paragraph{Humidity}
% TODO

\paragraph{Precipitation}
% TODO

\paragraph{Temperature}
% TODO

\paragraph{Wind}
% TODO

\subsubsection{Weather report}

There are two main attributes for a \emph{Weather report}: Its \emph{Start date} and its \emph{Source}. A number of sub-concepts is defined in order to reflect different values of these two attributes:

% TODO

\subsubsection{Weather source}

A \emph{Weather source} can either be a \emph{Sensor source} or a \emph{Service source}:

% TODO

\subsubsection{Weather state}

% TODO reformulate this
A \emph{Weather state} collects all \emph{Weather phenomena} for a certain \emph{Weather report}. Because of this, sub-concepts of \emph{Weather state} can be introduced that reflect certain combination of related instances of \emph{Weather phenomenon}:

% TODO

\subsection{Binary relations diagram}
\label{sec:binary_relations_diagram}

The binary relations diagram shows all binary relations between concepts in the ontology in a clear manner:

% TODO 
% Weather state <-> Weather phenomenon: belongs to state/has weather phenomenon
% Weather report <-> Weather state: belongs to weather report/has weather state
% Weather state <-> Weather condition: has condition
% Weather report <-> Weather report: has next weather report/has previous weather report
% Weather report <-> Weather source: has weather source/is source of
% put all this one diagram

\subsection{Concept dictionaries}

A concept dictionary lists all concepts together with their names, instances, class attributes, instance attributes and relations. For the sake of clarity, this table is split up into several tables, one for each of the concept-classification trees in section \ref{sec:concept_classification_trees}.

\subsubsection{Weather condition}

\begin{longtable}{|p{0.167\textwidth}|p{0.167\textwidth}|p{0.167\textwidth}|p{0.167\textwidth}|p{0.167\textwidth}|}
  \hline
  \textbf{Name} & \textbf{Instances} & \textbf{Class\newline attributes} & \textbf{Instance\newline attributes} & \textbf{Relations} \\
  \hline\hline
  Weather condition & Cloud\newline Fog\newline Partly cloudy\newline Mostly cloudy\newline Rain\newline Sleet\newline Snow\newline Sun\newline Thunder & - & - & hasCondition \\
  \hline
\end{longtable}

\subsubsection{Weather phenomenon}

% TODO
\begin{longtable}{|p{0.167\textwidth}|p{0.167\textwidth}|p{0.167\textwidth}|p{0.167\textwidth}|p{0.167\textwidth}|}
  \hline
  \textbf{Name} & \textbf{Instances} & \textbf{Class\newline attributes} & \textbf{Instance\newline attributes} & \textbf{Relations} \\
  \hline\hline
  lala & - & - & - & - \\
  \hline
\end{longtable}

\subsubsection{Weather report}

% TODO
\begin{longtable}{|p{0.167\textwidth}|p{0.167\textwidth}|p{0.167\textwidth}|p{0.167\textwidth}|p{0.167\textwidth}|}
  \hline
  \textbf{Name} & \textbf{Instances} & \textbf{Class\newline attributes} & \textbf{Instance\newline attributes} & \textbf{Relations} \\
  \hline\hline
  lala & - & - & - & - \\
  \hline
\end{longtable}

\subsubsection{Weather state}

% TODO
\begin{longtable}{|p{0.167\textwidth}|p{0.167\textwidth}|p{0.167\textwidth}|p{0.167\textwidth}|p{0.167\textwidth}|}
  \hline
  \textbf{Name} & \textbf{Instances} & \textbf{Class\newline attributes} & \textbf{Instance\newline attributes} & \textbf{Relations} \\
  \hline\hline
  lala & - & - & - & - \\
  \hline
\end{longtable}

\subsubsection{Weather source}

% TODO
\begin{longtable}{|p{0.167\textwidth}|p{0.167\textwidth}|p{0.167\textwidth}|p{0.167\textwidth}|p{0.167\textwidth}|}
  \hline
  \textbf{Name} & \textbf{Instances} & \textbf{Class\newline attributes} & \textbf{Instance\newline attributes} & \textbf{Relations} \\
  \hline\hline
  lala & - & - & - & - \\
  \hline
\end{longtable}

\subsection{Binary relations table}

The binary relation table specifies all relations from section \ref{sec:binary_relations_diagram} in detail. This includes the relations' names, their source and target concepts, their maximum source cardinalities and their inverse relations, if any.

\begin{longtable}{|p{0.167\textwidth}|p{0.167\textwidth}|p{0.167\textwidth}|p{0.167\textwidth}|p{0.167\textwidth}|}
  \hline
  \textbf{Name} & \textbf{Source\newline concept} & \textbf{Target\newline concept} & \textbf{Maximum\newline source\newline cardinality} & \textbf{Inverse\newline relation} \\
  \hline\hline
  belongs to state & Weather phenomenon & Weather state & 1 & has weather phenomenon \\
  \hline
  belongs to report & Weather state & Weather report & 1 & has weather state \\
  \hline
  has condition & Weather state & Weather condition & N & - \\
  \hline
  has next weather report & Weather report & Weather report & 1 & has previous weather report \\
  \hline
  has previous weather report & Weather report & Weather report & 1 & has next weather report \\
  \hline
  has source & Weather report & Weather source & 1 & is source of \\
  \hline
  has weather phenomenon & Weather state & Weather phenomenon & N & belongs to state \\
  \hline
  has weather state & Weather report & Weather state & 1 & belongs to report \\
  \hline
  is source of & Weather source & Weather report & N & has source \\
  \hline
\end{longtable}

\subsection{Instance attribute tables}

% TODO description of what is going on here
% TODO
\begin{longtable}{|p{0.167\textwidth}|p{0.167\textwidth}|p{0.167\textwidth}|p{0.167\textwidth}|p{0.167\textwidth}|}
  \hline
  \textbf{Attribute name} & \textbf{Concept name} & \textbf{Value type} & \textbf{Value range} & \textbf{Cardinality} \\
  \hline\hline
  lalala & lala & lala & lala & lala \\
  \hline
\end{longtable}

\subsection{Class attribute tables}

% TODO description of what is going on here
% TODO 
\begin{longtable}{|p{0.167\textwidth}|p{0.167\textwidth}|p{0.167\textwidth}|p{0.167\textwidth}|p{0.167\textwidth}|}
  \hline
  \textbf{Attribute name} & \textbf{Defined concept} & \textbf{Value type} & \textbf{Cardinality} & \textbf{Values} \\
  \hline\hline
  lalala & lala & lala & lala & lala \\
  \hline
\end{longtable}

\subsection{Logical axioms tables}

% TODO sub-concepts of weather state

\subsection{Formula tables}

% TODO sub-concepts of weather state

\subsection{Instance tables}

% TODO description of what is going on here
% TODO
\begin{longtable}{|p{0.2\textwidth}|p{0.2\textwidth}|p{0.2\textwidth}|p{0.2\textwidth}|}
  \hline
  \textbf{Instance name} & \textbf{Concept name} & \textbf{Attribute} & \textbf{Values} \\
  \hline\hline
  Cloud & Weather condition & - & - \\
  \hline
  Fog & Weather condition & - & - \\
  \hline
  Partly cloudy & Weather condition & - & - \\
  \hline
  Mostly cloudy & Weather condition & - & - \\
  \hline
  Rain & Weather condition & - & - \\
  \hline
  Sleet & Weather condition & - & - \\
  \hline
  Snow & Weather condition & - & - \\
  \hline
  Sun & Weather condition & - & - \\
  \hline
  Thunder & Weather condition & - & - \\
  \hline
\end{longtable}

\section{Integration}

% TODO mention other ontologies: why not SWEET; ontology for units, owl-time, wgs84
% TODO mention meta-ontology

\section{Implementation}

% TODO cover: integration of other ontologies; e.g. how the unit ontology being used works
% TODO document completed ontology thoroughly
% TODO diagrams for parts of the ontology

\section{Evaluation}
\label{ch:ontology_evaluation}

% TODO compare existing ontology with specification and competence questions

\section{Fetching data}

% TODO describe fetching weather data from yr.no
% TODO describe algorithms for calculation of the position of the sun
% TODO perhaps split into two categories (one for weather data and one for the sun's position according to one of the two available algorithms, see SunPositionCalculator's source code

\subsection{Weather data}

\subsection{Position of the sun}
