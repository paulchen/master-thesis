% TODO check if the things mentioned here are properly covered in the chapters referred
The previous chapters covered all aspects that have to be analyzed in order to build a completely new ontology from scratch: Chapter \ref{ch:weather_data} discussed all details about weather data necessary for the \thinkhome weather ontology, what data will be used and where to obain it. Chapter \ref{ch:existing_work} gave an overview over existing work regarding ontologies in the domain of weather data. As none of the existing ontologies fit the needs of an ontology for \thinkhome, chapter  \ref{ch:development_approaches} analyzed some of the most popular approaches for building ontologies from scratch. Among those \methontology was identified as the best suitable approach. % TODO citation for methontology?

Based on these findings, this chapter describes the process of building the \thinkhome weather ontology in detail.

\section{Specification}

Specification, the first step in the ontology development process defined by \methontology, aims at creating an ontology specification document using natural language. That document must include at least the intended purpose of the ontology, the level of 
formality and the ontology's scope.

% TODO ensure there are no problematic page breaks inside this box (e.g. between the headline 'Scope:' and the first bullet point)
\begin{mdframed}
\setlength{\parindent}{0pt}
\MakeUppercase{\textbf{Ontology requirements specification document}}

\vspace{.5cm}

\textbf{Domain}: Weather data

\vspace{.2cm}

\textbf{Date}: 2012 % TODO

\vspace{.2cm}

\textbf{Purpose}: The ontology shall cover data about weather phenomena occurring at a certain location somewhere on Earth between the present and 24 hours in the future. Weather data will be aquired from both internet services as well as from weather sensors mounted to a building. This weather data will enable the \textit{ThinkHome} system to make decisions based on current and future weather conditions.

\vspace{.2cm}

\textbf{Level of formality}: semi-informal % TODO citations for uschold & gruninger, 1996; uschuld & king (enterprise ontology)

\vspace{.2cm}

\textbf{Scope}:

\begin{itemize}
  \item \emph{Weather phenomenon}: Represents a certain weather element. Relevant weather elements are temperature, humidity, dew point, wind speed and direction, precipitation intensity and probability, atmospheric pressure, cloud cover, sun radiation, and the sun's position.

  \item \emph{Weather condition}: Overall state of the weather: sun, light clouds, partly cloudy, cloudy, fog, rain, snow, sleet, thunder.
  \item \emph{Weather state}: Summarizes all weather phenomena for a certain time. 
  \item \emph{Weather report}: Summarizes all data aquired at a certain time about the current weather or the weather some time in the future. Exactly one weather state is linked to each weather report.
  \item \emph{Weather source}: Source where the data belonging to a weather report has been obtained from (either a internet weather service or a local weather sensor).
\end{itemize}

\vspace{.2cm}

\textbf{Sources of knowledge}: % TODO reference more literature (e.g. for types of twilight)
\begin{itemize}
  \item Weather data that is available from internet services.
  \item Weather data that is available from local weather sensors.
  \item Existing ontologies that cover the domain of weather data.
  \item Information about weather in general, e.g. the Glossary of Meteorology by the American Meteorological Society. % TODO citation, more literature
\end{itemize}

\textbf{Competency questions}:
\begin{itemize}
  \item What is the current weather situation?
  \item What will the weather situation be in one hour, in two hours, …, in 24 hours?
  \item What is the current temperature, humidity, wind speed, …?
  \item What will be the temperature, humidity, wind speed, … in one hour, in two hours, …, in 24 hours?
  \item What will be the minimum temperature, humidity, … over the next 24 hours? What about maximum values?
  \item Will the weather change? Will the temperature, humidity, … rise or fall?
  \item Does it rain? Will it rain in the next hours? Will it rain today?
  \item Will there be sunshine today? 
  \item Do we need to irrigate the garden?
  \item Will there be severe weather?
  \item Will temperature drop/stay below $0°C$?
  \item When can we open windows and when do we have to keep them shut?
  \item When do we need sun protection?
  \item When will it outside be colder than inside the house? When will it be warmer?
\end{itemize}

\end{mdframed}

\vspace{.5cm}

In the following sections, the \thinkhomeweather ontology is built in a way to meet all above requirements. Section \ref{ch:ontology_evaluation} evaluates if the ontology that has been developed fits the specification and what the ontology's shortcomings are.

\section{Knowledge Aquisition}

% TODO mostly refer to previous chapters as knowledge aquisition is already completed

\section{Conceptualization}

% TODO check if these steps are correct (see referenced papers

% TODO glossary of terms
% TODO concept classification trees
% TODO data dictionaries
% TODO tables of instance attributes
% TODO tables of class attributes
% TODO tables of constants
% TODO tables of instances
% TODO attributes classification trees
% TODO verbs diagrams
% TODO verbs dictionary
% TODO table of conditions
% TODO table of formulas
% TODO table of rules

\section{Integration}

% TODO mention other ontologies: why not SWEET; ontology for units, owl-time, wgs84
% TODO mention meta-ontology

\section{Implementation}

% TODO cover: integration of other ontologies; e.g. how the unit ontology being used works
% TODO document completed ontology thoroughly
% TODO diagrams for parts of the ontology

\section{Evaluation}
\label{ch:ontology_evaluation}

% TODO compare existing ontology with specification and competence questions

\section{Fetching data}

% TODO describe fetching weather data from yr.no
% TODO describe algorithms for calculation of the position of the sun
% TODO perhaps split into two categories (one for weather data and one for the sun's position according to one of the two available algorithms, see SunPositionCalculator's source code

\subsection{Weather data}

\subsection{Position of the sun}
