% TODO aim of this chapter: find out which data is available, where can they be obtained, which values are useful

% TODO some section containing thoughts about which data is required for the ThinkHome ontology, e.g. in the chapter about the ontology itself?

Based on the insights acquired from existing work in chapter \ref{ch:existing_work}, this chapter aims at compiling a set of weather properties which are either necessary for providing useful data to the \thinkhome system or which would add benefit to the data provided. Furthermore, possible sources will are evaluated with respect to their usableness in the given context.

\section{Weather information}

% TODO reference competence questions
In order to identify weather properties required for \thinkhome, one must specify the requirements that have to be met. When designing an ontology, this is often done by asking numerous \emph{competence questions}.

These competence questions that shall be answered by \thinkhome's weather ontology:

% TODO answer: where do 24 hours come from? discuss questions further? how were these questions built?
\begin{itemize}
  \item What is the current weather situation?
  \item What will the weather situation be in one hour, in two hours, …, in 24 hours?
  \item What is the current temperature, humidity, wind speed, …?
  \item What will be the temperature, humidity, wind speed, … in one hour, in two hours, …, in 24 hours?
  \item What will be the minimum temperature, humidity, … over the next 24 hours? What about maximum values?
  \item Will the weather change? Will the temperature, humidity, … rise or fall?
  \item Does it rain? Will it rain in the next hours? Will it rain today?
  \item Will there be sunshine today? 
  \item Do we need to irrigate the garden?
  \item Will there be severe weather?
  \item Will temperature drop/stay below $0^\circ C$?
  \item When can we open windows and when do we have to keep them shut?
  \item When do we need sun protection?
  \item When will it outside be colder than inside the house? When will it be warmer?
\end{itemize}

These competence questions will again be used in section \label{sec:ontology_specification} for creating the \emph{Ontology requirements specification document}.

% TODO references
% TODO measurement readings
These questions can be answered when knowing the (predicted) state of the weather for particular points of time. The state of the weather is given by measurement readings of certain \emph{weather elements}. These weather elements are temperature, relative and absolute humidity, dew point temperature, wind speed, wind direction, precipitation, cloud coverage etc.

A set of measurement readings for one or more points of time is called \emph{weather data} from now on.

Sources for weather data are weather sensors and weather forecasts. As no weather forecasting is performed within \thinkhome, forecast data will be obtained from weather services via Internet. Data about the current weather state can be obtained from both weather sensors and Internet services; the only source for data about future weather states are Internet services.

However, there there are weather elements no sensor can provide data about, and there are weather elements that are not available via any of the Internet services that have been analyzed. Chapter \ref{sec:weather_sensors} covers everything that can be obtained from weather sensors, and chapter \ref{sec:weather_services} discusses weather data that can be fetched from Internet services. Based on the findings within these two chapters, \ref{sec:weather_conclusion} presents a set of weather elements that are incorporated into the ontology that is developed in chapter \ref{ch:thinkhomeweather_ontology}.

\section{Sensor data}
\label{sec:weather_sensors}

\subsection{KNX weather stations}

\section{Service data}
\label{sec:weather_services}

% TODO blah blah

\subsection{Available Internet services}

There is a tremendous amount of services providing weather data over the Internet. However, only a small number of them is suitable for use in a context as in \thinkhome. The services differ regarding the way data is provided, the data format, the area being covered, the terms of use etc.

For evaluation of Internet weather services, the following aspects are examined:

\begin{itemize}
  \item \textbf{Coverage area:} Which part of the world is covered by the weather service?\\
  In order to keep things simple, a service covering a larger part of the world is preferred over a service covering a smaller part.
  \item \textbf{Data format:} In what format is weather data being delivered? Can it be easily parsed and processed?\\
  A data format that is easier to handle on the client-side is preferred over a data format that requires more complicated handling.
  \item \textbf{Data access:} How does the access to weather data work? How is data requested and how is the answer being received?\\
  The less complicated a request is, the better the service suits the requirements.
  \item \textbf{Access restrictions, terms of use:} Is the service available freely or is the access restricted? Are credentials (e.g. a username and a password or an access key) required for access? If yes, can the credentials be obtained in a simple way or does that entail a complicated procedure? Are there any access fees for academic or commercial use?\\
  A service being less restricted is preferred over a service coming with more restrictions.
  \item \textbf{Documentation:} Is there any documentation for this service? Does it cover all aspects or are there some features that are undocumented?\\
  Of course, a service without documentation is unsuitable. A better documented service is always preferred over a service that is less documented.
  \item \textbf{Stability:} Can one expect the service to remain unchanged over a reasonable amount of time (e.g. several years)? If there will be future changes, will they be announced? How long will they be announced in advance?\\
  A stable service is preferred over an unstable one. A better handling of changes is preferred over a worse handling of changes.
  \item \textbf{Weather elements:} Which weather elements (e.g. temperature, relative humidity, dew point etc.) are covered by the service?\\
  A weather service covering a wider range of weather elements is preferred over a service covering a smaller range.
  \item \textbf{Timeframe:} Are forecasts available? If yes, how detailled are these forecasts? How far into the future are forecasts available? What is the interval between two forecasts?\\
  Forecasts for at least 48 hours are mandatory. Over the next 48 hours, there should be at least six forecasts with an interval of at most eight hours between two consecutive forecasts. A weather service covering a longer period than 48 hours into the future and/or more than six forecasts within the next 48 hours is preferred.
  \item \textbf{Updates:} How often is the weather data being updated?\\
  The data should be updated at least every six hours. A service having a shorter update interval is preferred over a service having a longer interval.
\end{itemize}

In the following sections, some of the most popular Internet weather services are evaluated. In addition to the aspects listed above, some general information is provided (operator, web page and ...). % TODO

\paragraph{DWD}

\begin{itemize}
  \item \textbf{Operator:}
  \item \textbf{Web page:}
  \item \textbf{Coverage area:}
  \item \textbf{Data format:}
  \item \textbf{Data access:}
  \item \textbf{Access restrictions, terms of use:}
  \item \textbf{Documentation:}
  \item \textbf{Stability:}
  \item \textbf{Weather elements:}
  \item \textbf{Timeframe:}
  \item \textbf{Updates:}
\end{itemize}

\paragraph{Yahoo Weather}

\begin{itemize}
  \item \textbf{Operator:}
  \item \textbf{Web page:}
  \item \textbf{Coverage area:}
  \item \textbf{Data format:}
  \item \textbf{Data access:}
  \item \textbf{Access restrictions, terms of use:}
  \item \textbf{Documentation:}
  \item \textbf{Stability:}
  \item \textbf{Weather elements:}
  \item \textbf{Timeframe:}
  \item \textbf{Updates:}
\end{itemize}

\paragraph{World Weather Online}

\begin{itemize}
  \item \textbf{Operator:}
  \item \textbf{Web page:}
  \item \textbf{Coverage area:}
  \item \textbf{Data format:}
  \item \textbf{Data access:}
  \item \textbf{Access restrictions, terms of use:}
  \item \textbf{Documentation:}
  \item \textbf{Stability:}
  \item \textbf{Weather elements:}
  \item \textbf{Timeframe:}
  \item \textbf{Updates:}
\end{itemize}

\paragraph{Google Weather Feed}

\begin{itemize}
  \item \textbf{Operator:}
  \item \textbf{Web page:}
  \item \textbf{Coverage area:}
  \item \textbf{Data format:}
  \item \textbf{Data access:}
  \item \textbf{Access restrictions, terms of use:}
  \item \textbf{Documentation:}
  \item \textbf{Stability:}
  \item \textbf{Weather elements:}
  \item \textbf{Timeframe:}
  \item \textbf{Updates:}
\end{itemize}

\paragraph{yr.no}

\begin{itemize}
  \item \textbf{Operator:}
  \item \textbf{Web page:}
  \item \textbf{Coverage area:}
  \item \textbf{Data format:}
  \item \textbf{Data access:}
  \item \textbf{Access restrictions, terms of use:}
  \item \textbf{Documentation:}
  \item \textbf{Stability:}
  \item \textbf{Weather elements:}
  \item \textbf{Timeframe:}
  \item \textbf{Updates:}
\end{itemize}

\paragraph{METAR}

\begin{itemize}
  \item \textbf{Operator:}
  \item \textbf{Web page:}
  \item \textbf{Coverage area:}
  \item \textbf{Data format:}
  \item \textbf{Data access:}
  \item \textbf{Access restrictions, terms of use:}
  \item \textbf{Documentation:}
  \item \textbf{Stability:}
  \item \textbf{Weather elements:}
  \item \textbf{Timeframe:}
  \item \textbf{Updates:}
\end{itemize}

\paragraph{National Weather Service}

\begin{itemize}
  \item \textbf{Operator:}
  \item \textbf{Web page:}
  \item \textbf{Coverage area:}
  \item \textbf{Data format:}
  \item \textbf{Data access:}
  \item \textbf{Access restrictions, terms of use:}
  \item \textbf{Documentation:}
  \item \textbf{Stability:}
  \item \textbf{Weather elements:}
  \item \textbf{Timeframe:}
  \item \textbf{Updates:}
\end{itemize}

\paragraph{Weather Underground}

\begin{itemize}
  \item \textbf{Operator:}
  \item \textbf{Web page:}
  \item \textbf{Coverage area:}
  \item \textbf{Data format:}
  \item \textbf{Data access:}
  \item \textbf{Access restrictions, terms of use:}
  \item \textbf{Documentation:}
  \item \textbf{Stability:}
  \item \textbf{Weather elements:}
  \item \textbf{Timeframe:}
  \item \textbf{Updates:}
\end{itemize}

\paragraph{Weather.com}

\begin{itemize}
  \item \textbf{Operator:}
  \item \textbf{Web page:}
  \item \textbf{Coverage area:}
  \item \textbf{Data format:}
  \item \textbf{Data access:}
  \item \textbf{Access restrictions, terms of use:}
  \item \textbf{Documentation:}
  \item \textbf{Stability:}
  \item \textbf{Weather elements:}
  \item \textbf{Timeframe:}
  \item \textbf{Updates:}
\end{itemize}

\section{Conclusion}
\label{sec:weather_conclusion}

Among the weather services that have been evaluated, \emph{yr.no} suits \thinkhome's requirements best. Therefore, it is used for developing a program that obtains weather data for a certain location and feeds it into the \thinkhomeweather ontology that is developed in section \ref{sec:ontology_specification}. See section \ref{subsec:weather_data_yr_no} for details on how to query \emph{yr.no}'s weather API.

However, the weather service shall be replaceable. The ontology is designed in a way that makes switching to another weather service simple. If \emph{yr.no} is to be replaced by another weather service, the ontology remains unchanged, but a new program for importing weather data into the ontology must be developed.

The ontology uses these weather elements which are available from most Internet weather services and many weather sensors:
\begin{itemize}
  \item Temperature,
  \item Relative humidity,
  \item Dew point,
  \item Cloud cover,
  \item Chance and intensity of precipitation,
  \item Speed and direction of window,
  \item Atmospheric pressure, and
  \item Solar radiation
\end{itemize}

Additionally, the ontology also supports the position of the sun. However, it is not necessary to obtain the sun's position from an internet service as there are a few algorithms available for calculating the sun's position given date, time and the geographical position (latitude and longitude); see \ref{subsec:sun_position} for details.

