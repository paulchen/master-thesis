Based on the insights acquired from existing work in chapter \ref{ch:existing_work}, this chapter aims at compiling a set of weather properties which are either necessary for providing useful data to the \thinkhome system or which would add benefit to the data provided. Furthermore, possible sources will are evaluated with respect to their usableness in the given context.

\section{Weather information}
\label{sec:weather_information}

In order to identify weather properties required for \thinkhome, one must specify the requirements that have to be met. When designing an ontology, this is often done by asking numerous \emph{competency questions}.

These competency questions that shall be answered by \thinkhome's weather ontology:

% TODO answer: where do 24 hours come from? discuss questions further? how were these questions built?
\begin{itemize}
  \item What is the current weather situation?
  \item What will the weather situation be in one hour, in two hours, …, in 24 hours?
  \item What is the current temperature, humidity, wind speed, …?
  \item What will be the temperature, humidity, wind speed, … in one hour, in two hours, …, in 24 hours?
  \item What will be the minimum temperature, humidity, … over the next 24 hours? What about maximum values?
  \item Will the weather change? Will the temperature, humidity, … rise or fall?
  \item Does it rain? Will it rain in the next hours? Will it rain today?
  \item Will there be sunshine today? 
  \item Do we need to irrigate the garden?
  \item Will there be severe weather?
  \item Will temperature drop/stay below $0^\circ C$?
  \item When can we open windows and when do we have to keep them shut?
  \item When do we need sun protection?
  \item When will it outside be colder than inside the house? When will it be warmer?
\end{itemize}

These competency questions will again be used in section \label{sec:ontology_specification} for creating the \emph{Ontology requirements specification document}.

% TODO references
% TODO replace "measurement readings" by other term?
These questions can be answered when knowing the (predicted) state of the weather for particular points of time. The state of the weather is given by measurement readings of certain \emph{weather elements}. These weather elements are temperature, relative and absolute humidity, dew point temperature, wind speed, wind direction, precipitation, cloud coverage etc.

A set of measurement readings for one or more points of time is called \emph{weather data} from now on.

Apart from weather elements, Internet weather services often provide some information that will be called \emph{overall weather condition} from now on. Generally speaking, the overall weather condition is what someone tells you if you ask her about the current weather situation. Examples for the overall weather condition are ``Sun'', ``Rain'', ``Fog'' etc. Some weather conditions can be split up into several conditions, e.g. ``It is overcast and raining'' into ``Overcast'' and ``Rain''.

Sources for weather data are weather sensors and weather forecasts. As no weather forecasting is performed within \thinkhome, forecast data will be obtained from weather services via Internet. Data about the current weather state can be obtained from both weather sensors and Internet services; the only source for data about future weather states are Internet services.

However, there there are weather elements no sensor can provide data about, and there are weather elements that are not available via any of the Internet services that have been analyzed. Chapter \ref{sec:weather_sensors} covers everything that can be obtained from weather sensors, and chapter \ref{sec:weather_services} discusses weather data that can be fetched from Internet services. Based on the findings within these two chapters, \ref{sec:weather_conclusion} presents a set of weather elements that are incorporated into the ontology that is developed in chapter \ref{ch:thinkhomeweather_ontology}.

\section{Sensor data}
\label{sec:weather_sensors}

\subsection{KNX weather stations}

\section{Service data}
\label{sec:weather_services}

% TODO blah blah

\subsection{Available Internet services}
\label{sec:internet_services}

There is a tremendous amount of services providing weather data over the Internet. However, only a small number of them is suitable for use in a context as in \thinkhome. The services differ regarding the way data is provided, the data format, the area being covered, the terms of use etc.

Handling of Internet weather services will be implemented in Java, hence an important question about a certain weather service is whether in can easily be accessed from within a Java program.

For evaluation of Internet weather services, the following aspects are examined:

\begin{itemize}
  \item \textbf{Coverage area:} Which part of the world is covered by the weather service?\\
  In order to keep things simple, a service covering a larger part of the world is preferred over a service covering a smaller part.
  \item \textbf{Data format:} In what format is weather data being delivered? Can it be easily parsed and processed?\\
  A data format that is easier to handle on the client-side is preferred over a data format that requires more complicated handling. XML can be handled natively within many programming languages, JSON may require an additional library (e.g. in Java). % TODO references
  \item \textbf{Data access:} How does the access to weather data work? How is data requested and how is the answer being received?\\
  The less complicated a request is, the better the service suits the requirements. Requests in HTTP are preferred due to the simplicity of performing requests on the client side in Java using the Apache HttpComponents project\footnote{\href{http://hc.apache.org/}{http://hc.apache.org/}}.
  \item \textbf{Access restrictions, terms of use:} Is the service available freely or is the access restricted? Are credentials (e.g. a username and a password or an access key) required for access? If yes, can the credentials be obtained in a simple way or does that entail a complicated procedure? Are there any access fees for academic or commercial use?\\
  A service being less restricted is preferred over a service coming with more restrictions.
  \item \textbf{Documentation:} Is there any documentation for this service? Does it cover all aspects or are there some features that are undocumented?\\
  Of course, a service without documentation is unsuitable. A better documented service is always preferred over a service that is less documented.
  \item \textbf{Stability:} Can one expect the service to remain unchanged over a reasonable amount of time (e.g. several years)? If there will be future changes, will they be announced? How long will they be announced in advance?\\
  A stable service is preferred over an unstable one. A better handling of changes is preferred over a worse handling of changes.
  \item \textbf{Weather elements:} Which weather elements (e.g. temperature, relative humidity, dew point etc.) are covered by the service?\\
  A weather service covering a wider range of weather elements is preferred over a service covering a smaller range.
  \item \textbf{Timeframe:} Are forecasts available? If yes, how detailled are these forecasts? How far into the future are forecasts available? What is the interval between two forecasts?\\
  Forecasts for at least 48 hours are mandatory. Over the next 48 hours, there should be at least six forecasts with an interval of at most eight hours between two consecutive forecasts. A weather service covering a longer period than 48 hours into the future and/or more than six forecasts within the next 48 hours is preferred.
  \item \textbf{Weather updates:} How often is the weather data being updated?\\
  The data should be updated at least every six hours. A service having a shorter update interval is preferred over a service having a longer interval.
\end{itemize}

In the following sections, some of the most popular Internet weather services are evaluated. In addition to the aspects listed above, some general information is provided (operator and web page). These sections aim at determining which Internet weather service best fits the given requirements. Furthermore, based on which weather elements are provided by various services, a set of weather elements is determined that will be used in the ontology. Including weather elements that are not available from any weather services are useless.

% TODO where do the informations in this sections come from?

\paragraph{DWD}

\begin{itemize}
  \item \textbf{Operator:} \emph{Deutscher Wetterdienst} (\emph{DWD}, ``German weather service'')
  \item \textbf{Web page:} \href{http://www.dwd.de/}{http://www.dwd.de/}
  \item \textbf{Coverage area:} Current weather data is available worldwide, forecasts are available within Germany and only for large cities outside Germany.
  \item \textbf{Data format:} Current weather data is available in SYNOP format (Surface Synoptic Observations) \footnote{\href{http://weather.unisys.com/wxp/Appendices/Formats/SYNOP.html}{http://weather.unisys.com/wxp/Appendices/Formats/SYNOP.html}}. A Java parser may be time-consuming to implement. Forecasts are available as weather maps, tables and text (not machine-readable).
  \item \textbf{Data access:} Via FTP (supported in Java by the Apache HttpComponents project\footnote{\href{http://hc.apache.org/}{http://hc.apache.org/}}).
  \item \textbf{Access restrictions, terms of use:} An account is mandatory for access. Account creation is a matter of minutes and new accounts do not need to be approved by DWD. The use of DWD is free for use within projects like \thinkhome.
  \item \textbf{Documentation:} There is extensive documentation 
  \item \textbf{Stability:} Every few weeks there are a few changes to the data format. However, data required by \thinkhome has never been changed during the last six months. Changes are announced via email at least one week in advance.
  \item \textbf{Weather elements:} Temperature, relative humidity, wind speed and direction, precipitation, atmospheric pressure, cloud coverage. Forecasts only include temperature and the overall weather condition.
  \item \textbf{Timeframe:} Forecasts are available for 72 hours into the future.
  \item \textbf{Weather updates:} Weather data is updated every few hours.
\end{itemize}

\paragraph{Yahoo! Weather}

\begin{itemize}
  \item \textbf{Operator:} \emph{Yahoo! Inc.}
  \item \textbf{Web page:} \href{http://developer.yahoo.com/weather/}{http://developer.yahoo.com/weather/}
  \item \textbf{Coverage area:} Worldwide coverage.
  \item \textbf{Data format:} RSS feed.
  \item \textbf{Data access:} Via HTTP.
  \item \textbf{Access restrictions, terms of use:} Free for use for individuals and non-profit organizations, providing attribution required. No statement about commercial use anywhere in the documentation.
  \item \textbf{Documentation:} Full documentation is available.
  \item \textbf{Stability:} It is unknown whether this API is subject to change or not.
  \item \textbf{Weather elements:} Data about current weather state includes temperature, wind chill, speed and direction, relative humidity, visibility, atmospheric pressure, sunrise and sunset times. Forecasts only include temperature and an overall weather condition.
  \item \textbf{Timeframe:} Forecasts are available for 48 hours, but only one forecast per day.
  \item \textbf{Weather updates:} Weather data is updated every few hours.
\end{itemize}

\paragraph{World Weather Online}

\begin{itemize}
  \item \textbf{Operator:} World Weather Online
  \item \textbf{Web page:} \href{http://www.worldweatheronline.com}{http://www.worldweatheronline.com}
  \item \textbf{Coverage area:} Worldwide coverage.
  \item \textbf{Data format:} Data is available in XML, JSON, JSON-P and CSV formats.
  \item \textbf{Data access:} Via HTTP.
  \item \textbf{Access restrictions, terms of use:} An account must be created for accessing both the Free API and the Premium API. The Free API is free to use for personal and commercial use, credits must be given to World Weather Online. Costs for the Premium API start at 20 US dollars per month.
  \item \textbf{Documentation:} Full documentation is available.
  \item \textbf{Stability:} It is unknown whether the APIs are subject to change or not.
  \item \textbf{Weather elements:} Free API offers temperature, wind speed and direction, relative humidity and precipitation intensity. Premium API additionally offers dew point, chance of precipitation and atmospheric pressure.
  \item \textbf{Timeframe:} The Free API offers daily forecasts over a range of five days, the Premium API comes with % TODO
  \item \textbf{Weather updates:} Weather data is updated every 3 to 4 hours.
\end{itemize}

\paragraph{Google Weather Feed}

\begin{itemize}
  \item \textbf{Operator:} Google Inc.
  \item \textbf{Web page:} As Google's weather feed is not a publicly advertised API, there is no web page for it.
  \item \textbf{Coverage area:} Worldwide coverage.
  \item \textbf{Data format:} XML feed.
  \item \textbf{Data access:} Via HTTP.
  \item \textbf{Access restrictions, terms of use:} There is no explicit licence for this API as it is an inofficial interface. Hence, usage of this API within any project is probably not allowed at all.
  \item \textbf{Documentation:} No official documentation available. The inofficial documentation being available from third-party websites is sufficient for properly using Google's weather feed.
  \item \textbf{Stability:} It is unknown whether the API is subject to change or not. It must be expected that it may be changed without prior notice.
  \item \textbf{Weather elements:} Temperature, humidity (not included in forecasts) and the overall weather condition.
  \item \textbf{Timeframe:} Daily forecasts are available over the next 72 hours.
  \item \textbf{Weather updates:} Weather data is updated every few hours.
\end{itemize}

\paragraph{yr.no}

\begin{itemize}
  \item \textbf{Operator:} Meteorologisk institutt (Norwegian Meteorological Institute), Norsk rikskringkasting AS (NRK, Norwegian Broadcasting Corporation)
  \item \textbf{Web page:} \href{http://api.yr.no/}{http://api.yr.no/}
  \item \textbf{Coverage area:} Worldwide.
  \item \textbf{Data format:} XML.
  \item \textbf{Data access:} Via HTTP.
  \item \textbf{Access restrictions, terms of use:} Data is available under a Creative Commons licence (CC-BY 3.0\footnote{\href{http://creativecommons.org/licenses/by/3.0}{http://creativecommons.org/licenses/by/3.0}})
  \item \textbf{Documentation:} Full documentation is available.
  \item \textbf{Stability:} There are new releases of the API one or two times per year. The old API remains in operation a few months after a new version has been released. New releases are announced on the web page of the API.\\
  Additionally, there is a \emph{long time support} weather API that changes less frequent.
  \item \textbf{Weather elements:} Temperature, wind direction and speed, chance and intensity of precipitation, atmospheric pressure, relative humidity, cloud coverage, fog.
  \item \textbf{Timeframe:} Forecasts for nine days in advance, interval of six hours between two forecasts.
  \item \textbf{Weather updates:} Weather data is updated every few hours.
\end{itemize}

\paragraph{METAR}

% TODO references
\begin{itemize}
  \item \textbf{Operator:} Airports around the world.
  \item \textbf{Web page:} \href{http://www.aviationweather.gov/adds/metars/}{http://www.aviationweather.gov/adds/metars/}
  \item \textbf{Coverage area:} Worldwide coverage.
  \item \textbf{Data format:} The \emph{METAR} format is standardized by \emph{ICAO} (\emph{International Civil Aviation Organization}). It is not human-readable. Parsers for various programming languages are available, e.g. \emph{PyMETAR}\footnote{\href{http://www.schwarzvogel.de/software-pymetar.shtml}{http://www.schwarzvogel.de/software-pymetar.shtml}}.
  \item \textbf{Data access:} Via HTML from \emph{NOAA}'s web page (\emph{National Oceanic and Atmospheric Administration}). Various other data sources are available.
  \item \textbf{Access restrictions, terms of use:} % TODO nothing stated anywhere?
  \item \textbf{Documentation:} \emph{METAR} is well documented as it is used around the world.
  \item \textbf{Stability:} \emph{METAR} is standardized by \emph{ICAO}. It was introduced in 1968 and has been modified a numer of times since. However, most elements remained the same since introduction and can be expected to remain unchanged in the long run.
  \item \textbf{Weather elements:} Among others, temperature, dew point, wind speed and direction, precipitation, cloud coverage, visibility and atmospheric pressure are provided.
  \item \textbf{Timeframe:} Only current data, no forecasts.
  \item \textbf{Weather updates:} \emph{METAR}s are updated once per hour.
\end{itemize}

\paragraph{National Weather Service}

\begin{itemize}
  \item \textbf{Operator:} National Weather Service (part of \emph{NOAA}, \emph{National Oceanic and Atmospheric Administration})
  \item \textbf{Web page:} \href{http://graphical.weather.gov/xml/}{http://graphical.weather.gov/xml/}
  \item \textbf{Coverage area:} Only US.
  \item \textbf{Data format:} XML feed.
  \item \textbf{Data access:} Via SOAP or REST.
  \item \textbf{Access restrictions, terms of use:} % TODO
  \item \textbf{Documentation:} Full documentation is available.
  \item \textbf{Stability:} Small changes are applied every few months, although most of the interface remains unchanged in the long run. Changes are announced via an RSS feed.
  \item \textbf{Weather elements:} % TODO
  \item \textbf{Timeframe:} % TODO
  \item \textbf{Weather updates:} Weather data is updated about once per hour. 
\end{itemize}

\paragraph{Weather Underground}

\begin{itemize}
  \item \textbf{Operator:} Weather Underground
  \item \textbf{Web page:} \href{http://www.wunderground.com/weather/api/}{http://www.wunderground.com/weather/api/}
  \item \textbf{Coverage area:} Worldwide coverage.
  \item \textbf{Data format:} JSON and XML are available.
  \item \textbf{Data access:} Via HTTP.
  \item \textbf{Access restrictions, terms of use:} The public API is only available for personal, non-commercial use. Account creation is mandatory. Custom weather services are available.
  \item \textbf{Documentation:} Full documentation is available.
  \item \textbf{Stability:} There has been a change to a new API during the last months. Although it can be expected that the new API will stay there for a while, it must be expected that a change of API will always be possible.
  \item \textbf{Weather elements:} Only an overall weather condition and minimum and maximum temperatures are available. Custom weather services include more data.
  \item \textbf{Timeframe:} Daily forecasts for five days.
  \item \textbf{Weather updates:} Updated every few hours.
\end{itemize}

\paragraph{Weather.com}

\begin{itemize}
  \item \textbf{Operator:} The Weather Channel, LLC
  \item \textbf{Web page:} \href{http://www.weather.com/services/xmloap.html}{http://www.weather.com/services/xmloap.html}
  \item \textbf{Coverage area:} Worldwide coverage.
  \item \textbf{Data format:} XML or JSON.
  \item \textbf{Data access:} Via HTTP.
  \item \textbf{Access restrictions, terms of use:} For accessing the API, an account must be created in a complicated process. 
  \item \textbf{Documentation:} Full documentation is available.
  \item \textbf{Stability:} It is unknown whether the API is subject to change or not.
  \item \textbf{Weather elements:} Temperature, humidity, precipitation, wind speed and direction, sunrise and sunset times and the overall weather condition. Furthermore, the current weather report includes atmospheric pressure, visibility and dew point.
  \item \textbf{Timeframe:} Forecasts are available for five days.
  \item \textbf{Weather updates:} Updated every few hours.
\end{itemize}

\section{Conclusion}
\label{sec:weather_conclusion}

\begin{table}
  \begin{tabularx}{\textwidth}{|X|X|X|}
  \hline
  \textbf{Weather service} & \textbf{Advantage(s)} & \textbf{Disadvantage(s)} \\
  \hline \hline
  DWD & - & Data format (\emph{SYNOP}); forecasts lack important weather elements \\
  \hline
  Yahoo! Weather & Simple data format (RSS) & Low number of forecasts; forecasts lack important weather elements \\
  \hline
  World Weather Online & Wide range of data formats & Free API lacks important data \\
  \hline
  Google Weather Feed & Simple data format (XML) & Unofficial, undocumented API; unknown terms of use; only a small number of available weather elements \\
  \hline
  yr.no & Simple data format (XML); available under CC-BY 3.0 & - \\
  \hline
  METAR & - & Data format (\emph{METAR}); no forecasts \\
  \hline
  National Weather Service & Simple data format (XML) & No worldwide coverage \\
  \hline
  Weather Underground & Data formats (XML and JSON) & Public API lacks important data \\
  \hline
  Weather.com & Data formats (XML and JSON) & Complicated account creation \\
  \hline
  \end{tabularx}
  \caption{Advantages and disadvantages of Internet weather services}
  \label{table:weather_services}
\end{table}

Table \ref{table:weather_services} summarizes the main advantages and disadvantages of the weather services discussed in \ref{sec:internet_services} regarding the requirements listed there. Based on that evaluation, it is determined that \emph{yr.no} suits \thinkhome's requirements best. Therefore, it is used for developing a program that obtains weather data for a certain location and feeds it into the \thinkhomeweather ontology that is developed in section \ref{sec:ontology_specification}. See section \ref{subsec:weather_data_yr_no} for details on how to query \emph{yr.no}'s weather API.

However, the weather service shall be replaceable. The ontology is designed in a way that makes switching to another weather service simple. If \emph{yr.no} is to be replaced by another weather service, the ontology remains unchanged, but a new program for importing weather data into the ontology must be developed. If the weather service to be used does not support one or more weather elements, they are simply omitted in the program's output.

The ontology uses these weather elements which are available from most Internet weather services and many weather sensors:
\begin{itemize}
  \item Temperature,
  \item relative humidity,
  \item dew point,
  \item cloud cover,
  \item chance and intensity of precipitation,
  \item speed and direction of window,
  \item atmospheric pressure, and
  \item solar radiation.
\end{itemize}

Furthermore, an overall weather condition is added to the ontology. This weather condition is described by the following states: \emph{Cloudy}, \emph{fog}, \emph{light clouds}, \emph{partly cloudy}, \emph{rain}, \emph{sleet}, \emph{sun}, \emph{thunder}. At most three of them may be used together to describe the overall weather condition (of course, two contradictory states, e.g. \emph{light clouds} and \emph{partly cloudy}, may not be used together). These states are taken from the list of symbols used for describing the weather condition in the API of \emph{yr.no}\footnote{\href{http://api.yr.no/faq.html}{http://api.yr.no/faq.html}}.

Additionally, the ontology also supports the position of the sun. However, it is not necessary to obtain the sun's position from an internet service as there are a few algorithms available for calculating the sun's position given date, time and the geographical position (latitude and longitude); see section \ref{subsec:sun_position} for details.

To ensure the ontology can be used in the desired manner, it is necessary that the competency questions from section \ref{sec:weather_information} can be answered by the ontology using the provided input data. See table \ref{table:competency_questions} for details about which competency questions can be answered by using which weather element(s). After developing the ontology, section \ref{sec:ontology_evaluation} will discuss whether this ontology can answer the competency questions.

\begin{table}
\begin{tabularx}{\textwidth}{|X|X|}
\hline
\textbf{Competency question} & \textbf{Weather element(s)} \\
\hline \hline
What is the current weather situation? & \emph{all available elements} \\
\hline
What will the weather situation be in one hour, in two hours, …, in 24 hours? & \emph{all available elements} \\
\hline
What is the current temperature, humidity, wind speed, …? & \emph{the corresponding weather element} \\
\hline
What will be the temperature, humidity, wind speed, … in one hour, in two hours, …, in 24 hours? & \emph{the corresponding weather element} \\
\hline
What will be the minimum temperature, humidity, … over the next 24 hours? What about maximum values? & \emph{the corresponding weather element} \\
\hline
Will the weather change? Will the temperature, humidity, … rise or fall? & \emph{the corresponding weather element} \\
\hline
Does it rain? Will it rain in the next hours? Will it rain today? & Precipitation \\
\hline
Will there be sunshine today? & Cloud coverage, sun radiation, sun position \\
\hline
Do we need to irrigate the garden? & Precipitation, cloud coverage, sun radiation \\
\hline
Will there be severe weather? & Wind, precipitation, temperature \\
\hline
Will temperature drop/stay below $0^\circ C$? & Temperature \\
\hline
When can we open windows and when do we have to keep them shut? & Precipitation, wind \\
\hline
When do we need sun protection? & Sun radiation, cloud coverage, sun position \\
\hline
When will it outside be colder than inside the house? When will it be warmer? & Temperature \\
\hline
\end{tabularx}
\label{table:competency_questions}
\caption{Assignment of competency questions to weather element(s)}
\end{table}
